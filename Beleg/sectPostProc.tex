
\section{Postprocessing}

Postprocessing bezeichnet im Kontext der Videospielgrafik die Nachbearbeitung eines bereits ganz oder teilweise gerenderten Bildes mit einem oder mehreren Bildeffekten. Das Ziel des Postprocessings bei Cubix ist es, alte Videoaufnahme- und -wiedergabegeräte nachzubilden. Um diesen Effekt zu erzielen, werden Bildartefakte, die durch diese Hardware entstehen, als Nachbearbeitungseffekte dem fertigen Bild hinzugefügt.

Die verwendeten Effekte sind Bloom, Lens Flare, Chromatic Aberration, Vignette, Scanlines und Displaykrümmung.
Sie werden teils durch alte Kameralinsen und teils durch alte Monitore oder auch von beiden verursacht.
Jeder der genannten Effekte wird später genauer erläutert. Einige davon wurden aus artistischen Gründen ein wenig anders gestaltet, als sie in der Realität auftreten würden, dazu ebenfalls mehr in den entsprechenden Abschnitten.



\subsection{Postprocessing - Pipeline}

Das Postprocessing wird über ein ebenso benanntes Skript gesteuert. Bloom und Lens Flare werden von einem jeweils eigenen Compute Shader durchgeführt, alle anderen Effekte werden in eigenen Passes eines Image Effect Shaders durchgeführt. Diese sind im Skript als Objekte angelegt:

\begin{csh}
    public ComputeShader bloom;
    public ComputeShader lensFlare;
    public Material postProcMat;
\end{csh}

Weiterhin werden für den Lens Flare eine Textur einer schmutzigen Linse und eine Starburst Textur benötigt:

\begin{csh}
    public Texture2D lensDirtTex;
    public Texture2D starburstTex;
\end{csh}

Zur Sequenziellen Ausführung aller Effekte werden insgesamt 6 RenderTextures angelegt:

\begin{csh}
    public RenderTexture sourceTex;
    public RenderTexture brightTex;
    public RenderTexture blurBuff;
    public RenderTexture caResult;
    public RenderTexture lfResult;
    public RenderTexture lensTex;
\end{csh}

Zum Start des Programms oder falls die Fenstergrö{\ss}e geändert wird, müssen alle diese RenderTextures mit der Funktion createTexture() neu angelegt werden.

Durchgeführt wird das Postprocessing in der Funktion 
\begin{csh} 
private void OnRenderImage(RenderTexture source, RenderTexture destination) 
\end{csh}
Diese Funktion wird von der Unity Engine zur Verfügung gestellt und in einem Skript, das an eine Kamera angehängt ist, immer dann aufgerufen, wenn ein neues Bild gerendert wurde. Source ist dabei das Eingabebild und Destination ist das Bild, welches zum Schluss abgebildet wird. Source kann also noch beliebig verändert werden, bevor es nach Destination geschrieben wird. Der erste Effekt, welcher auf das Bild angewendet wird, ist Bloom. Bloom ist als Compute Shader verfasst, für den Texturen, in die geschrieben werden soll, das Flag 'enableRandomWrite' benötigen. Dieses lässt sich für einmal erstellte Texturen im Nachhinein nicht mehr ändern und ist in Source standardmäßig deaktiviert. Also ist der erste Schritt, mit der Funktion 
\begin{csh}
Graphics.Blit(source, sourceTex);
\end{csh}
 den Inhalt von Source in die zuvor angelegte Textur sourceTex zu schreiben, für welche das benötigte Flag gesetzt ist.

%image source to source tex

Nun kann der Bloom-Effekt angewandt werden. Zuerst werden mithilfe des Bloomshaders alle hellen Stellen aus sourceTex nach brightTex geschrieben. Der Inhalt von brightTex wird nun verschwommen gemacht, wozu die Textur blurBuff benötigt wird. Anschlie{\ss}end wird das verschwommene Resultat (brightTex) additiv zurück nach sourceTex geschrieben.

%image bloom

Als Nächstes wird Chromatic Aberration auf das fertige Bild mit Bloom (sourceTex) angewandt. Es wird keine zusätzliche Textur benötigt und das Resultat befindet sich in caResult.

%image ca

Nun ist Lens Flare an der Reihe. Als Ausgangstextur dient ebenfalls sourceTex und ähnlich wie bei Bloom werden bestimmte Features des Ursprungsbildes in eine separate Textur (lfResult) geschrieben. Diese wird dann wiederum mithilfe von blurBuff verschwommen gemacht und dieses Mal additiv auf caResult zurückgeschrieben. 

%image lf

Vorher jedoch wird das Ergebnis mit der Textur lensTex kombiniert, die zum Start des Programms einmal aus den beiden Texturen lensDirtTex und starburstTex erstellt wird.

%image lensTex

Die Effekte Vignette, Scanlines und Displaykrümmung können zuletzt alle mit einmal auf caResult angewendet werden, wobei das Ergebnis direkt nach Destination (Argument von OnRenderImage) geschrieben wird.

%image crt

Somit hat das resultierende Bild alle gewünschten Effekte.

%complete pipeline



\subsection{Blur}
\label{label:blur}

Für die Effekte Bloom und Lens Flare wird eine Methode benötigt, Texturen verschwommen zu machen, was die Aufgabe der Blur-Shaders ist.
Er ist ein Fragment Shader und wendet einen einfachen Gau{\ss}-schen Weichzeichnungsalgorithmus auf eine Textur an. Benötigt wird ein Ursprungsbild und ein Puffer der selben Grö{\ss}e.

Übergebene Variablen:
\begin{description}
\item[sampler2D MainTex] Die Eingabetextur
\item[float4 MainTexTexelSize] Die Pixelgröße der Textur in x- und y-Richtung
\item[int horizontal] Angabe, ob horizontal oder vertikal verwaschen werden soll
\end{description}

Der Algorithmus funktioniert, indem er zu einem Pixel immer auch den Wert jedes Pixels in der näheren Umgebung abfragt. Alle Werte werden dann mit unterschiedlichen Gewichtungen zusammenaddiert (je weiter vom Ursprungspixel entfernt, desto weniger Einfluss) und ergeben ein verschwommenes Abbild des Originals. Mit einer Distanz von vier Pixeln in jede Richtung würden die abgefragten Pixel wie folgt gewichtet werden:

\includegraphics[height=100pt]{gauss_normal.png}

Diese Methode hat jedoch den Nachteil, dass 81 (9x9) Texturzugriffe für jedes Pixel notwendig sind, was nicht sehr effizient ist.
Die Texturzugriffe lassen sich jedoch verringern, indem in zwei Durchläufen des Shaders die Pixeldaten zuerst nur in horizontale Richtung und dann nur in vertikale Richtung abgefragt werden:

\includegraphics[height=100pt]{gauss_horizontal.png}
\includegraphics[height=100pt]{gauss_vertical.png}

Das Resultat ist dabei fast identisch: 

\includegraphics[height=100pt]{gauss_twopass.png}

Diese Methode nennt sich two-pass Gaussian Blur und es sind hier nur 18 (2x9) Texturzugriffe notwendig. Die angelegte Variable 'horizontal' gibt bei jedem Aufruf des Shaders an, ob in die horizontale oder die vertikale Richtung verwaschen werden soll, und wird nach jedem Aufruf invertiert. Zusätzlich ist nun ein separater Puffer vonnöten (hier blurBuff), in dem das resultat des ersten Durchlaufes (horizontal) zwischengespeichert wird und dessen Inhalt im zweiten Durchlauf (vertikal) verschwommen wieder zurück in die ursprüngliche Textur geschrieben wird. Dieser Prozess ist im Postprocessing-Skript in eine eigene Blur-Methode ausgelagert worden:

\begin{csh}
private void blur(RenderTexture tex, int count)
{
    for (int i = 0; i < count; i++)
    {
        postProcMat.SetInt("_horizontal", 1);
        Graphics.Blit(tex, blurBuff, postProcMat, BlurPass);
        postProcMat.SetInt("_horizontal", 0);
        Graphics.Blit(blurBuff, tex, postProcMat, BlurPass);
    }
}
\end{csh}

Zusätzlich zur zu bearbeitenden Textur wird der Methode eine Variable 'count' übergeben, über welche die Stärke der Verwaschung reguliert werden kann. Dazu wird die Shadersequenz so oft durchlaufen, wie count angibt. Nach jedem Shaderaufruf wird die Uniform Variable 'horizontal' auf den entsprechenden Wert neu gesetzt.

Jedoch lässt sich die Anzahl der Texturzugriffe noch weiter senken. In einem Fragment Shader kann nicht nur auf diskrete Texturkoordinaten zugegriffen werden. Fragt man den Farbwert einer Textur an einer Stelle zwischen zwei Pixeln ab, so werden deren Farben linear interpoliert. In diesem Fall ermöglicht dies, das gleiche Resultat mit nur 10 (2x5) Texturzugriffen zu erzielen. Dazu werden zwei Wertelisten angelegt:
\begin{hlsl}
    static const float weight[3] = { 0.2270270270, 0.3162162162, 0.0702702703 };
    static const float offset[3] = { 0.0, 1.3846153846, 3.2307692308 };
\end{hlsl}
'Offset' gibt an, wie viele Pixel entfernt vom Hauptpixel die Farbdaten bestimmt werden sollen. Zu beachten ist hier, dass keine ganzzahligen Pixelabstände gewählt wurden, um den gewünschten Effekt zu erzielen.
'Weight' gibt zu jedem Offset an, wie stark dieser Wert in das Endergebnis einflie{\ss}t. Mit der richtigen Werteliste ist das Ergebnis identisch zu einem Algorithmus, der mit den 5 ganzzahligen Pixeloffsets 0, 1, 2, 3 und 4 arbeitet, benötigt jedoch nur reichlich die Hälfte der Texturzugriffe.

%https://john-chapman.github.io/2017/11/05/pseudo-lens-flare.html
%http://john-chapman-graphics.blogspot.com/2013/02/pseudo-lens-flare.html
%rastergrid.com/blog/2010/09/efficient-gaussian-blur-with-linear-sampling/
%https://learnopengl.com/Advanced-Lighting/Bloom



\subsection{Bloom}
\label{label:bloom}

Bloom (Überstrahlung) bezeichnet das Phänomen, dass sehr helle Bereiche eines Bildes nahe Bereiche 'überstrahlen'. Dies resultierte bei alten Kameras daraus, dass sie nicht den gesamten Helligkeitsbereich richtig aufnehmen konnten. In Videospielen und der Filmindustrie wird dieser Effekt gern verwendet, um den Eindruck von gro{\ss}er Helligkeit zu vermitteln, jedoch sollte er auch nicht zu stark sein und ablenken.
Der Algorithmus ist als Compute Shader verfasst und besteht aus zwei Schritten, die von einem jeweils eigenen Kernel ausgeführt werden. 

Übergebene Variablen:
\begin{description}
\item[RWTexture2D<float4> Source] Bildquelle
\item[RWTexture2D<float4> BrightSpots] Textur, in die Helle Stellen geschrieben werden
\item[uniform float threshold] Schwellenwert für die Helligkeit eines Pixels
\end{description}

Den ersten Schritt übernimmt das Kernel 'CSExtractBright'. Hier werden alle Pixel von Source mit der Formel $dot(float3(0.2126, 0.7152, 0.0722), Source[id.xy].rgb) > threshold$ auf ihre Helligkeit geprüft. Da unterschiedliche Lichtfarben für die menschliche Wahrnehmung unterschiedlich zur Helligkeit eines Pixels beitragen, werden die Rot-, Grün- und Blaukanäle des ursprünglichen Bildes unterschiedlich gewichtet (mit den Werten 0.2126, 0.7152 und 0.0722). Ist der Schwellenwert erreicht, werden die Pixeldaten unverändert in BrightSpots geschrieben (BrightSpots[id.xy] = Source[id.xy]), anderenfalls wird in BrigthSpots an dieser Stelle schwarz eingefügt: BrightSpots[id.xy] = float4(0, 0, 0, 1).
Somit erhält man in BrightSpots ein Abbild aller hellen Stellen eines Bildes:

\captionsetup{type=figure}
\includegraphics[height=100pt]{bloom_source.png}
\includegraphics[height=100pt]{bloom_bright.png}
\captionof{figure}{Links: Source, Rechts: BrightSpots}

Anschlie{\ss}end wird der Blur-Effekt auf BrightSpots angewendet, wodurch die hellen Stellen an den Rändern mit den Dunklen verwaschen werden. Im zweiten Schritt wird dann das verschwommene Resultat in BrightSpots vom zweiten Kernel 'CSWriteBack' additiv zurück auf das Ursprungsbild geschrieben. Dadurch wird hellen Objekten ein leichter 'Schein' auf die nahe Umgebung verliehen:

\captionsetup{type=figure}
\includegraphics[height=75pt]{bloom_off.png}
\includegraphics[height=75pt]{bloom_on.png}
\captionof{figure}{Links: Bloom aus, Rechts: Bloom an}




\subsection{Chromatic Aberration}
\label{label:chromatic aberration}

Chromatic Aberration ist ein von Kameralinsen und Röhrenmonitoren verursachtes Bildartefakt, das im Falle der Kameralinse dadurch auftritt, dass das eintreffende Licht von dieser nicht korrekt gebündelt wird. Licht unterschiedlicher Wellenlängen wird unterschiedlich stark von der Linse gebrochen, wodurch besonders kurz- oder langwellige Lichtfarben (zum Beispiel rot und blau) mit einiger Abweichung auf dem Sensor eintreffen. Normalerweise tritt Chromatic Aberration in Richtung der Ränder des Bildes stärker auf, jedoch erschwerte eine zu gro{\ss}e Effektstärke die Leserlichkeit des GUI. In der Mitte des Bildschirmes sollte der Effekt wiederum nicht zu schwach sein, da er bei der wei{\ss}en Spielfigur am besten zur Geltung kam. Deshalb fiel die Entscheidung, die Stärke der Chromatic Aberration für den gesamten Bildschirm auf einen einheitlichen Wert zu setzen.

Als Shader kann der Effekt dadurch nachgebildet werden, dass vom ursprünglichen Bild der rote und blaue Farbkanal mit einer kleinen Abweichung zur eigentlichen Pixelposition abgefragt werden und dann gemeinsam mit dem Grünkanal an der richtigen Position das neue Pixel bilden. Die Abweichung der Position erfolgt für rot und blau in jeweils entgegengesetzte Richtung in einem Fragment Shader:

Übergebene Variablen:
\begin{description}
\item[sampler2D MainTex] Die Eingabetextur
\item[float CAAmount] Abweichung für Zugriffe auf den Rot- und Blaukanal
\end{description}

\begin{hlsl}
    fixed4 col = fixed4(0, 0, 0, 1);
    col.r = tex2D(_MainTex, i.uv + float2(_CAAmount, 0)).r;
    col.g = tex2D(_MainTex, i.uv).g;
    col.b = tex2D(_MainTex, i.uv - float2(_CAAmount, 0)).b;
\end{hlsl}

Wie zu sehen ist, wird der Rotkanal der resultierenden Farbe in MainTex an der entsprechenden Texturpixelposition (i.uv) plus einer Abweichung 'CAAmount' ermittelt. Für den Blaukanal wird i.uv minus CAAmount abgefragt, beim Grünkanal gibt es keine Abweichung. Damit CAAmount als Abweichung nur die x-Position des Pixels beeinflusst, wird ein float2 erstellt, der CAAmount als x-Wert und 0 als y-Wert besitzt, welcher dann zu i.uv (auch zweidimensional) dazu addiert beziehungsweise davon abgezogen werden kann.
CAAmount kann vom Hauptprogramm aus gesteuert werden, als Standard wurde der Wert 0.0005 gewählt. Das bedeutet, dass rot und blau um ein halbes Tausendstel der Displaybreite (ein paar Pixel) von ihrer ursprünglichen Position abweichen. So sehen verschiedene Werte für CAAmount im Vergleich aus:

\captionsetup{type=figure}
\includegraphics[height=75pt]{bloom_on.png}
\includegraphics[height=75pt]{ca_0005.png}
\includegraphics[height=75pt]{ca_005.png}
\captionof{figure}{Links: kein Effekt, Mitte: CAAmount = 0.0005, Rechts: CAAmount = 0.005}

Chromatic Aberration eignet sich sich bei gro{\ss}en Abweichungswerten auch gut, um das Gefühl von Fehlern beziehungsweise Glitches zu vermitteln. So wird zum Beispiel für kurze Zeit nachdem der Spieler getroffen wurde, die Abweichung auf zufällig wechselnde, hohe Werte gesetzt, was den erlittenen Schaden gut verdeutlicht.



\subsection{Lens Flare}

Als Lens Flare werden Abbilder von hellen Bereichen eines Bildes bezeichnet, die entlang einer Linie durch dessen Zentrum auftreten. Der Effekt entsteht dadurch, dass helles Licht zwischen einer oder mehreren Kameralinsen reflektiert wird (siehe \fullref{img:lensFlareImg}\footnote{\href{https://de.wikipedia.org/wiki/Lens_Flare}{Lens Flare} Zugriff 23.06.2021}). Unter Umständen kann zusätzlich auch ein sogenannter Halo entstehen, ein blasser Ring um die Lichtquelle.

\includegraphics[height=100pt]{lensflare.png}
\caption{Lens Flare Beispiel}
\label{img:lensFlareImg}

Lens Flare wird aus artistischen Gründen oft in Filmen sowie in der Fotografie und in Videospielen emuliert. In letzterem Fall gibt es dafür grundlegend zwei Ansätze: Zum Einen der Spritebasierte Ansatz, bei dem die Abbilder (auch Ghosts genannt) der hellen Stellen als Bilder (Sprites) vorgefertigt sind, die dann entlang der Linie durch das Zentrum platziert werden. Dafür ist es notwendig, Lichtquellen, welche Lens Flare verursachen sollen, entsprechend zu markieren und dann beim Rendern des Bildes zu prüfen, ob diese Lichtquellen im Blickfeld zu sehen sind. Danach werden anhand deren Position die Ghost-Sprites auf dem Bild platziert. Der Nachteil dieser Methode ist, dass viele Informationen über die Lichtquellen notwendig sind. Au{\ss}erdem werden Bereiche, die zwar hell genug sind, aber nicht für Lens Flare markiert wurden, auch keinen Lens Flare erzeugen. Zusätzlich ist die Form der Ghosts von Beginn an festgelegt, was unter Umständen auch nicht gewünscht ist.

Eine Alternative bildet der Ansatz des Screen Space Lens Flare. Dabei werden dynamisch alle Bereiche eines Bildes extrahiert, deren Helligkeit einen gewissen Schwellenwert übersteigt und danach automatisch entsprechende Ghosts erzeugt. Dieser Ansatz ist wesentlich Flexibler, da keine anderen Informationen als das Bild selbst notwendig sind, er ist jedoch auch rechenintensiver und es lassen sich bestimmten Lichtquellen nicht einfach ausschlie{\ss}en.

Cubix verwendet den zweiten Ansatz. Lens Flare ist hier als Compute Shader mit zwei Kernels verfasst, welche ähnliche Aufgaben wie die des \nameref{label:bloom} Effekts haben. 

Übergebene Variablen:
\begin{description}
\item[RWTexture2D<float4> Source] Die Eingabetextur
\item[RWTexture2D<float4> Result] Textur für die entstandenen Ghosts und den Halo
\item[Texture2D<float4> lensTex] Transparente Textur einer dreckigen Linse
\item[uniform int ghostCount] Anzahl der Ghosts pro hellem Bereich
\item[uniform float ghostSpacing] Abstand der Ghosts zueinander
\item[uniform float threshold] Schwellenwert der Helligkeit, ab dem Ghosts/Halo produziert werden
\item[uniform float caStrength] Stärke der Chromatic Aberration der Ghosts/des Halo
\end{description}

Der Algorithmus läuft in zwei Schritten ab: Zuerst werden für helle Stellen im Bild die Ghosts und der Halo generiert, anschlie{\ss}end werden mittels \nameref{label:blur} die entstandenen Features verschwommen gemacht und unter Einbezug der Textur lensTex auf das Ursprungsbild zurückgeschrieben.

Für den ersten Schritt wurde im Shader der Kernel 'CSFlare' angelegt. Texturzugriffe erfolgen hier mithilfe der übergebenen Variable 'id', deren Werte sind diskrete Pixelkoordinaten. Damit später mit ihnen einfacher zu rechnen ist, müssen sie auf Werte zwischen 0 und 1 transformiert werden. Dies erfolgt über das Teilen durch die grö{\ss}e der Textur in Pixeln, welche mit der Funktion GetDimensions ermittelt werden kann:

\begin{hlsl}
uint width, height;
Source.GetDimensions(width, height);
float2 iResolution = float2(width, height); // Size of Source
float2 texcoord = (id.xy / iResolution); //transform coordinates to [0, 1]
\end{hlsl}

Texcoord ist nun also ein Vektor mit einem Wertebereich von 0 bis 1, der auf das derzeitige Pixel zeigt. Um nun wieder auf die Texturen zugreifen zu können, muss dieser Vektor mit deren Grö{\ss}e multipliziert werden:

\begin{hlsl}
Source[texcoords * iResolution];
\end{hlsl}

Sehr wichtig ist auch der Vektor 'ghostVec'. Dieser zeigt vom derzeitigen Pixel in Richtung Mitte des Bildes und wird wie folgt berechnet:

$ghostVec = ( -texcoord + .5 ) * ghostSpacing$

GhostSpacing ist eine der Uniform Variablen und gibt an, um wie viel länger beziehungsweise kürzer ghostVec sein soll als der Abstand von texcoord zur Bildmitte (siehe \fullref{img:lfghostvec})

\captionsetup{type=figure}
\includegraphics[height=150pt]{lf_ghostvec.png}
\captionof{figure}{ghostVec}
\label{img:lfghostvec}

Nun werden die Ghosts und der Halo jeweils in ihrer eigenen Funktion generiert und zu einem Ergebnis zusammenaddiert:

\begin{hlsl}
float3 col = 0;
col += generateGhosts(texcoord, ghostVec, iResolution);
col += generateHalo(texcoord, ghostVec, iResolution);
\end{hlsl}

Zuerst werden mit 'generateGhosts', wie der Name schon sagt, die Ghosts generiert. Eine Schleife, die an unterschiedlichen Stellen des Bildes den entsprechenden Helligkeitswert überprüft, läuft so oft ab, wie durch den übergebenen Parameter ghostCount angegeben. Dabei wird in jedem Durchlauf einmal der Vektor ghostVec als Abweichung zu den derzeitigen Texturkoordinaten hinzuaddiert. An der resultierenden Stelle wird dann die Farbe des Ursprungsbildes bestimmt und der Schwellenwert wird von ihr abgezogen: $texColor = max(texColor - threshold, 0)$

Zuletzt wird der Wert von texColor mit einem Faktor multipliziert, der davon abhängig ist, wie weit die helle Stelle vom Zentrum des Bildes entfernt ist, da Lichtquellen am Rand weniger bis gar kein Lens Flare auslösen sollen. \fullref{img:lfghostgen} zeigt, wie die Berechnung der Koordinaten der hellen Stellen für jeden Ghost stattfindet.

\captionsetup{type=figure}
\includegraphics[height=150pt]{lf_ghostgen_1.png}
\includegraphics[height=150pt]{lf_ghostgen_2.png}
\includegraphics[height=150pt]{lf_ghostgen_3.png}
\captionof{figure}{Berechnung der Position der hellen Stelle für Ghosts (rot umrandet sei ein heller Bereich)}
\label{img:lfghostgen}

Wie zu sehen ist, wird ghostVec kleiner, je näher der Ghost dem Zentrum des Bildes ist. Mit jeder Iteration entfernt sich Offset einmal um die Länge von ghostVec von den ursprünglichen Koordinaten. Deshalb muss die Schleife für jeden Ghost einmal öfter durchlaufen werden, um an jeweils die gleichen Offset-Koordinaten zu gelangen. Für ghostCount war in diesem Beispiel 3 eingestellt.

Somit ist die Erstellung der Ghosts abgeschlossen und es folgt der Halo. Die Funktion 'generateHalo' arbeitet grundsätzlich sehr ähnlich zu 'generateGhosts'. Die Unterschiede bestehen darin, dass es keine Schleife gibt (zu jedem Objekt existiert immer nur ein Halo) und dass ghostVec auf eine feste Länge normalisiert und nun haloVec genannt wird: $haloVec = normalize(ghostVec) * 0.35$

Der Faktor 0.35 legt den Radius des Halo fest (auf 0.35 Bildbreiten/-höhen). HaloVec zeigt genau wie ghostVec in Richtung Mitte des Bildes, hat aber eine feste Länge. Dadurch bilden sich um jede Stelle des Bildes teile eines Ringes, in denen haloVec auf besagte Stelle zeigt. Besser veranschaulicht ist der Sachverhalt in \fullref{img:lfhalogen}

\captionsetup{type=figure}
\includegraphics[height=150pt]{lf_halogen_1.png}
\includegraphics[height=150pt]{lf_halogen_2.png}
\captionof{figure}{Berechnung der Stellen, die Teil des Halo sind (rot umrandet sei ein heller Bereich)}
\label{img:lfhalogen}

Je näher die helle Stelle dem Zentrum ist, desto grö{\ss}er wird der Halo, bis er sich gänzlich schlie{\ss}t, falls sie sich genau dort befindet.

Nun sind alle Features erstellt, die für den Lens Flare notwendig sind. Sie befinden sich in der Textur Result und müssten noch per \nameref{label:blur} verschwommen gemacht werden. Jedoch lässt sich die Qualität der Features noch verbessern: Da Licht unterschiedlicher Wellenlängen an der Linse unterschiedlich stark gebrochen wird, haben auch deren Ghosts unterschiedliche Abstände. Es wird hier also ähnlich vorgegangen wie bei \nameref{label:chromatic aberration}, die Rot- und Blaukanäle werden mit einer gewissen Abweichung zu den eigentlichen Koordinaten abgefragt. Der Unterschied besteht darin, dass die Abweichung richtung Zentrum erfolgt und deren Stärke anhand der Entfernung davon skaliert wird. Dies führt dann zu folgendem Unterschied in Result:

\captionsetup{type=figure}
\includegraphics[height=100pt]{lf_ca_off.png}
\includegraphics[height=100pt]{lf_ca_on.png}
\captionof{figure}{Links: ohne Chromatic Aberration, Rechts: mit Chromatic Aberration}

Erzielt wird dieser Effekt mithilfe der Funktion 'float3 getTexColor(float2 texcoords, float2 iResolution)'. Diese wird immer dort aufgerufen, wo ein Texturzugriff auf Source erfolgen würde. Für den Zugriff auf den Rotkanal wird ein Offset zu den Koordinaten addiert, beim Blaukanal wird Offset abgezogen:

\begin{hlsl}
return float3(
	Source[(texcoords + offset) * iResolution].r,
	Source[texcoords * iResolution].g,
	Source[(texcoords - offset) * iResolution].b
	);
\end{hlsl}

Die Richtung von Offset entspricht der Richtung von texcoord zum Zentrum, die Länge ergibt sich aus der Entfernung von texcoord zum Zentrum und der übergebenen Variable caStrength, die vom Skript aus steuerbar ist:

\begin{hlsl}
    float amount = length(.5 - texcoords) * caStrength;
    float2 offset = normalize(.5 - texcoords) * amount;
\end{hlsl}

Nun ist der erste Schritt abgeschlossen und Result kann (nachdem es verschwommen gemacht wurde) auf die Ausgangstextur zurückgeschrieben werden. Jedoch soll das Ergebnis nicht von der normalen \nameref{label:chromatic aberration} beeinflusst werden, weshalb Source für das Zurückschreiben zuvor auf das Resultat des Chromatic Aberration Effekts geändert wird. Dieser Schritt wird ähnlich wie bei Bloom von einem eigenen Kernel 'CSWriteBack' übernommen:

\captionsetup{type=figure}
\includegraphics[height=100pt]{lf_ca_on.png}
\includegraphics[height=100pt]{lf_writeback_2.png}
\captionof{figure}{Links: Lens Flare Features, Rechts: Resultat}

So sieht der Lens Flare schon fast wie im fertigen Spiel aus, es fehlen nur noch die Streifen und Unreinheiten auf der Linse. Dafür wird die Textur lensDirt verwendet. Diese wird zu Beginn des Spiels aus einer fertigen Textur mit Linsendreck und einem kreisförmig gerenderten Barcode durch einen Fragment Shader generiert:

\captionsetup{type=figure}
\includegraphics[width=1\linewidth]{lf_lensgen.png}
\captionof{figure}{Links: Linsendreck, Mitte: Barcode, Rechts: Resultat}

Dieses Bild wird im Writeback Kernel des Lens Flare multiplikativ mit den Ghosts und dem Halo kombiniert, bevor diese auf die Ausgangstextur zurückgeschrieben werden (siehe \fullref{img:lfcubix}). Der Lens Flare Effekt ist somitvollständig.

\captionsetup{type=figure}
\includegraphics[height=150pt]{lf_cubix.png}
\captionof{figure}{Lens Flare in Cubix}
\label{img:lfcubix}

%https://docs.unrealengine.com/4.26/Images/RenderingAndGraphics/PostProcessEffects/Bloom/DirtMaskTextureExample.png


\subsection{Vignette, warped Display und Scanlines}

Die drei Effekte Vignette, Displaykrümmung und Scanlines werden unter anderem von alten CRT-Monitoren verursacht. Vignette bezeichnet eine Abdunklung verschiedener Bildbereiche (meist der Ränder). Scanlines sind (meist horizontale) dunkle Linien, die dadurch entstehen, dass das Bild vom Monitor zeilenweise gezeichnet wird. Auch hatten diese Monitore gewölbte Displays, deren Effekt ebenfalls emuliert wird.
%link CRT

Sie sind in einer Sektion zusammengefasst, da sie in einem einzigen Fragment Shader Pass gemeinsam durchgeführt werden können.

Übergebene Variablen:
\begin{description}
\item[sampler2D MainTex] Die Eingabetextur
\item[float vignetteAmount] gibt an, wie stark die Ränder abgedunkelt werden
\item[float vignetteWidth] gibt an, wie weit die Verdunklung in Richtung Bildmitte vordringt
\end{description}

Zuerst wird der Warped Display Effekt durchgeführt. Dafür werden die UV-Koordinaten, welche Werte von 0 bis 1 annehmen können, zuerst in einen Wertebereich zwischen -1 und 1 transformiert: $uv = i.uv * 2 - 1$.
Anschlie{\ss}end wird mit folgenden Formeln der x-Wert der Koordinaten gestreckt: $uv.x = uv.x * (1 + pow(abs(uv.y) / 8, 2))$

Das lässt sich so verstehen, dass der x-Wert für gro{\ss}e Absolutwerte von y (also weit oben und unten im Bild) besonders gestreckt wird. Es werden so bei der Erstellung des Bildes schon vor den Rändern Koordinaten erreicht, die über den Wertebereich der ursprünglichen Textur hinausgehen:

\includegraphics[height=100pt]{warp_x.png}

Diese Bereiche bleiben schwarz.

Für die y-Werte wird analog verfahren: $uv.y = uv.y * (1 + pow(abs(uv.x) / 8, 2))$

\includegraphics[height=100pt]{warp_y.png}

Nun werden dem Bild noch Scanlines und Vignette hinzugefügt. Da die Scanlines vertikal angeordnet sind, kann deren Stärke als Sinusfunktion der y-Koordinate dargestellt werden und wird wie folgt berechnet: $sin(texcoord.y * 6.28 * 100) / 2 + 0.5$ 

Die Faktoren 6.28 und 100 bedeuten, dass 100 Periodendurchläufe der Sinusfunktion (6.28 ist rund 2 Pi, also eine Periode) für y-Werte zwischen 0 und 1 geschehen. Da sich die Koordinaten in einem Wertebereich von -1 bis 1 befinden, entstehen auf der gesamten Displayhöhe insgesamt 200 Scanlines. Das Ergebnis der Sinusfunktion wird halbiert und es werden 0.5 addiert, womit es in einen Wertebereich zwischen 0 und 1 gebracht wird. Zuletzt wird der erhaltene Wert als Faktor mir der Farbe des Bildes multipliziert. 

Da starke Scanlines besonders in der Mitte des Bildschirmes die Sichtbarkeit erheblich beeinträchtigen können, wird eine zusätzliche Variable 'scanLineIntensity' angelegt, die als Faktor mit in das Ergebnis einflie{\ss}t und die Intensität der Scanlines entsprechend verringert. Dazu wird die Funktion 'smoothstep' verwendet: $scanLineIntensity = smoothstep(.8, 1.41422, length(texcoord))$. Somit erhält man ein Ergebnis von 0 (also keine Scanlines) für alle Bereiche, die weniger als 0.8 vom Zentrum entfernt sind, danach steigt die Intensität linear bis auf einen Wert 1 in den Ecken. Es entsteht folgendes Resultat:

\includegraphics[height=100pt]{scanlines.png}

Zuletzt erfolgt die Vignettierung der Ränder. Dabei wird wie bei der Displaykrümmung mit den Absolutwerten der Koordinaten gearbeitet, da die Vignette an allen Rändern symmetrisch auftritt. Wieder kommt smoothstep zum Einsatz. Es wird geprüft, ob der grö{\ss}ere der beiden Koordinatenwerte (mit 'max' bestimmt) nicht weiter als der übergebene Wert vignetteWidth vom Rand entfernt ist: $vignetteStrength = smoothstep(1 - vignetteWidth, 1, max(texcoord.x, texcoord.y))$

Damit der Effekt zur Mitte hin schneller abnimmt, wird das Ergebnis anschlie{\ss}end hoch 4 gerechnet, au{\ss}erdem wird der erhaltene Wert mit der zweiten Uniform Variable 'vignetteAmount' multipliziert, um die Stärke des Effekts vom Kontrollskript aus steuern zu können. Zusätzlich muss das Resultat mit 1 invertiert werden, um richtig mit den Farbwerten multipliziert werden zu können: $vignette = 1 - (pow(vignetteStrength, 4) * vignetteAmount)$
Für vignetteWidth gleich 0.1 und vignetteAmount gleich 0.8 sieht das Ergebnis wie folgt aus:

\includegraphics[height=100pt]{vignette.png}

