\documentclass[a4paper,ngerman,12pt]{report}
\usepackage[a4paper,left=2.5cm,right=2cm,top=1.8cm,bottom=1.5cm,includefoot,head=18pt]{geometry}

\usepackage{hyphsubst} %[ngerman=ngerman-x-latest]
\usepackage[onehalfspacing]{setspace}            % Zeilenabstand 1.5
\usepackage[ngerman]{babel}               % Sprache


%\usepackage{step} %fbb,step
%\usepackage{newtxmath,newtxtext}                % pdflatex Times New Roman
%\usepackage[utf8x]{inputenc}
\usepackage[T1]{fontenc}

%\usepackage{blindtext}
%\usepackage{enumitem}

\usepackage{footnote}
\usepackage{longtable}
\usepackage[font={footnotesize,it}]{caption} % scriptsize
\captionsetup{singlelinecheck=true,justification=centering} % justification=raggedright,

%\usepackage[Q=yes]{examplep}
%\usepackage{algpseudocode}        % Algorithmen
%\usepackage{amsmath}              % mathematische Befehle
%\usepackage{amssymb}              % mathematische Symbole
%\usepackage{arev}                 % monospace for Codeblocks
\usepackage{color}                % Farben
\usepackage{subfig}
\usepackage{float}
%\usepackage{fontspec}             % Schriftart
%\usepackage{glossaries}           % Glossar
\usepackage{graphicx}             % Bilder
\usepackage{listings}             % Quelltext
\usepackage{multirow}             %
\usepackage{colortbl}			  % hline color
%\usepackage{lmodern}              % Beliebige Schriftgrößen
\usepackage{mdframed}
\usepackage{scrlayer-scrpage}     % Kopf- und Fußzeilen
\PassOptionsToPackage{hyphens}{url}

\usepackage[allcolors=blue]{hyperref}             % Links & Verweise (last)
\usepackage[nameinlink]{cleveref}             % Verweise         (last > hyperref)

\usepackage{lipsum}

% Link- / Verweiseinstellungen
\hypersetup{
    colorlinks,breaklinks,
    % urlcolor=[rgb]{0,0,1},
    linkcolor=[rgb]{0,0.1,0.6}
}

\renewcommand\citeform[1]{[#1]}

%\renewcommand{\ttdefault}{pcr}
%\renewcommand{\normalsize}{\fontsize{12}{0.231in}\selectfont}
%\renewcommand*\familydefault{\sfdefault}
%\setmainfont[Ligatures=TeX]{Times New Roman}     % Schriftart Times New Roman

% \setkomafont{pageheadfoot}{\textrm}
\cfoot{\thepage}                  % Fußzeile Zentriert = Seitennummer
\def\UrlFont{\em}                 % italic URL

\newcommand*{\fullref}[1]{\hyperref[{#1}]{\autoref*{#1} \nameref*{#1}}}
\newcommand*{\bitem}[1]{ \setbox0\hbox{\bfseries{#1}} \item[\usebox0]}
\newcommand*{\hitem}[1]{ \setbox0\hbox{#1} \item[\usebox0] \hfill \\}
\newcommand*{\hbitem}[1]{ \setbox0\hbox{\bfseries{#1}} \item[\usebox0] \hfill \\}
\newcommand\rurl[1]{%
	\href{http://#1}{\nolinkurl{#1}}%
}
\newcommand{\savefootnote}[2]{\footnote{\label{#1}#2}}
\newcommand{\repeatfootnote}[1]{\textsuperscript{\ref{#1}}}

\newenvironment{mldescription}{
    \begin{addmargin}[1em]{1em}
    \setlength{\parindent}{-1em}
    \newcommand*{\mlitem}[1]{\par{##1\hfill\\}\quad}\indent
}{
    \end{addmargin}
    \medskip
}


\newcommand{\sq}[1]{`#1'}     % single quote
\renewcommand{\dq}[1]{``#1''} % double quote
\newcommand{\mono}[1]{\lstinline[breaklines=true]{#1}}
\newcommand{\inlind}{\vspace{-.65cm}\hspace{.7cm}}
\newcommand{\greyline}{
	\arrayrulecolor{cllgrey}
	%\setlength{\arrayrulewidth}{.3em}
	\hline
	%\setlength{\arrayrulewidth}{.1em}
}
\newcommand{\dgreyline}{
	\arrayrulecolor{clgrey}
	%\setlength{\arrayrulewidth}{.3em}
	\hline
	%\setlength{\arrayrulewidth}{.1em}
}
\newcommand{\cross}[1]{\includegraphics[height=#1]{cross.png}}
\newcommand{\crosss}[1]{\includegraphics[height=#1]{cross.png}\hspace{2mm}}


\newcommand{\itmspace}{5em}
\newcommand{\imgspace}{3em}

\newcommand\imgtxt[2]{%
\begin{flushleft}%
\parbox{\imgspace}{\includegraphics[width=\linewidth]{#1}}%
\hfil
\parbox{\dimexpr\textwidth-\imgspace-1em}{#2}%
\end{flushleft}}

\newcommand\imgitmtxt[3]{%
\begin{flushleft}%
\parbox{\imgspace}{\includegraphics[width=\linewidth]{#1}}%
\hspace{2mm}
\parbox{\itmspace}{\textbf{#2}}%
\parbox{\dimexpr\textwidth-\imgspace-\itmspace-1em}{#3}%
\end{flushleft}}


% Benutzerdefinierte Farben
\definecolor{clgreen}{rgb}{0,0.6,0}
\definecolor{clgrey}{gray}{0.5}
\definecolor{clmauve}{rgb}{0.58,0,0.82}
\definecolor{clorange}{rgb}{1.0, 0.49, 0.0}
\definecolor{clgray}{gray}{0.95}
\definecolor{cllgrey}{gray}{0.75}

\lstset{
    basicstyle=\selectfont\sffamily,
    % postbreak=\mbox{\textcolor{red}{$\hookrightarrow$}\space},
    captionpos=b, keepspaces=true, numberfirstline=false,
    numbers=left, numbersep=9pt, showspaces=false, showstringspaces=false,
    numberstyle=\tiny\color{clgrey},
    showtabs=false, stepnumber=5, tabsize=4, title=\lstname, firstnumber=1
    , literate={ß}{{\ss}}1 {€}{{\euro}}1 {£}{{\pounds}}1
    {ä}{{\"a}}1 {ö}{{\"o}}1 {ü}{{\"u}}1 {Ä}{{\"A}}1 {Ö}{{\"O}}1 {Ü}{{\"U}}1
}

\lstdefinestyle{codecolors}{
    keywordstyle=\color{blue}, commentstyle=\color{clgreen}, stringstyle=\color{clmauve}, rulecolor=\color{black},
    basicstyle=\fontsize{9.8}{11}\selectfont\ttfamily
}

\lstdefinestyle{codeblock}{
    columns=fullflexible, frame=tlbr, framesep=0pt, framerule=0pt,
    breaklines=true, aboveskip=10pt, belowskip=-5pt,
    backgroundcolor=\color{clgray}
}

\lstdefinestyle{codeblock1}{
    columns=fullflexible, frame=tlbr, framesep=0pt, framerule=0pt,
    breaklines=true, aboveskip=5pt, belowskip=-30pt
}

\lstdefinestyle{codeblock2}{
    columns=fullflexible, frame=tlbr, framesep=5pt, framerule=0pt,
	breaklines=true, aboveskip=-10pt, belowskip=-15pt, backgroundcolor=\color{clgray}
}
\lstdefinelanguage{csharp}{
    language=[sharp]c,
    morekeywords={
        % Shader keywords
        Shader,SubShader,Pass,Properties,
        CGPROGRAM,ENDCG
    },
    moredelim=*[directive]\#,
    moredirectives={region,endregion}%
}[keywords,comments,strings,directives]%

\lstdefinestyle{style_csh}{
    language=csharp, style=codecolors,
%    morecomment=[l]{\#},
    classoffset=1,
    morekeywords={
        % Custom enums
        Shape,
        % Custom classes
        Particles,
        % Custom structs
        Stats,GeneralProps,RenderSettings,DynamicEffect,Colors,
%
        % C# types
        Serializable,
%
        % Unity namespaces
        Application,Mathf,Graphics,UnityEditor,
        % Unity types
        MonoBehaviour,ComputeShader,ComputeBuffer,Material,Texture,Texture2D,Gradient,
        % Shader Settings
        Tags,Blend,BlendOp,Lighting,ZWrite,2D,
        % Vector types
        Vector2,Vector3,Vector3Int,Vector4,Color,
        % Attribute types
        Header,Tooltip,NotNull,WarnNull,Range,HideInInspector,MainTexture
    },
    keywordstyle=\color{clgreen},
}[directives]

\newcommand{\cshinline}[1]{\lstinline[style=style_csh]{#1}}

\newcommand{\cshmlinline}[1]{
    \lstinline[style=style_csh,style=codeblock]{#1}
}

\lstnewenvironment{csh}[1][]{\lstset{style=style_csh, style=codeblock1, #1}}{}
\lstnewenvironment{csh2}[1][]{\lstset{style=style_csh, style=codeblock2, #1}}{}

\newcommand*{\cshitem}[1]{
  \setbox0\hbox{\cshinline{#1}}
  \item[\usebox0] \hfill \\
}

\newcommand*{\cshmlitem}[1]{
	\setbox0\hbox{\cshmlinline{#1}}
	\item[\usebox0] \hfill \\
}

\surroundwithmdframed[
    hidealllines=true,
    skipabove=12pt,
    skipbelow=16pt,
    backgroundcolor=clgray,
    innerleftmargin=5pt,
    innertopmargin=10pt,
    innerbottommargin=0pt
]{csh}
%%
%% GLSL definition (c) 2020 Benno Bielmeier
%%
\lstdefinelanguage{hlsl}%
{%
    language=c,
	morekeywords={%
	% HLSL constants
		false,FALSE,NULL,true,TRUE,%
	% GLSL predefinde macro constant
		__LINE__,__FILE__,__VERSION__,GL_core_profile,GL_es_profile,GL_compatibility_profile,%
	% GLSL precision modifier
		precision,highp,mediump,lowp,%
	% GLSL control keywords
	% GLSL types
		uint,vec2,vec3,vec4,dvec2,dvec3,dvec4,bvec2,bvec3,bvec4,ivec2,ivec3,ivec4,uvec2,uvec3,uvec4,mat2,mat3,mat4,mat2x2,mat2x3,mat2x4,mat3x2,mat3x3,mat3x4,mat4x2,mat4x3,mat4x4,dmat2,dmat3,dmat4,dmat2x2,dmat2x3,dmat2x4,dmat3x2,dmat3x3,dmat3x4,dmat4x2,dmat4x3,dmat4x4,sampler1D,sampler2D,sampler3D,image1D,image2D,image3D,samplerCube,imageCube,sampler2DRect,image2DRect,sampler1DArray,sampler2DArray,image1DArray,image2DArray,samplerBuffer,imageBuffer,sampler2DMS,image2DMS,sampler2DMSArray,image2DMSArray,samplerCubeArray,imageCubeArray,sampler1DShadow,sampler2DShadow,sampler2DRectShadow,sampler1DArrayShadow,sampler2DArrayShadow,samplerCubeShadow,samplerCubeArrayShadow,isampler1D,isampler2D,isampler3D,iimage1D,iimage2D,iimage3D,isamplerCube,iimageCube,isampler2DRect,iimage2DRect,isampler1DArray,isampler2DArray,iimage1DArray,iimage2DArray,isamplerBuffer,iimageBuffer,isampler2DMS,iimage2DMS,isampler2DMSArray,iimage2DMSArray,isamplerCubeArray,iimageCubeArray,atomic_uint,usampler1D,usampler2D,usampler3D,uimage1D,uimage2D,uimage3D,usamplerCube,uimageCube,usampler2DRect,uimage2DRect,usampler1DArray,usampler2DArray,uimage1DArray,uimage2DArray,usamplerBuffer,uimageBuffer,usampler2DMS,uimage2DMS,usampler2DMSArray,uimage2DMSArray,usamplerCubeArray,uimageCubeArray,%
	% HLSL types
		int2,int3,int4,uint2,uint3,uint4,float2,float3,float4,
	% GLSL support variables
		gl_BackColor,gl_BackLightModelProduct,gl_BackLightProduct,gl_BackMaterial,gl_BackSecondaryColor,gl_ClipDistance,gl_ClipPlane,gl_ClipVertex,gl_Color,gl_DepthRange,gl_DepthRangeParameters,gl_EyePlaneQ,gl_EyePlaneR,gl_EyePlaneS,gl_EyePlaneT,gl_Fog,gl_FogCoord,gl_FogFragCoord,gl_FogParameters,gl_FragColor,gl_FragCoord,gl_FragData,gl_FragDepth,gl_FrontColor,gl_FrontFacing,gl_FrontLightModelProduct,gl_FrontLightProduct,gl_FrontMaterial,gl_FrontSecondaryColor,gl_InstanceID,gl_Layer,gl_LightModel,gl_LightModelParameters,gl_LightModelProducts,gl_LightProducts,gl_LightSource,gl_LightSourceParameters,gl_MaterialParameters,gl_ModelViewMatrix,gl_ModelViewMatrixInverse,gl_ModelViewMatrixInverseTranspose,gl_ModelViewMatrixTranspose,gl_ModelViewProjectionMatrix,gl_ModelViewProjectionMatrixInverse,gl_ModelViewProjectionMatrixInverseTranspose,gl_ModelViewProjectionMatrixTranspose,gl_MultiTexCoord0,gl_MultiTexCoord1,gl_MultiTexCoord2,gl_MultiTexCoord3,gl_MultiTexCoord4,gl_MultiTexCoord5,gl_MultiTexCoord6,gl_MultiTexCoord7,gl_Normal,gl_NormalMatrix,gl_NormalScale,gl_ObjectPlaneQ,gl_ObjectPlaneR,gl_ObjectPlaneS,gl_ObjectPlaneT,gl_Point,gl_PointCoord,gl_PointParameters,gl_PointSize,gl_Position,gl_PrimitiveIDIn,gl_ProjectionMatrix,gl_ProjectionMatrixInverse,gl_ProjectionMatrixInverseTranspose,gl_ProjectionMatrixTranspose,gl_SecondaryColor,gl_TexCoord,gl_TextureEnvColor,gl_TextureMatrix,gl_TextureMatrixInverse,gl_TextureMatrixInverseTranspose,gl_TextureMatrixTranspose,gl_Vertex,gl_VertexID,%
	% HLSL support variables
        SV_DispatchThreadID,
	% GLSL support constants
		gl_MaxClipPlanes,gl_MaxCombinedTextureImageUnits,gl_MaxDrawBuffers,gl_MaxFragmentUniformComponents,gl_MaxLights,gl_MaxTextureCoords,gl_MaxTextureImageUnits,gl_MaxTextureUnits,gl_MaxVaryingFloats,gl_MaxVertexAttribs,gl_MaxVertexTextureImageUnits,gl_MaxVertexUniformComponents,%
	% HLSL support constants
		SV_VertexID,SV_INSTANCEID,UNITY_MATRIX_P,UNITY_MATRIX_V,
		TEXCOORD0,SV_POSITION,COLOR,SV_Target,
	% GLSL support functions
		abs,acos,all,any,asin,atan,ceil,clamp,cos,cross,degrees,dFdx,dFdy,distance,dot,equal,exp,exp2,faceforward,floor,fract,ftransform,fwidth,greaterThan,greaterThanEqual,inversesqrt,length,lessThan,lessThanEqual,log,log2,matrixCompMult,max,min,mix,mod,noise1,noise2,noise3,noise4,normalize,not,notEqual,outerProduct,pow,radians,reflect,refract,shadow1D,shadow1DLod,shadow1DProj,shadow1DProjLod,shadow2D,shadow2DLod,shadow2DProj,shadow2DProjLod,sign,sin,smoothstep,sqrt,step,tan,texture1D,texture1DLod,texture1DProj,texture1DProjLod,texture2D,texture2DLod,texture2DProj,texture2DProjLod,texture3D,texture3DLod,texture3DProj,texture3DProjLod,textureCube,textureCubeLod,transpose,%
	% HLSL support functions
        numthreads,frac,round,tex2D
	},
	sensitive=true,%
	morecomment=[s]{/*}{*/},%
	morecomment=[l]//,%
	morestring=[b]",%
	morestring=[b]',%
	moredelim=*[directive]\#,%
	% keyword.control.hlsl
	moredirectives={define,defined,elif,else,if,ifdef,endif,line,error,ifndef,include,pragma,undef,warning,extension,version}%
}[keywords,comments,strings,directives]%

\lstdefinestyle{style_hlsl}{
    language=hlsl, style=codecolors,
%    morecomment=[l]{\#},
    classoffset=1,
    morekeywords={
        % Custom types
        Particle,
        % HLSL types
        Texture2D,RWStructuredBuffer,AppendStructuredBuffer,ConsumeStructuredBuffer,%
    },
    keywordstyle=\color{clgreen}
}

\newcommand{\hlslinline}[1]{\lstinline[style=style_hlsl]{#1}}

\newcommand{\hlslmlinline}[1]{
    \lstinline[style=style_hlsl,style=codeblock]{#1}
}

\lstnewenvironment{hlsl}[1][]{\lstset{style=style_hlsl, style=codeblock1, #1}}{}
\lstnewenvironment{hlsl2}[1][]{\lstset{style=style_hlsl, style=codeblock2, #1}}{}

\newcommand*{\hlslitem}[1]{
  \setbox0\hbox{\hlslinline{#1}}
  \item[\usebox0] \hfill \\
}

\newcommand*{\hlslmlitem}[1]{
	\setbox0\hbox{\hlslmlinline{#1}}
	\item[\usebox0] \hfill \\
}

\surroundwithmdframed[
    hidealllines=true,
    skipabove=12pt,
    skipbelow=16pt,
    backgroundcolor=clgray,
    innerleftmargin=5pt,
    innertopmargin=10pt,
    innerbottommargin=0pt
]{hlsl}

\graphicspath{{img/}{img/coll/}{img/enem/}{img/shot/}{img/particles/}{img/postproc/}}

% Daten für Titelseite
\title{Kombinierte Belegarbeit \\ GameProgramming und \\ GPU Programmierung}
\author{Alexander Feilke, Jakob Ma{\ss}alsky \\ Sem.Gr.: IF19wS-B}
%\date{}


%help: https://de.wikibooks.org/wiki/LaTeX-Wörterbuch
% \part \chapter \section \subsection \subsubsection \paragraph \subparagraph
% ß: {\ss}
% label{X} \nameref \pageref \fullref \vref {X}
% href{URL}{text}
% für Lit.Verz.: \cite[extra]{X,Y}



% Glossaries:
%\usepackage[chapter,toc]{glossaries}
%\newcommand{\printmyglossary}{\printunsrtglossaries}
%\newcommand{\printmyglossary}{\printglossaries}
%\makeglossaries % use TeX to sort
%\renewcommand{\glstextformat}[1]{\textit{#1}}
%\newglossaryentry{Basisklasse}{name={Basisklasse},
%description={Vererbt Eigenschaften an davon abgeleitete Klassen}}




\begin{document}

%\thispagestyle{empty}\quad\newpage % leeres Deckblatt
%\selectfont\sffamily
\normalfont

\maketitle
\newpage % Titelblatt

\renewcommand{\contentsname}{Inhaltsverzeichnis}
\tableofcontents
\newpage


\lefthyphenmin 5
\righthyphenmin 5
\sloppy
\flushbottom



\chapter{Aufgabenstellung}

\lipsum[3]





\chapter{Spielbeschreibung}

%Konzept, ~farben
Cubix ist ein Top-Down Bullet-Hell Shooter angelehnt an den Hacking-Mode aus Nier:Automata\cite{qNierHM}.
Dort steuert man eine Figur in einer Welt von Gegnern und Hindernissen und muss möglichst schnell und Fehlerfrei sämtliche Gegner durch gekonntes Ausweichen und gezielte Schüsse eliminieren.

Als Neuerung wird ein Farbkonzept verwendet, bei dem die Grundfarben Rot, Grün und Blau mit verschiedenen Eigenschaften assoziiert werden. Alle Akteure des Spiels (sowohl der Spieler selbst als auch die Gegner und aufsammelbare Objekte) können in Beliebiger Kombination dieser Farben auftreten. Ist mehr als eine Farbe aktiv, so addieren sich deren Eigenschaften. Dieses System ist für den Spieler leicht verständlich, denn das Wissen über die Eigenschaften der drei Grundfarben reicht aus, um die Ergebnisse jeder möglichen Kombination schon im Voraus zu erschlie{\ss}en. Diese Kombinationsmöglichkeiten sorgen aber trotz des simplen Systems für genug Abwechslung und Tiefe. Die einzige Ausnahme beim Kombinieren bilden die sammelbaren Objekte, dazu später mehr (siehe \fullref{sect:Collectables}).




\section{Spielziel}

Das Spiel ist aus verschiedenen Stufen (siehe \fullref{sect:stages}) aufgebaut, die vom Spieler abgeschlossen werden müssen.
Solange sich der Spieler im roten Ladebereich (Charger) in der Mitte des Spielfeldes aufhält, wird dieser aufgeladen.
Sobald der Charger aktiviert wird, fangen Gegner an zu erscheinen (spawnen). Diese können vom Spieler durch Schie{\ss}en besiegt werden, sie selbst schie{\ss}en aber ebenfalls auf den Spieler und können ihm so schaden.
Ist der Charger vollständig geladen, hören die Gegner auf zu spawnen und ein Portal zur nächsten Stufe wird aktiviert.




\section{Schussmodi}

Der Spieler kann seinen Schüssen jeweils eine der Eigenschaften der drei Grundfarben verleihen, welche er im Tutorial der Reihe nach freischalten und ausprobieren kann. Die ausgewählte Farbe kann jederzeig gewechselt werden:

\renewcommand{\itmspace}{4.5em}
\imgitmtxt{s_red.png}{Rote Kugeln}
{sind langsamer aber explodieren beim Aufprall.}

\imgitmtxt{s_green.png}{Grüne Kugeln}
{sind schneller und fügen mehr Schaden zu, prallen von Wänden ab und durchdringen Gegner, haben aber eine geringere Schussrate.}

\imgitmtxt{s_blue.png}{Blaue Kugeln}
{werden in einem Fächer von fünf Kugeln in einer höheren Schussrate verschossen, fügen aber weniger Schaden zu.}

Die gleichzeitige Kombination mehrerer Farben ist nur auf Kosten von Ressourcen erlaubt  (siehe \fullref{sect:ressourcen}). Dabei erhalten die Kugeln alle Eigenschaften der aktiven Farben.
Beispiel (in Klammern jeweils die Farbe, von der der Effekt stammt): Kombiniert man die Farben rot und grün, so erhält man gelbe Kugeln, die etwas langsamer als Grüne sind (durch rot wird der Bonus von grün verringert) und eine geringere Feuerrate haben (grün). Sie durchdringen Gegner und prallen von Wänden ab (grün), wobei jeder Kontakt eine Explosion erzeugt (rot), die zusätzlich mehr Schaden verursacht (grün).

Bei den anderen Kombinationen (magenta, cyan, wei{\ss}) ist analog zu verfahren.


\section{Gegner}

Im Spiel gibt es drei verschiedene Gegnertypen, erkennbar an ihrer Form:

\renewcommand{\itmspace}{4.5em}
\imgitmtxt{e_hunter.png}{Hunter}
{jagen den Spieler um ihm Kollisionsschaden zuzufügen.}

\imgitmtxt{e_archer.png}{Archer}
{verfolgen den Spieler aber bleiben auf Abstand und schie{\ss}en aus der Ferne auf ihn.}

\imgitmtxt{e_stray.png}{Strays}
{irren langsam über das Spielfeld und feuern 4 Schüsse in alle Richtungen.}

Auch die Gegner können in drei verschiedenen Farben vorkommen:

\renewcommand{\itmspace}{7.5em}
\imgitmtxt{e_red.png}{Rote Gegner}
{sind langsam, haben dafür aber mehr Leben.}

\imgitmtxt{e_green.png}{Grüne Gegner}
{bewegen sich schneller.}

\imgitmtxt{e_blue.png}{Blaue Gegner}
{spawnen in Zweiergruppen, haben aber weniger Leben.}

Bei der Kombination der Gegnerfarben ist ebenfalls wie bei den Schussfarben vorzugehen. Beispiel:

\imgitmtxt{e_white.png}{Wei{\ss}e Gegner}
{spawnen als Kombination aller drei Farben und haben mehrere Boni. Sie sind so schnell wie grüne Gegner, haben etwas weniger Leben als Rote und spawnen ebenfalls in Gruppen.}




\section{Collectables}

Wenn Gegner besiegt werden, lassen sie zu einer gewissen Wahrscheinlichkeit aufsammelbare Objekte (Collectables) abhängig von ihrer eigenen Farbe fallen.

\renewcommand{\itmspace}{5.5em}
\imgitmtxt{c_black.png}{Schwarz}
{lässt den Spieler einen Lebenspunkt zurückerhalten.}

\imgitmtxt{c_blue.png}{Rot, Grün, Blau}
{verleihen dem Spieler Ressourcen in der entsprechenden Farbe.}

\imgitmtxt{c_yellow.png}{Gelb}
{lässt den Spieler für kurze Zeit in alle Richtungen feuern.}

\imgitmtxt{c_cyan.png}{Cyan}
{erhöht kurzzeitig die Schussrate stark.}

\imgitmtxt{c_magenta.png}{Magenta}
{verleiht für einen kurzen Zeitraum Unsichtbarkeit/Unverwundbarkeit.}

\imgitmtxt{c_gold.png}{Gold}
{lässt einen goldenen Container fallen der bei Kontakt eine gro{\ss}e Explosion auslöst, die sämtliche Gegner auf dem Spielfeld auslöscht. (Selten und nur von wei{\ss}en Gegnern erhältlich.)}




\section{Ressourcen}

Für jede der drei Grundfarben existiert eine entsprechende Ressource, deren Stand in der Ressourcenanzeige oben rechts im Fenster angezeigt wird:

\includegraphics[height=100pt]{ressources.png}

Ressourcen erhält man durch das Aufsammeln von entsprechend gefärbten Collectables und das Besiegen von farbigen Gegnern. Die erhaltenen Ressourcen werden für mehrfarbige Gegner auf die entsprechenden Grundfarben aufgeteilt, so gibt zum Beispiel ein Cyanfarbiger Gegner blaue und grüne Ressourcen.

Benutzen kann man Ressourcen, sobald die Balken von mindesten zwei Farben vollständig gefüllt sind. Drückt man nun die Taste zum Aktivieren der Farbkombination (Leertaste), so werden den Schüssen die Eigenschaften der aufgefüllten Farben verliehen und die entsprechenden Ressourcen leeren sich über ungefähr 10 Sekunden. Die Farbkombination bleibt so lange aufrecht erhalten, bis eine der benutzten Ressourcen leer ist. Während dieser Zeit kann die Schussfarbe nicht mehr manuell gewechselt werden, dafür ist eine Kombination aus mehreren Farben deutlich stärker als jede einzelne Farbe.

Diese stärkere Phase wird also durch die Ressourcen zeitlich begrenzt, wodurch sie sich für den Spieler besonders anfühlt und dieser darauf hinarbeiten kann, sie erneut zu erleben. Zusätzlich lassen sich die Ressourcen auch während sie aktiv sind weiter aufladen (zum Beispiel durch das Töten von Gegnern), wodurch ein agressiver, spannender Spielstil gefördert wird.



\section{Interface und Hauptmenü}

Um die Immersion zu steigern, ist das Interface auf das Nötigste beschränkt. Allein ein Knopf zum Pausieren und die Ressourcenanzeige bilden gemeinsam das GUI. Ist das Spiel pausiert, können über insgesamt 5 Knöpfe die grundlegensten Funktionen wie 'Fortfahren', 'Spiel beenden' oder 'zurück ins Hauptmenü'  aufgerufen werden. 
Das Hauptmenü selbst ist dabei kein wirkliches Menü, sondern bereits die erste Stufe. Anstatt wie üblich per Knopfdruck einen bestimmten Spielmodus auszuwählen (hier Tutorial oder Endlos), muss der Spieler erst den Charger aufladen und dann zum entsprechenden Portal navigieren. Gegner erscheinen hier noch keine. Hinweise für den Spieler, wie zum Beispiel die grundlegende Steuerung oder Informationen über Schussfarben und Gegner, erscheinen ebenfalls nicht im GUI, sondern als Text auf dem Boden. 

So ist das Hauptmenü schon stark mit dem Gameplay verbunden, es gibt einen nahtlosen Übergang zum eigentlichen Spiel und der Fokus wird auf das Wesentliche gesetzt.

Auf eine herkömmliche Lebensanzeige des Spielers wurde auch verzichtet, stattdessen zeigen kleine um die Spielfigur kreisende Würfel die aktuellen Lebenspunkte an.



\section{Tutorial}

Das Tutorial besteht aus einer Folge von 10 Stufen (das Hauptmenü eingeschlossen), in denen der Spieler der Reihe nach mit den einzelnen Spielmechaniken konfrontiert wird. Ereignisgesteuerte Hinweistexte erklären neue Sachverhalte, sodass sich der Spieler mit ihnen vertraut machen kann. Das Tutorial ist wie folgt aufgebaut:

\begin{description}
\item[Hauptmenü] Steuerung, Spielmodusauswahl
\item[Stufe 1] Gegner, Schie{\ss}en
\item[Stufe 2] Erste Schussfarbe freischalten
\item[Stufe 3] Gegner in der freigschalteten Farbe erscheinen
\item[Stufe 4] Zweite Schussfarbe freischalten
\item[Stufe 5] Farbkombimnation erlaubt
\item[Stufe 6] Zweifarbige Gegner erscheinen
\item[Stufe 7] Dritte Schussfarbe freischalten
\item[Stufe 8] Möglichkeit zum Ausprobieren der neuen Farbkombinationen
\item[Stufe 9] Wei{\ss}e Gegner, letzte Herausforderung vor dem Endlosmodus
\item[Stufe 10+] Endlosmodus
\end{description}

Sobald Stufe 10 das erste mal erreicht wurde, wird die Möglichkeit freigeschalten, den Endlosmodus vom Hauptmenü aus direkt zu starten und das Tutorial zu überspringen.



\chapter{Dokumentation}




\section{Software-Architektur}

\lipsum[1]

\begin{figure}[h!]
\begin{center}
\includegraphics[height=170pt]{cross.png}
\caption{Klassendiagramm}
\label{fig:classdiag}
\end{center}
\end{figure}
    



\section{Stages}
\label{sect:stages}


\subsection{Aufbau}

Eine Stufe besteht im Wesentlichen aus 2 Gruppen von Komponenten: den statischen, unveränderlichen und den dynamisch generierten.

\renewcommand{\imgspace}{8em}
\imgtxt{stageInspect.png}{
Zu den statischen gehören Wände, die den Spielbereich abgrenzen und ggf. Hindernisse auf dem Spielfeld darstellen, der Spawn auf dem der Spieler die Stufe betritt, das Portal durch das der Spieler in die nächste Stufe gelangt und der Charger, den der Spieler aufladen muss, um das Portal zu aktivieren. (siehe \fullref{img:stage}) \\

Zu den dynamischen (im Bild links markiert) gehören die Gegner, abgefeuerte Bullets, Collectables, ggf. dynamisch generierte Hindernisse sowie dazugehörige Partikeleffekte. Zudem kann es in einer Stufe ein oder mehrere für den Spieler unsichtbare Gegnerspawner (siehe \nameref{subsect:spawner}) geben.
}

Alle dynamischen Objekte werden in einem dafür angelegten Containerobjekt in der Stage hinterlegt, damit er in seiner Gesamtheit beim Entladen einer Stufe gelöscht werden kann.



\subsection{Tutorialstages}

Das Tutorial besteht aus geskripteten Stages mit voreingestellten Spawnern, Farbcollectables und Hinweistexten. Die Hinweistexte und manchmal auch andere Komponenten werden hierbei durch zu den Stages gehörigen Skripte gesteuert. Dabei kann es unterschiedliche Verhalten geben, wenn der Endlosmodus bereits freigeschalten wurde, bspw. für die Anzeige von Statistiken und dem Highscore im Hauptmenü.



\subsection{Stagegenerator}

\lipsum[3]



\section{Spieler}

\lipsum[3]



\subsection{Input}

\lipsum[3]


\subsection{Movement}

\lipsum[3]


\subsection{Shooting}

\lipsum[3]


\subsection{Ressource}

\lipsum[3]




\section{Gegner}

\lipsum[3]



\subsection{Context-Steering}

Um das Gefecht interessanter zu machen wurde die Gegner-KI mit dem sog. Context-Steering Algorithmus\cite{qCtxSteer} (auch gen. Behavior-Driven Steering) ausgestattet.

Grundlegend wird hierbei jedem Objekt ein Wert zugeordnet der angibt, wie anstrebenswert es von bestimmten Entitäten ist. Ein positiver Wert ist hierbei ein Ziel, welches verfolgt wird, während ein negativer Wert ein zu vermeidendes Objekt darstellt.

In dieser Implementierung gibt es sog. Effektoren, die jeweils vier Werte speichern:

\begin{description}
\item[Zieltag] Zeichenkette, die die Gruppe an Objekten mit dieser Begehrlichkeit beschreibt
\item[Begehrlichkeit] wie anstrebenswert ein Objekt ist. Ein positiver Wert ist hierbei ein Ziel, welches verfolgt wird, während ein negativer Wert ein zu vermeidendes Objekt darstellt.
\item[Mindestabstand] Wie nah eine Entität sein muss, um von diesem Effektor beeinflusst zu werden
\item[Typ] Wie stark der Effektor abhängig vom Abstand wirkt. Das kann hier entweder logisch, linear oder quadratisch sein. 
\end{description}

Jede Entität bestimmt dann anhand der Formel $ clamp(typed(1 - dist / dist_{min}), 0, 1) * v_{dir} $ einen Vektor der die die Richtung und die Kraft zu diesem Objekt angibt.

Für die Erkennung von Wänden schaut jede Entität in in 30° Schritten um sich herum und schickt in jede dieser Richtungen einen Physik-Raycast. Von den Effektor-Vektoren wird dann  mittels Punktprodukt ein anteiliger Faktor in diese Richtung ermittelt und die resultierende Richtung durch die Summe dieser Richtungsvektoren ermittelt. Für etwas mehr Abwechslung wird diese noch durch Perlin-Noise ein wenig abgeändert.

\begin{figure}[H]
\centering
\includegraphics[width=0.7\linewidth]{ctxSteer.png}
\caption{Debug-Linien vom Kontext-Steering}
\label{img:ctxSteer}
\end{figure}

Da jede Gegnerart ein eigenes Verhalten hat, haben diese auch unterschiedliche Effektorenlisten. Strays besitzen nur das Standardverhalten und bewegen sich rein durch noise und vermeiden Wände und andere Gegner. Hunter haben zusätzlich eine Kraft zum Spieler hin und Archer wiederum noch einen Effektor mit kleinerem Radius vom Spieler weg, ebenfalls mit etwas Noise damit der Archer scheinbar vom SPieler strategisch vor- und zurückweicht.



\subsection{Gegnerspawner}
\label{subsect:spawner}

Die Gegnerspawner sind standardisierte Rechteck-Objekte, die unsichtbar für den Spieler einen Bereich markieren in dem Gegner mit einstellbaren Wahrscheinlichkeiten und Eigenschaften erscheinen:

\begin{description}
\item[Typverteilung] Wahrscheinlichkeitsverteilung der Gegnertypen.
\item[Farbverteilung] Wahrscheinlichkeitsverteilung der Farbmodifikatoren, abhängig von der Reihenfolge, in der die Schussfarben eingesammelt wurden.
\item[maxColors] maximale Anzahl an Farben die die Gegner haben können
\item[amount] maximale Anzahl an Gegner die dieser Spawner spawnt
\item[wavesize] Anzahl an Gegnern die der Spawner auf einmal spawnt (typischerweise 1-3)
\item[initDelay] Zeitspanne die der Spawner nach Aktivierung des Chargers wartet bis er anfängt Gegner zu spawnen
\item[delay] Zeitspanne zwischen Gegnerwellen
\item[variation] Zeitspanne, nach der einzelne Gegnerspawns verzögert werden
\end{description}

\begin{figure}[H]
\centering
\includegraphics[width=\linewidth]{spawner.png}
\caption{Spawner im Unity-Inspektor}
\label{img:spawner}
\end{figure}

Zudem hat jede Stage eine Maximale Anzahl an Gegnern, die sich Zeitgleich in der Stage aufhalten können, damit der Spieler nicht von Gegnern überrant wird.

In jeder Stage können beliebig viele Spawner platziert werden, jedoch ist durch die verfügbaren Einstellungsmöglichkeiten schon mit einem einzelnen Spawner meist ausreichend Variabilität möglich. Mehrere Spawner werden erst vonnöten, wenn man in einer Stage zeitlich verteilt verschiedene Wellen spawnen möchte.




\section{Physik}

\lipsum[3]



\section{Sound}

\lipsum[3]



\section{Spielerdaten}

Im Spiel werden verschiedene Arten von Daten persistent in Dateien abgespeichert. Als Datenformat wurde hier der Einfachheit halber JSON verwendet, da es einfach und menschenlesbar ist, sowie eine ausgezeichnete Portabilität zwischen Programmiersprachen hat, was ggf. für eine spätere Weiterverarbeitung und Auswertung nützlich sein kann. Gespeichert werden hierbei Fortschrittsstatistiken, Einstellungen und Nutzerstatistiken.



\textbf{Fortschritt} \\
Um im Spiel einen Wettbewerbsfaktor einzubringen, werden Statistiken aufgezeichnet, mit denen man Rückschlüsse auf die Fähigkeit eines Spielers ziehen kann. Diese werden ihm nach erfolgreicher Beendigung des Tutorials im Hauptmenü angezeigt. Dazu gehören zum einen allgemeine Informationen über die Spieldauer, dem Sessionzähler und die Anzahl der verschossenen Projektile, aber auch welche die maximal erreichte Stage ist, wie viele Stages insgesamt geschafft wurden, wie viele Gegner er besiegt hat und wie oft er selbst dabei daran glauben musste.

\begin{figure}[H]
\centering
\includegraphics[width=0.9\linewidth]{stats.png}
\caption{Statistiken im Hauptmenü}
\label{img:gameStats}
\end{figure}



\textbf{Einstellungen} \\
Weiterhin kann der Spieler im Spiel Einstellungen bezüglich Sound und Grafik vornehmen, um sein Spielerlebnis anzupassen, aber auch um die Plattformkompatibilität zu gewährleisten, da ältere Grafikkarten und vor allem integrierte Grafikeinheiten wie z. B. bei Laptops keine Compute Shader unterstützen oder allgemein zu wenig Leistung haben, um das Spiel in höchster Qualität flüssig spielen zu können. Diese sind auch als Tastaturkürzel bedienbar, sollte es zu einem Grafikfehler kommen und die GUI nicht bedienbar sein. (Z. B. öffnet die Kombination [X + M] das Einstellungsmenü)

\begin{figure}[H]
\centering
\includegraphics[width=0.8\linewidth]{config.png}
\caption{Einstellungen im Pausenmenü}
\label{img:gameConfig}
\end{figure}



\textbf{Nutzerstatistiken} \\
Au{\ss}erdem werden über die Dauer einer Spielsession Daten über den Spieler aufgezeichnet, mit denen der grobe Ablauf seines Spielerlebnisses nachvollzogen werden kann. Diese sollen später benutzt werden, um das Spiel zu verbessern (mehr dazu siehe \fullref{chapt:PlayerStats}).
Dabei wird für jede Stage die der Spieler betritt aufgezeichnet welchen Zustand der Spieler zu Beginn der Stage hatte (Leben, Schussfarben, Ressourcen), wie lang der Spieler zum Laden des Chargers gebraucht hat, wie oft Gegner eines bestimmten Typs und Farbe vorgekommen sind, welche Collectables gedroppt wurden und welche Schussfarben er benutzt hat.

\chapter{Grafikeffekte}




\section{Shader in Unity}

\lipsum[3]


\section{Partikelsystem}

Ursprünglich wurde das Partikelsystem basierend auf \href{https://docs.unity3d.com/ScriptReference/Graphics.DrawMeshInstancedIndirect.html}{DrawMeshInstancedIndirect} entworfen. Das hatte den Nachteil, 
Das Partikelsystem besteht aus einer Controller-Klasse in CSharp\#, einem Compute-Shader für die Berechnung und einem Unlit-Shader für die Darstellung.


\subsection{Partikel-Struktur}

Um den aktuellen Status eines einzelnen Partikels zu repräsentieren sollten möglichst kompakt alle für die Bewegung und Darstellung notwendigen Daten gespeichert werden. Dazu gehören die aktuelle Position, Geschwindigkeit und einwirkende Kraft, die Farbe und grö{\ss}e des Partikels und ein Random-Seed wert, der ggf. für zufälliges Verhalten verwendet wird.

\begin{csh}[caption=Partikelstruktur,label=lst:particle]
public struct Particle
{
    public Vector3 pos, vel, force;
    public Vector4 col, size; // xy: size,  z: age, w: maxage
    public float rand;
}
\end{csh}


\subsection{Partikel-Verhalten}

Darüber hinaus gibt es Variablen, die für jedes Partikel im Partikelsystem gleich gelten. Diese bestimmen, wie genau sich Partikel verhalten. Dazu gehören Ausgangswerte aller Positionsbezogenen Daten (Spawnposition und Geschwindigkeit), Grö{\ss}e, Farbe und Lebensdauer und wie sie sich über letztere hinweg verändern.

% \begin{csh}
% public class Particles : MonoBehaviour
% {
%     #region Public Variables
%     [Header("General")]
%     public Stats stats;
%     public GeneralProps properties;
%     public RenderSettings renderSettings;

%     [Header("Particles")]
%     public DynamicEffect pos, vel, force, posFac;
%     public Colors color = Colors.dflt;
%     [Tooltip("xy: offset, z: timefac")]
%     public Capsule<Vector3> size;
%     [Tooltip("Vector3 + w: factor")]
%     public Vector4 attractor;
%     #endregion
%     // ...
%     #endregion
% \end{csh}



\subsubsection{Positionsdaten}

Für Positionsdaten wird eine allgemeine Struktur mit standardisierter Uniform-Benennung verwendet, die einen Ausgangsvektor, einen Abweichungsvektor und eine Form in der diese Abweichung auftritt, definiert.

\begin{csh}[caption=DynamicEffect Struktur]
public enum Shape { DOT, CIRCLE, RECT, SPHERE, CUBE };

[System.Serializable]
public struct DynamicEffect
{
    public Vector3 offset, scale;
    public Shape shape;

    public void Uniform(ComputeShader cs, string name)
    {
        cs.SetVector(name + "Offset", offset);
        cs.SetVector(name + "Scale", scale);
        cs.SetInt(name + "Shape", (int)shape);
    }
}
\end{csh}



\subsubsection{Farbdaten}

Farbdaten werden entweder als Absoluter oder als Gradient angegeben. Dabei kann optional Variation durch zufällige Interpolation zwischen zwei Absoluter oder Gradientwerte erreicht werden. Die Gradiente werden als dynamisch generierte Textur an den Shader gereicht, die Absoluten Farbwerte als einfache Uniforms.

\begin{csh}
[System.Serializable]
public struct Colors
{
    public Color color, color2;
    public bool useVariation, useGradient;
    [Range(2, 256)] public int steps;
    public Gradient gradient, gradient2;

    // returns a k x steps Texture of the defined gradient
    // when using color variation, k = 2, otherwise 1
    private Texture2D getTexture();

    // uniform gradient texture to compute kernel (generated by getTexture())
    public void Uniform(ComputeShader compute, int kernel, string name);

    // uniform initial color(s) to compute shader
    public void UniformEmit(ComputeShader compute, string name)
}
\end{csh}



\subsection{Partikelsystem Eigenschaften}

Zu guter Letzt hat ein Partikelsystem noch einige globale Eigenschaften, die dessen Verhalten definieren. Dazu gehört die maximale Partikelzahl, die Emmitierrate, die Lebensdauer eines Partikels, sowie ob das PS aktiv ist, ob es sich wiederholt (d.h. ständig generiert wird oder nur bis insgesamt die maximale Partikelzahl erreicht wurde), ob es vorwärmen (d.h. einen Lebenszyklus beim Initialisieren emulieren soll) und einige weniger Relevante Eigenschaften auf die ich ggf. später zurückkomme.

\begin{csh}[caption=Partikel Struktur]
[System.Serializable]
public struct GeneralProps
{
    public bool enabled;
    public bool repeat;
    public bool prewarm;
    public bool destroyOnFinished;
    public PerformanceMode performance;

    public int maxParts;
    public float startDelay, emissionRate, lifetime;
}
\end{csh}



\subsection{Berechnung}


\subsubsection{Setup}

Um die Partikel möglichst effizient berechnen zu können werden Compute Shader verwendet. Alle Daten die an den ComputeShader gegeben werden sollen werden in sog. ComputeBuffer geschrieben. Diese werden direkt auf der Grafikkarte abgelegt. Das sind zum einen sämtliche Partikeldaten als Standard StructuredBuffer und ein AppendBuffer von IDs toter Partikel.

\begin{csh}[caption=Controller Setup,label=lst:partSetup]
#region Variables
// Calculation
[NotNull] public ComputeShader compute;
private ComputeBuffer particlesBuf; // particle properties
private ComputeBuffer deadBuf;      // dead particle indices
private ComputeBuffer counterBuf;   // deadBuf count

// Display
private ComputeBuffer quadVertBuf;  // particle vertices
public RenderSettings renderSettings = RenderSettings.Default;
[NotNull] public Texture tex;  // particle texture
[NotNull] public Material mat; // UnlitShader rendering material
#endregion

#region Setup

private void InitializePartBuffer()
{
    particlesBuf = new ComputeBuffer(stats.bufferSize, Marshal.SizeOf<Particle>());

    deadBuf = new ComputeBuffer(stats.bufferSize, sizeof(int), ComputeBufferType.Append);
    deadBuf.SetCounterValue(0);

    counterBuf = new ComputeBuffer(counterArray.Length, sizeof(int), ComputeBufferType.IndirectArguments);
    counterBuf.SetData(counterArray);

    quadVertBuf = new ComputeBuffer(meshVerts.Length, Marshal.SizeOf<Vector2>());
    quadVertBuf.SetData(meshVerts);
}

#endregion
\end{csh}

Der ComputeShader beinhaltet an dem Punkt nur die Kernel-Deklarationen, ein Include einiger Präprozessor-Makros und der obigen Partikel-Struktur (siehe \fullref{lst:particle}), sowie dem Partikel-Buffer. % Diese kommen manchmal doppelt vor, da Unity intern eine Restriktion von $2^16 = 65536$ Threadgruppen pro Kernel vorgibt. Dh. mit einem Aufruf mit mehr als $2^16$ Elementen würde er nicht alle Elemente abarbeiten können, und mit weniger würde die Grafikkarte unnötig ausgelastet.

\begin{hlsl}[caption=Compute Setup]
#pragma kernel Init
#pragma kernel Emit
#pragma kernel Update

#define THREADS 1024
#include "./Particle.cginc"

RWStructuredBuffer<Particle> _Particles;
\end{hlsl}

\subsubsection{Initialisierung}

Bei der Initialisierung werden sämtliche Partikel IDs dem deadBuffer hinzugefügt und deren Lebenszeit auf 0 gesetzt. Das ist der Ausgangszustand bei dem es keine lebendingen Partikel gibt.

\begin{csh}[caption=Controller Init Dispatch]
private void DispatchInit()
{
    compute.SetBuffer(kernelInit, "_Particles", particlesBuf);
    compute.SetBuffer(kernelInit, "_Dead", deadBuf);
    compute.Dispatch(kernelInit, stats.groupCount, 1, 1);
}
\end{csh}

\begin{hlsl}[caption=Compute Init Kernel]
AppendStructuredBuffer<uint> _Dead;

[numthreads(THREADS, 1, 1)]
void Init(uint3 id : SV_DispatchThreadID)
{
    _Particles[id.x].size.w = 0;
    _Dead.Append(id.x);
}
\end{hlsl}



\subsubsection{Emission}

Um Partikel zu emmittieren wird in zu jedem Frame-Update geprüft, wie viele Partikel abhängig von der Spawn- und der Framerate gespawnt werden sollen. Um auch sehr niedrige Raten zu unterstützen wird eine Partikelzeit mitgezählt, die jedes Frame um die Frametime erhöht, und bei jedem Emit um die Emissiontime gesenkt wird. So kann bspw. auch bei einer Bildrate von 60fps ein einzelnes Partikel pro Sekunde emmittiert werden, da die Zeit zwischen zwei Frames bei dem kein Partikel emmittiert wird nicht verloren geht.

\begin{csh}[caption=Controller Emission]
void Update()
{
    /* ... */
    if (properties.enabled && properties.emissionRate > 1e-2)
    {   // spawn particles
        if (partTimer == 0) partTimer = -10 * Time.deltaTime;
        partTimer += Time.deltaTime;

        DispatchEmit((int)(partTimer * properties.emissionRate));
        partTimer -= (int)(partTimer * properties.emissionRate) / properties.emissionRate;
    }
}
\end{csh}

Bevor die Partikel tatsächlich gespawnt werden wird noch die Anzahl gegen verschiedene Grenzen getestet - ua. die maximal lebendige Partikelzahl und bei nichtwiederholendem PS die bisher insgesamt emmittierte Partikelzahl.

Dann werden dem Compute-Shader sämtliche für den Spawn relevanten Informationen per Uniform überreicht. Ua. wie beim Init-Kernel die Partikeldaten und die IDs toter Partikel - diesmal aber als Comsume- statt AppendBuffer, da wiederbelebten Partikel nicht tot sind und somit aus dem deadBuffer entfernt werden müssen. Weiterhin werden einige Flags als Bitmaske in einem Integer übergeben, um ggf. den Datenverkehr gering zu halten, sowie eine Reihe an Zufallswerten für die Randomisierung einiger Werte.

\begin{csh}[caption=Controller Emission Dispatch]
#region Emit
private static int F(bool v, int p) => v ? 1 << p : 0;
int GetFlags() =>
    F(!stats.prewarmed, 0) +
    F(color.useGradient, 1) +
    F(color.useVariation, 2);

private void UniformEmit(int kernel)
{
    compute.SetBuffer(kernel, "_Particles", particlesBuf);
    compute.SetBuffer(kernel, "_Alive", deadBuf);

    compute.SetInt("_Flags", GetFlags());
    compute.SetVector("_Seeds", new Vector4(Random.value, Random.value, Random.value, Random.value));
    compute.SetFloat("_Lifetime", properties.lifetime);

    /* color size pos vel force posfac and parent Uniforms ... */
}

// try to emit <count> particles
public int DispatchEmit(int count)
{
    // skip if no repeat and all emmitted
    if (!properties.repeat && stats.emitted > curMaxParts) return 0;

    // calculate actually possible emission count
    count = Mathf.Min(count, 1 << 15, curMaxParts - stats.alive);
    if (count <= 0) return 0;
    if (!properties.repeat && stats.emitted + count > curMaxParts)
        count = curMaxParts - stats.emitted;

    // calculate alive count
    stats.alive += count;
    stats.dead = curMaxParts - count;

    // dispatch compute shader
    UniformEmit(kernelEmit);
    compute.Dispatch(kernelEmit, count, 1, 1);
    return count;
}
#endregion
\end{csh}

Im Shader werden nun Partikel aus dem DeadBuffer entnommen und diese mit Eigenschaften eines Lebendingen Partikels initialisiert. Um eine zufällige Streuung von Vektoren ua. für die Positionsdaten zu erreichen, wurde eine Funktion definiert, die aus drei Eingabewerten mittels einer stark schwankenden Funktion einen neuen, möglichst zufälligen Wert generiert. Damit kann durch Umordnung der vier einmalig im Hauptprogramm berechneten Seedwerte, kombiniert mit der Partikel-ID eine Vielzahl an Pseudozufallswerten generiert und das Verhalten der Partikel individualisiert werden.

Um die Partikel in einem bestimmten Bereich spawnen zu lassen werden Scales verwendet, zwischen dem die resultierenden Vektoren schwanken können. Um noch etwas mehr Variabilität zu sorgen kann man die Vektoren mittels einer Shape-ID 'formen'. Derzeit werden ausschlie{\ss}lich DOT (formlos) CIRCLE, RECT (2D), SPHERE und CUBE (3D) unterstützt. Die Dimensionsunterscheidung muss unternommen werden um die Vektoren nicht auf einer Achse zu verteilen, die garnicht verwendet wird und dann dort gehäuft auftreten.

In folgendem Listing wurde noch beispielhaft die Berechnung der Spawnposition gezeigt, die von Geschwindigkeit und Beschleunigung erfolgt analog, wenn auch aus Intuitivitätsgründen nicht 100\% identisch.

\begin{hlsl}[caption=Compute Emit Kernel]
int _Flags;
float _Lifetime;
float2 _Size;
float3 _PosParent, _SpdParent;
float4 _Color, _Color2, _Seeds;

float3 _PosOffset, _SpdOffset, _ForceOffset, _PosFacOffset;
float3 _PosScale, _SpdScale, _ForceScale, _PosFacScale;
int _PosShape, _SpdShape, _ForceShape, _PosFacShape;

// random functions
float c : SV_DispatchThreadID;
float rand(float3 o) {
    return frac(sin(dot(o*c, float3(12.9898, 78.233, 45.5432))) * 43758.5453);
}
float ran(float3 o) { return 2 * rand(o) - 1; }

// shaped vector distribution
float3 getShaped(float3 d, int shape) { /* ... */ };

[numthreads(THREADS, 1, 1)]
void Emit()
{
    uint id = _Alive.Consume(); // pop dead particle to alive
    Particle p = _Particles[id];
    c = id + 1; // randomize
    
    p.size.xy = float4(_Size, _Lifetime);
    p.rand = rand(_Seeds.wzy);
    p.color = F(P_CLRVARY) ? lerp(_Color, _Color2, p.rand) : _Color;
    
    // position
    float3 d = float3(ran(_Seeds.xzy), ran(_Seeds.yxz), ran(_Seeds.zyx));
    d = getShaped(d, _PosShape);
    p.pos = d * (_PosOffset + _PosScale * rand(_Seeds.wxy));
    
    /* velocity, force, parent ... */
    
    _Particles[id] = p;
}
\end{hlsl}


\subsubsection{Update}

Bei jedem Frame-Update wird der Status jedes Partikels aktualisiert. Dazu werden nurnoch Daten benötigt, die das Partikel während seiner Lebensspanne von au{\ss}en verändern. Dazu gehört in jedem Fall die Zeitdifferenz zum letzten Update, die Statusflags, der Farbgradient falls verwendet, sowie die Grö{\ss}enänderung, da diese bisher für alle Partikel gleich ist. Des weiteren kann man ein Objekt definieren, zu dessem Position das Partikel konstant hingezogen werden soll. Das wurde ua. für den Ressourcenpartikeleffekt benötigt.

\begin{csh}[caption=Controller Update]
void Update()
{
    /* ... */
    DispatchUpdate();

    if (properties.performance > PerformanceMode.LOW)
    {
        _alive -= Time.deltaTime * Mathf.Ceil(stats.alive) / properties.lifetime;
        stats.alive = Mathf.CeilToInt(_alive);
        stats.dead = curMaxParts - stats.alive;
    }

    if (Application.isPlaying && !properties.repeat && stats.emitted >= curMaxParts && stats.alive == 0)
    {
        if (_onFinished != null) _onFinished();
        if (properties.destroyOnFinished) Destroy(gameObject);
        return;
    }

    /* Emit ... */
}

#region Update
private void UniformUpdate(int kernel) { /* ... */ }

private void DispatchUpdate()
{
    UniformUpdate(kernelUpdate);
    compute.Dispatch(kernelUpdate, stats.groupCount, 1, 1);
}
#endregion
\end{csh}

Im Shader werden sämtliche Änderungen an einem Partikel abhängig von der Zeitdifferenz vorgenommen. Erreicht es hierbei ein Alter größer als die Lebensdauer, wird es wieder als tot markiert und dem DeadBuffer hinzugefügt. Bereits tote Partikel werden übersprungen, es wird aber dennoch für jedes Partikel die Update-Funktion aufgerufen da der Shader das nicht im vorhinein bestimmen kann. Die Farbe wird ggf. aus der Gradienttextur gesampled und interpoliert.

\begin{hlsl}[caption=Compute Update]
int _Flags;
float4 _Attractor;

float _SizeVel, _DeltaTime, _ColorSteps;
Texture2D<float4> _ColorGrad;

#define getAge(life) (1 - life.x / life.y)
float4 getLifeColor(float2 life, float seed)
{
    float4 color, color2;
    color = _ColorGrad.Load(int3(round(_ColorSteps * getAge(life)), 0, 0));
    if (F(P_CLRVARY)) {
        color2 = _ColorGrad.Load(int3(round(_ColorSteps * getAge(life)), 1, 0));
        color = lerp(color, color2, seed);
    }
    return color;
}

void DoUpdate(uint3 id)
{
    Particle p = _Particles[id.x];
    if (p.size.w == 0) return;
    
    if (p.size.z < _DeltaTime) {
        _Dead.Append(id.x);
        _Particles[id.x].size.w = 0;
        return;
    }
    
    float3 magnet = _Attractor.w * normalize(_Attractor.xyz - p.pos);
    
    p.size.z -= _DeltaTime;
    p.size.xy += _SizeVel * _DeltaTime;
    p.vel += p.force * _DeltaTime;
    p.pos += (p.vel + magnet) * _DeltaTime;
    if (F(P_CLRGRAD)) p.color = getLifeColor(p.size.zw, p.rand);

    _Particles[id.x] = p;
}
\end{hlsl}

Der Update-Kernel hat hierbei eine zusätzliche Funktion zum Vorwärmen das Partikelsystems. Beim Vorwärmen eines Partikelsystems werden beim Zurücksetzen zunächst so viele Partikel emmittiert, wie in einem Lebenszyklus durchschnittlich leben würden. Dann berechnet der Shader für jedes lebendige Partikel ein zufälliges Alter und führt so oft die Update-Funktion mit der eingestellten Deltatime auf, bis dieses erreicht ist.

\begin{csh}[caption=Controller Reset]
public void ResetPS()
{
    /* reset PS stats ... */
    if (properties.prewarm)
    {
        int count = (int)(properties.emissionRate * properties.lifetime);
        DispatchEmit(Mathf.Min(curMaxParts, count));
        DispatchUpdate();
    }
    stats.prewarmed = true;
}
\end{csh}

\begin{hlsl}[caption=Compute Update Kernel]
// update particle up to a random age
void DoInitUpdate(uint3 id)
{
    Particle p = _Particles[id.x];
    float age = p.size.w * p.rand - _DeltaTime;
    for (float i = 0; i < age; i += _DeltaTime) DoUpdate(id);
}

[numthreads(THREADS, 1, 1)]
void Update(uint3 id : SV_DispatchThreadID)
{
    if (F(P_PREWARM)) DoInitUpdate(id);
    else DoUpdate(id);
}
\end{hlsl}

Nach jedem Update wird anhand der durchschnittlichen Sterbenderate $r_{mort} = n_{alive} / t_{alive}$ berechnet, wie viele Partikel aktuell am leben sind. Das ist erheblich effizienter im Vergleich zu einer vorherigen Version, in der ein separater CounterBuffer dazu verwendet wurde, die genaue Grö{\ss}e des deadBuffers auszulesen. Da dieser physisch nur auf der Grafikkkarte existiert scheint Unity dazu eine Funktion namens Gfx.GetComputeBufferData\_Request zu benutzen, die zur Synchronisation eine Semaphore benutzt, was schon bei wenigen Partikelsystemen erhebliche Performanceeinbu{\ss}en mit sich brachte.

\begin{minipage}{0.95\linewidth}
\vspace{5mm}
\captionsetup{type=figure}
\includegraphics[width=\linewidth]{partBufCount.png}
\includegraphics[width=\linewidth]{partDataReq.png}
\captionof{figure}{ReadDeadCount im Unity Profiler}
%\label{img:partBufCount}
\end{minipage}

\begin{csh}[caption=Controller DeadCount]
private void ReadDeadCount()
{
    ComputeBuffer.CopyCount(deadBuf, counterBuf, 0);
    counterBuf.GetData(counterArray);
    stats.dead = counterArray[0];
    stats.alive = curMaxParts - stats.dead;
}
\end{csh}

\subsection{Darstellung}

Das Rendern der Partikel erfolgt über einen Unlit-Shader. Das Grundgerüst dafür sieht wie in \fullref{lst:partShader} aus. Diesem wird als Argument die Rendertextur übergeben, sowie BlendModes, die für korrektes Alpha-Blending nach Bedarf eingestellt werden können. Diese werden in der Struktur RenderSettings hinterlegt, zusammen mit einigen voreingestellten Modi\footnote{nach Artikel \href{https://elringus.me/blend-modes-in-unity}{BlendModes in Unity} vom 12.05.2015, Zugriff 23.06.2021}. Da Partikel transparent sind, wird die Transparent queue zum Render verwendet. Es ist wichtig ZWrite auszuschalten, da die Partikel sonst nicht korrekt sortiert werden (siehe \fullref{img:zoff}).

\begin{minipage}{\linewidth}
\begin{center}
\vspace{5mm}
\captionsetup{type=figure}
\includegraphics[width=0.2\linewidth]{zoff.png}
\captionof{figure}{ZWrite Off}
\label{img:zoff}
\end{center}
\end{minipage}

\begin{csh}[caption=Unlit Partikel-Shader,label=lst:partShader]
Shader "Custom/Particles"
{
    Properties
    {
        [MainTexture] _MainTex ("Main Texture", 2D) = "white" {}
        [HideInInspector] _BlendSrc ("Source BlendMode", int) = 0
        [HideInInspector] _BlendDst ("Dest BlendMode", int) = 0
        [HideInInspector] _BlendOp ("BlendOp", int) = 0
    }

    SubShader
    {
        Pass
        {
            Tags { "Queue" = "Transparent"  "IgnoreProjector" = "True" }
            BlendOp [_BlendOp]
            Blend [_BlendSrc] [_BlendDst]
            Lighting Off
            ZWrite Off
            
            CGPROGRAM
            #pragma target 4.5 % Compute Shaders
            #include "Particles.cginc"
            ENDCG
        }
    }
}
\end{csh}

Die Variablen für das Rendering sind im \fullref{lst:partSetup} bereits enthalten. Erwähnenswert ist der IndirectArguments-Buffer für Vertexkoordinaten eines Partikels - in diesem Falle die eines Quads, der an den Unlit-Shader gegeben wird. Gerendert werden die Partikel letztendlich mit der Funktion \href{https://docs.unity3d.com/ScriptReference/Graphics.DrawProceduralNow.html}{DrawProceduralNow}. Dieser bezieht seine Daten aus dem aktuell eingestellten ShaderPass (0), an dem das Shader-Material gebunden wurde. Die Partikeldaten werden somit direkt vom ComputeShader an den UnlitShader weitergereicht.

\begin{csh}[caption=Partikel Rendering]
void OnRenderObject()
{
    if (!enableParticles || isAnimPaused()) return;
    mat.mainTexture = tex;

    mat.SetBuffer("_Particles", particlesBuf);
    mat.SetBuffer("_QuadVert", quadVertBuf);
    renderSettings.Uniform(mat);
    mat.SetPass(0);

    Graphics.DrawProceduralNow(MeshTopology.Triangles, meshVerts.Length, deadBuf.count);
}
\end{csh}

Der Fragment Shader berechnet zunächst von lebendigen Partikeln das Billboard - dh. es dreht die Vertices richtung Kamera. Dafür werden die Partikel zunächst durch die Kamera-View Matrix transformiert, dann die Vertex-Positionen addiert und schließlich durch die Projektionsmatrix richtung Kamera gedreht. Die UV-Koordinaten sowie die Farbdaten werden dann an den Fragment-Shader weitergegeben, der die Farbdaten der Textur mit der Partikelfarbe multipliziert.

\begin{hlsl}
struct v2f
{
    float4 pos : SV_POSITION, color : COLOR;
    float2 uv : TEXCOORD0;
};

v2f vert(uint vid : SV_VertexID, uint iid : SV_INSTANCEID)
{
    v2f o = { 0,0,0,0, 0,0, 0,0,0,0 };
    Particle p = _Particles[iid];
    if (p.size.w == 0 || p.size.x < 0 || p.size.y < 0) return o;
    
    // Billboard
    float3 vpos = float3(p.size.xy * _QuadVert[vid], 0);
    float4 ppos = float4(p.pos, 1);
    o.pos = mul(UNITY_MATRIX_P, float4(vpos, 1) + mul(UNITY_MATRIX_V, ppos));
    
    o.uv = _QuadVert[vid] + 0.5;  // TRANSFORM_TEX((_QuadVert[vid] + 0.5), _MainTex);
    o.color = p.color;
    return o;
}

float4 frag(v2f i) : SV_Target { return i.color * tex2D(_MainTex, i.uv); }    
\end{hlsl}

% \\begin\{.*?\}[\s\S\n]*?\\end\{.*



\section{Postprocessing}

Postprocessing bezeichnet im Kontext der Videospielgrafik die Nachbearbeitung eines bereits ganz oder teilweise gerenderten Bildes mit einem oder mehreren Bildeffekten. Das Ziel des Postprocessings bei Cubix ist es, alte Videoaufnahme- und -wiedergabegeräte nachzubilden. Um diesen Effekt zu erzielen, werden Bildartefakte, die durch diese Hardware entstehen, als Nachbearbeitungseffekte dem fertigen Bild hinzugefügt.

Die verwendeten Effekte sind Bloom, Lens Flare, Chromatic Aberration, Vignette, Scanlines und Displaykrümmung.
Sie werden teils durch alte Kameralinsen und teils durch alte Monitore oder auch von beiden verursacht.
Jeder der genannten Effekte wird später genauer erläutert. Einige davon wurden aus artistischen Gründen ein wenig anders gestaltet, als sie in der Realität auftreten würden, dazu ebenfalls mehr in den entsprechenden Abschnitten.



\subsection{Postprocessing - Pipeline}

Das Postprocessing wird über ein ebenso benanntes Skript gesteuert. Bloom und Lens Flare werden von einem jeweils eigenen Compute Shader durchgeführt, alle anderen Effekte werden in eigenen Passes eines Image Effect Shaders durchgeführt. Diese sind im Skript als Objekte angelegt:

\begin{csh}
    public ComputeShader bloom;
    public ComputeShader lensFlare;
    public Material postProcMat;
\end{csh}

Weiterhin werden für den Lens Flare eine Textur einer schmutzigen Linse und eine Starburst Textur benötigt:

\begin{csh}
    public Texture2D lensDirtTex;
    public Texture2D starburstTex;
\end{csh}

Zur Sequenziellen Ausführung aller Effekte werden insgesamt 6 RenderTextures angelegt:

\begin{csh}
    public RenderTexture sourceTex;
    public RenderTexture brightTex;
    public RenderTexture blurBuff;
    public RenderTexture caResult;
    public RenderTexture lfResult;
    public RenderTexture lensTex;
\end{csh}

Zum Start des Programms oder falls die Fenstergrö{\ss}e geändert wird, müssen alle diese RenderTextures mit der Funktion createTexture() neu angelegt werden.

Durchgeführt wird das Postprocessing in der Funktion 
\begin{csh} 
private void OnRenderImage(RenderTexture source, RenderTexture destination) 
\end{csh}
Diese Funktion wird von der Unity Engine zur Verfügung gestellt und in einem Skript, das an eine Kamera angehängt ist, immer dann aufgerufen, wenn ein neues Bild gerendert wurde. Source ist dabei das Eingabebild und Destination ist das Bild, welches zum Schluss abgebildet wird. Source kann also noch beliebig verändert werden, bevor es nach Destination geschrieben wird. Der erste Effekt, welcher auf das Bild angewendet wird, ist Bloom. Bloom ist als Compute Shader verfasst, für den Texturen, in die geschrieben werden soll, das Flag 'enableRandomWrite' benötigen. Dieses lässt sich für einmal erstellte Texturen im Nachhinein nicht mehr ändern und ist in Source standardmäßig deaktiviert. Also ist der erste Schritt, mit der Funktion 
\begin{csh}
Graphics.Blit(source, sourceTex);
\end{csh}
 den Inhalt von Source in die zuvor angelegte Textur sourceTex zu schreiben, für welche das benötigte Flag gesetzt ist.

%image source to source tex

Nun kann der Bloom-Effekt angewandt werden. Zuerst werden mithilfe des Bloomshaders alle hellen Stellen aus sourceTex nach brightTex geschrieben. Der Inhalt von brightTex wird nun verschwommen gemacht, wozu die Textur blurBuff benötigt wird. Anschlie{\ss}end wird das verschwommene Resultat (brightTex) additiv zurück nach sourceTex geschrieben.

%image bloom

Als Nächstes wird Chromatic Aberration auf das fertige Bild mit Bloom (sourceTex) angewandt. Es wird keine zusätzliche Textur benötigt und das Resultat befindet sich in caResult.

%image ca

Nun ist Lens Flare an der Reihe. Als Ausgangstextur dient ebenfalls sourceTex und ähnlich wie bei Bloom werden bestimmte Features des Ursprungsbildes in eine separate Textur (lfResult) geschrieben. Diese wird dann wiederum mithilfe von blurBuff verschwommen gemacht und dieses Mal additiv auf caResult zurückgeschrieben. 

%image lf

Vorher jedoch wird das Ergebnis mit der Textur lensTex kombiniert, die zum Start des Programms einmal aus den beiden Texturen lensDirtTex und starburstTex erstellt wird.

%image lensTex

Die Effekte Vignette, Scanlines und Displaykrümmung können zuletzt alle mit einmal auf caResult angewendet werden, wobei das Ergebnis direkt nach Destination (Argument von OnRenderImage) geschrieben wird.

%image crt

Somit hat das resultierende Bild alle gewünschten Effekte.

%complete pipeline



\subsection{Blur}
\label{label:blur}

Für die Effekte Bloom und Lens Flare wird eine Methode benötigt, Texturen verschwommen zu machen, was die Aufgabe der Blur-Shaders ist.
Er ist ein Fragment Shader und wendet einen einfachen Gau{\ss}-schen Weichzeichnungsalgorithmus auf eine Textur an. Benötigt wird ein Ursprungsbild und ein Puffer der selben Grö{\ss}e.

Übergebene Variablen:
\begin{description}
\item[sampler2D MainTex] Die Eingabetextur
\item[float4 MainTexTexelSize] Die Pixelgröße der Textur in x- und y-Richtung
\item[int horizontal] Angabe, ob horizontal oder vertikal verwaschen werden soll
\end{description}

Der Algorithmus funktioniert, indem er zu einem Pixel immer auch den Wert jedes Pixels in der näheren Umgebung abfragt. Alle Werte werden dann mit unterschiedlichen Gewichtungen zusammenaddiert (je weiter vom Ursprungspixel entfernt, desto weniger Einfluss) und ergeben ein verschwommenes Abbild des Originals. Mit einer Distanz von vier Pixeln in jede Richtung würden die abgefragten Pixel wie folgt gewichtet werden:

\includegraphics[height=100pt]{gauss_normal.png}

Diese Methode hat jedoch den Nachteil, dass 81 (9x9) Texturzugriffe für jedes Pixel notwendig sind, was nicht sehr effizient ist.
Die Texturzugriffe lassen sich jedoch verringern, indem in zwei Durchläufen des Shaders die Pixeldaten zuerst nur in horizontale Richtung und dann nur in vertikale Richtung abgefragt werden:

\includegraphics[height=100pt]{gauss_horizontal.png}
\includegraphics[height=100pt]{gauss_vertical.png}

Das Resultat ist dabei fast identisch: 

\includegraphics[height=100pt]{gauss_twopass.png}

Diese Methode nennt sich two-pass Gaussian Blur und es sind hier nur 18 (2x9) Texturzugriffe notwendig. Die angelegte Variable 'horizontal' gibt bei jedem Aufruf des Shaders an, ob in die horizontale oder die vertikale Richtung verwaschen werden soll, und wird nach jedem Aufruf invertiert. Zusätzlich ist nun ein separater Puffer vonnöten (hier blurBuff), in dem das resultat des ersten Durchlaufes (horizontal) zwischengespeichert wird und dessen Inhalt im zweiten Durchlauf (vertikal) verschwommen wieder zurück in die ursprüngliche Textur geschrieben wird. Dieser Prozess ist im Postprocessing-Skript in eine eigene Blur-Methode ausgelagert worden:

\begin{csh}
private void blur(RenderTexture tex, int count)
{
    for (int i = 0; i < count; i++)
    {
        postProcMat.SetInt("_horizontal", 1);
        Graphics.Blit(tex, blurBuff, postProcMat, BlurPass);
        postProcMat.SetInt("_horizontal", 0);
        Graphics.Blit(blurBuff, tex, postProcMat, BlurPass);
    }
}
\end{csh}

Zusätzlich zur zu bearbeitenden Textur wird der Methode eine Variable 'count' übergeben, über welche die Stärke der Verwaschung reguliert werden kann. Dazu wird die Shadersequenz so oft durchlaufen, wie count angibt. Nach jedem Shaderaufruf wird die Uniform Variable 'horizontal' auf den entsprechenden Wert neu gesetzt.

Jedoch lässt sich die Anzahl der Texturzugriffe noch weiter senken. In einem Fragment Shader kann nicht nur auf diskrete Texturkoordinaten zugegriffen werden. Fragt man den Farbwert einer Textur an einer Stelle zwischen zwei Pixeln ab, so werden deren Farben linear interpoliert. In diesem Fall ermöglicht dies, das gleiche Resultat mit nur 10 (2x5) Texturzugriffen zu erzielen. Dazu werden zwei Wertelisten angelegt:
\begin{hlsl}
    static const float weight[3] = { 0.2270270270, 0.3162162162, 0.0702702703 };
    static const float offset[3] = { 0.0, 1.3846153846, 3.2307692308 };
\end{hlsl}
'Offset' gibt an, wie viele Pixel entfernt vom Hauptpixel die Farbdaten bestimmt werden sollen. Zu beachten ist hier, dass keine ganzzahligen Pixelabstände gewählt wurden, um den gewünschten Effekt zu erzielen.
'Weight' gibt zu jedem Offset an, wie stark dieser Wert in das Endergebnis einflie{\ss}t. Mit der richtigen Werteliste ist das Ergebnis identisch zu einem Algorithmus, der mit den 5 ganzzahligen Pixeloffsets 0, 1, 2, 3 und 4 arbeitet, benötigt jedoch nur reichlich die Hälfte der Texturzugriffe.

%https://john-chapman.github.io/2017/11/05/pseudo-lens-flare.html
%http://john-chapman-graphics.blogspot.com/2013/02/pseudo-lens-flare.html
%rastergrid.com/blog/2010/09/efficient-gaussian-blur-with-linear-sampling/
%https://learnopengl.com/Advanced-Lighting/Bloom



\subsection{Bloom}
\label{label:bloom}

Bloom (Überstrahlung) bezeichnet das Phänomen, dass sehr helle Bereiche eines Bildes nahe Bereiche 'überstrahlen'. Dies resultierte bei alten Kameras daraus, dass sie nicht den gesamten Helligkeitsbereich richtig aufnehmen konnten. In Videospielen und der Filmindustrie wird dieser Effekt gern verwendet, um den Eindruck von gro{\ss}er Helligkeit zu vermitteln, jedoch sollte er auch nicht zu stark sein und ablenken.
Der Algorithmus ist als Compute Shader verfasst und besteht aus zwei Schritten, die von einem jeweils eigenen Kernel ausgeführt werden. 

Übergebene Variablen:
\begin{description}
\item[RWTexture2D<float4> Source] Bildquelle
\item[RWTexture2D<float4> BrightSpots] Textur, in die Helle Stellen geschrieben werden
\item[uniform float threshold] Schwellenwert für die Helligkeit eines Pixels
\end{description}

Den ersten Schritt übernimmt das Kernel 'CSExtractBright'. Hier werden alle Pixel von Source mit der Formel $dot(float3(0.2126, 0.7152, 0.0722), Source[id.xy].rgb) > threshold$ auf ihre Helligkeit geprüft. Da unterschiedliche Lichtfarben für die menschliche Wahrnehmung unterschiedlich zur Helligkeit eines Pixels beitragen, werden die Rot-, Grün- und Blaukanäle des ursprünglichen Bildes unterschiedlich gewichtet (mit den Werten 0.2126, 0.7152 und 0.0722). Ist der Schwellenwert erreicht, werden die Pixeldaten unverändert in BrightSpots geschrieben (BrightSpots[id.xy] = Source[id.xy]), anderenfalls wird in BrigthSpots an dieser Stelle schwarz eingefügt: BrightSpots[id.xy] = float4(0, 0, 0, 1).
Somit erhält man in BrightSpots ein Abbild aller hellen Stellen eines Bildes:

\captionsetup{type=figure}
\includegraphics[height=100pt]{bloom_source.png}
\includegraphics[height=100pt]{bloom_bright.png}
\captionof{figure}{Links: Source, Rechts: BrightSpots}

Anschlie{\ss}end wird der Blur-Effekt auf BrightSpots angewendet, wodurch die hellen Stellen an den Rändern mit den Dunklen verwaschen werden. Im zweiten Schritt wird dann das verschwommene Resultat in BrightSpots vom zweiten Kernel 'CSWriteBack' additiv zurück auf das Ursprungsbild geschrieben. Dadurch wird hellen Objekten ein leichter 'Schein' auf die nahe Umgebung verliehen:

\captionsetup{type=figure}
\includegraphics[height=75pt]{bloom_off.png}
\includegraphics[height=75pt]{bloom_on.png}
\captionof{figure}{Links: Bloom aus, Rechts: Bloom an}




\subsection{Chromatic Aberration}
\label{label:chromatic aberration}

Chromatic Aberration ist ein von Kameralinsen und Röhrenmonitoren verursachtes Bildartefakt, das im Falle der Kameralinse dadurch auftritt, dass das eintreffende Licht von dieser nicht korrekt gebündelt wird. Licht unterschiedlicher Wellenlängen wird unterschiedlich stark von der Linse gebrochen, wodurch besonders kurz- oder langwellige Lichtfarben (zum Beispiel rot und blau) mit einiger Abweichung auf dem Sensor eintreffen. Normalerweise tritt Chromatic Aberration in Richtung der Ränder des Bildes stärker auf, jedoch erschwerte eine zu gro{\ss}e Effektstärke die Leserlichkeit des GUI. In der Mitte des Bildschirmes sollte der Effekt wiederum nicht zu schwach sein, da er bei der wei{\ss}en Spielfigur am besten zur Geltung kam. Deshalb fiel die Entscheidung, die Stärke der Chromatic Aberration für den gesamten Bildschirm auf einen einheitlichen Wert zu setzen.

Als Shader kann der Effekt dadurch nachgebildet werden, dass vom ursprünglichen Bild der rote und blaue Farbkanal mit einer kleinen Abweichung zur eigentlichen Pixelposition abgefragt werden und dann gemeinsam mit dem Grünkanal an der richtigen Position das neue Pixel bilden. Die Abweichung der Position erfolgt für rot und blau in jeweils entgegengesetzte Richtung in einem Fragment Shader:

Übergebene Variablen:
\begin{description}
\item[sampler2D MainTex] Die Eingabetextur
\item[float CAAmount] Abweichung für Zugriffe auf den Rot- und Blaukanal
\end{description}

\begin{hlsl}
    fixed4 col = fixed4(0, 0, 0, 1);
    col.r = tex2D(_MainTex, i.uv + float2(_CAAmount, 0)).r;
    col.g = tex2D(_MainTex, i.uv).g;
    col.b = tex2D(_MainTex, i.uv - float2(_CAAmount, 0)).b;
\end{hlsl}

Wie zu sehen ist, wird der Rotkanal der resultierenden Farbe in MainTex an der entsprechenden Texturpixelposition (i.uv) plus einer Abweichung 'CAAmount' ermittelt. Für den Blaukanal wird i.uv minus CAAmount abgefragt, beim Grünkanal gibt es keine Abweichung. Damit CAAmount als Abweichung nur die x-Position des Pixels beeinflusst, wird ein float2 erstellt, der CAAmount als x-Wert und 0 als y-Wert besitzt, welcher dann zu i.uv (auch zweidimensional) dazu addiert beziehungsweise davon abgezogen werden kann.
CAAmount kann vom Hauptprogramm aus gesteuert werden, als Standard wurde der Wert 0.0005 gewählt. Das bedeutet, dass rot und blau um ein halbes Tausendstel der Displaybreite (ein paar Pixel) von ihrer ursprünglichen Position abweichen. So sehen verschiedene Werte für CAAmount im Vergleich aus:

\captionsetup{type=figure}
\includegraphics[height=75pt]{bloom_on.png}
\includegraphics[height=75pt]{ca_0005.png}
\includegraphics[height=75pt]{ca_005.png}
\captionof{figure}{Links: kein Effekt, Mitte: CAAmount = 0.0005, Rechts: CAAmount = 0.005}

Chromatic Aberration eignet sich sich bei gro{\ss}en Abweichungswerten auch gut, um das Gefühl von Fehlern beziehungsweise Glitches zu vermitteln. So wird zum Beispiel für kurze Zeit nachdem der Spieler getroffen wurde, die Abweichung auf zufällig wechselnde, hohe Werte gesetzt, was den erlittenen Schaden gut verdeutlicht.



\subsection{Lens Flare}

Als Lens Flare werden Abbilder von hellen Bereichen eines Bildes bezeichnet, die entlang einer Linie durch dessen Zentrum auftreten. Der Effekt entsteht dadurch, dass helles Licht zwischen einer oder mehreren Kameralinsen reflektiert wird (siehe \fullref{img:lensFlareImg}\footnote{\href{https://de.wikipedia.org/wiki/Lens_Flare}{Lens Flare} Zugriff 23.06.2021}). Unter Umständen kann zusätzlich auch ein sogenannter Halo entstehen, ein blasser Ring um die Lichtquelle.

\includegraphics[height=100pt]{lensflare.png}
\caption{Lens Flare Beispiel}
\label{img:lensFlareImg}

Lens Flare wird aus artistischen Gründen oft in Filmen sowie in der Fotografie und in Videospielen emuliert. In letzterem Fall gibt es dafür grundlegend zwei Ansätze: Zum Einen der Spritebasierte Ansatz, bei dem die Abbilder (auch Ghosts genannt) der hellen Stellen als Bilder (Sprites) vorgefertigt sind, die dann entlang der Linie durch das Zentrum platziert werden. Dafür ist es notwendig, Lichtquellen, welche Lens Flare verursachen sollen, entsprechend zu markieren und dann beim Rendern des Bildes zu prüfen, ob diese Lichtquellen im Blickfeld zu sehen sind. Danach werden anhand deren Position die Ghost-Sprites auf dem Bild platziert. Der Nachteil dieser Methode ist, dass viele Informationen über die Lichtquellen notwendig sind. Au{\ss}erdem werden Bereiche, die zwar hell genug sind, aber nicht für Lens Flare markiert wurden, auch keinen Lens Flare erzeugen. Zusätzlich ist die Form der Ghosts von Beginn an festgelegt, was unter Umständen auch nicht gewünscht ist.

Eine Alternative bildet der Ansatz des Screen Space Lens Flare. Dabei werden dynamisch alle Bereiche eines Bildes extrahiert, deren Helligkeit einen gewissen Schwellenwert übersteigt und danach automatisch entsprechende Ghosts erzeugt. Dieser Ansatz ist wesentlich Flexibler, da keine anderen Informationen als das Bild selbst notwendig sind, er ist jedoch auch rechenintensiver und es lassen sich bestimmten Lichtquellen nicht einfach ausschlie{\ss}en.

Cubix verwendet den zweiten Ansatz. Lens Flare ist hier als Compute Shader mit zwei Kernels verfasst, welche ähnliche Aufgaben wie die des \nameref{label:bloom} Effekts haben. 

Übergebene Variablen:
\begin{description}
\item[RWTexture2D<float4> Source] Die Eingabetextur
\item[RWTexture2D<float4> Result] Textur für die entstandenen Ghosts und den Halo
\item[Texture2D<float4> lensTex] Transparente Textur einer dreckigen Linse
\item[uniform int ghostCount] Anzahl der Ghosts pro hellem Bereich
\item[uniform float ghostSpacing] Abstand der Ghosts zueinander
\item[uniform float threshold] Schwellenwert der Helligkeit, ab dem Ghosts/Halo produziert werden
\item[uniform float caStrength] Stärke der Chromatic Aberration der Ghosts/des Halo
\end{description}

Der Algorithmus läuft in zwei Schritten ab: Zuerst werden für helle Stellen im Bild die Ghosts und der Halo generiert, anschlie{\ss}end werden mittels \nameref{label:blur} die entstandenen Features verschwommen gemacht und unter Einbezug der Textur lensTex auf das Ursprungsbild zurückgeschrieben.

Für den ersten Schritt wurde im Shader der Kernel 'CSFlare' angelegt. Texturzugriffe erfolgen hier mithilfe der übergebenen Variable 'id', deren Werte sind diskrete Pixelkoordinaten. Damit später mit ihnen einfacher zu rechnen ist, müssen sie auf Werte zwischen 0 und 1 transformiert werden. Dies erfolgt über das Teilen durch die grö{\ss}e der Textur in Pixeln, welche mit der Funktion GetDimensions ermittelt werden kann:

\begin{hlsl}
uint width, height;
Source.GetDimensions(width, height);
float2 iResolution = float2(width, height); // Size of Source
float2 texcoord = (id.xy / iResolution); //transform coordinates to [0, 1]
\end{hlsl}

Texcoord ist nun also ein Vektor mit einem Wertebereich von 0 bis 1, der auf das derzeitige Pixel zeigt. Um nun wieder auf die Texturen zugreifen zu können, muss dieser Vektor mit deren Grö{\ss}e multipliziert werden:

\begin{hlsl}
Source[texcoords * iResolution];
\end{hlsl}

Sehr wichtig ist auch der Vektor 'ghostVec'. Dieser zeigt vom derzeitigen Pixel in Richtung Mitte des Bildes und wird wie folgt berechnet:

$ghostVec = ( -texcoord + .5 ) * ghostSpacing$

GhostSpacing ist eine der Uniform Variablen und gibt an, um wie viel länger beziehungsweise kürzer ghostVec sein soll als der Abstand von texcoord zur Bildmitte (siehe \fullref{img:lfghostvec})

\captionsetup{type=figure}
\includegraphics[height=150pt]{lf_ghostvec.png}
\captionof{figure}{ghostVec}
\label{img:lfghostvec}

Nun werden die Ghosts und der Halo jeweils in ihrer eigenen Funktion generiert und zu einem Ergebnis zusammenaddiert:

\begin{hlsl}
float3 col = 0;
col += generateGhosts(texcoord, ghostVec, iResolution);
col += generateHalo(texcoord, ghostVec, iResolution);
\end{hlsl}

Zuerst werden mit 'generateGhosts', wie der Name schon sagt, die Ghosts generiert. Eine Schleife, die an unterschiedlichen Stellen des Bildes den entsprechenden Helligkeitswert überprüft, läuft so oft ab, wie durch den übergebenen Parameter ghostCount angegeben. Dabei wird in jedem Durchlauf einmal der Vektor ghostVec als Abweichung zu den derzeitigen Texturkoordinaten hinzuaddiert. An der resultierenden Stelle wird dann die Farbe des Ursprungsbildes bestimmt und der Schwellenwert wird von ihr abgezogen: $texColor = max(texColor - threshold, 0)$

Zuletzt wird der Wert von texColor mit einem Faktor multipliziert, der davon abhängig ist, wie weit die helle Stelle vom Zentrum des Bildes entfernt ist, da Lichtquellen am Rand weniger bis gar kein Lens Flare auslösen sollen. \fullref{img:lfghostgen} zeigt, wie die Berechnung der Koordinaten der hellen Stellen für jeden Ghost stattfindet.

\captionsetup{type=figure}
\includegraphics[height=150pt]{lf_ghostgen_1.png}
\includegraphics[height=150pt]{lf_ghostgen_2.png}
\includegraphics[height=150pt]{lf_ghostgen_3.png}
\captionof{figure}{Berechnung der Position der hellen Stelle für Ghosts (rot umrandet sei ein heller Bereich)}
\label{img:lfghostgen}

Wie zu sehen ist, wird ghostVec kleiner, je näher der Ghost dem Zentrum des Bildes ist. Mit jeder Iteration entfernt sich Offset einmal um die Länge von ghostVec von den ursprünglichen Koordinaten. Deshalb muss die Schleife für jeden Ghost einmal öfter durchlaufen werden, um an jeweils die gleichen Offset-Koordinaten zu gelangen. Für ghostCount war in diesem Beispiel 3 eingestellt.

Somit ist die Erstellung der Ghosts abgeschlossen und es folgt der Halo. Die Funktion 'generateHalo' arbeitet grundsätzlich sehr ähnlich zu 'generateGhosts'. Die Unterschiede bestehen darin, dass es keine Schleife gibt (zu jedem Objekt existiert immer nur ein Halo) und dass ghostVec auf eine feste Länge normalisiert und nun haloVec genannt wird: $haloVec = normalize(ghostVec) * 0.35$

Der Faktor 0.35 legt den Radius des Halo fest (auf 0.35 Bildbreiten/-höhen). HaloVec zeigt genau wie ghostVec in Richtung Mitte des Bildes, hat aber eine feste Länge. Dadurch bilden sich um jede Stelle des Bildes teile eines Ringes, in denen haloVec auf besagte Stelle zeigt. Besser veranschaulicht ist der Sachverhalt in \fullref{img:lfhalogen}

\captionsetup{type=figure}
\includegraphics[height=150pt]{lf_halogen_1.png}
\includegraphics[height=150pt]{lf_halogen_2.png}
\captionof{figure}{Berechnung der Stellen, die Teil des Halo sind (rot umrandet sei ein heller Bereich)}
\label{img:lfhalogen}

Je näher die helle Stelle dem Zentrum ist, desto grö{\ss}er wird der Halo, bis er sich gänzlich schlie{\ss}t, falls sie sich genau dort befindet.

Nun sind alle Features erstellt, die für den Lens Flare notwendig sind. Sie befinden sich in der Textur Result und müssten noch per \nameref{label:blur} verschwommen gemacht werden. Jedoch lässt sich die Qualität der Features noch verbessern: Da Licht unterschiedlicher Wellenlängen an der Linse unterschiedlich stark gebrochen wird, haben auch deren Ghosts unterschiedliche Abstände. Es wird hier also ähnlich vorgegangen wie bei \nameref{label:chromatic aberration}, die Rot- und Blaukanäle werden mit einer gewissen Abweichung zu den eigentlichen Koordinaten abgefragt. Der Unterschied besteht darin, dass die Abweichung richtung Zentrum erfolgt und deren Stärke anhand der Entfernung davon skaliert wird. Dies führt dann zu folgendem Unterschied in Result:

\captionsetup{type=figure}
\includegraphics[height=100pt]{lf_ca_off.png}
\includegraphics[height=100pt]{lf_ca_on.png}
\captionof{figure}{Links: ohne Chromatic Aberration, Rechts: mit Chromatic Aberration}

Erzielt wird dieser Effekt mithilfe der Funktion 'float3 getTexColor(float2 texcoords, float2 iResolution)'. Diese wird immer dort aufgerufen, wo ein Texturzugriff auf Source erfolgen würde. Für den Zugriff auf den Rotkanal wird ein Offset zu den Koordinaten addiert, beim Blaukanal wird Offset abgezogen:

\begin{hlsl}
return float3(
	Source[(texcoords + offset) * iResolution].r,
	Source[texcoords * iResolution].g,
	Source[(texcoords - offset) * iResolution].b
	);
\end{hlsl}

Die Richtung von Offset entspricht der Richtung von texcoord zum Zentrum, die Länge ergibt sich aus der Entfernung von texcoord zum Zentrum und der übergebenen Variable caStrength, die vom Skript aus steuerbar ist:

\begin{hlsl}
    float amount = length(.5 - texcoords) * caStrength;
    float2 offset = normalize(.5 - texcoords) * amount;
\end{hlsl}

Nun ist der erste Schritt abgeschlossen und Result kann (nachdem es verschwommen gemacht wurde) auf die Ausgangstextur zurückgeschrieben werden. Jedoch soll das Ergebnis nicht von der normalen \nameref{label:chromatic aberration} beeinflusst werden, weshalb Source für das Zurückschreiben zuvor auf das Resultat des Chromatic Aberration Effekts geändert wird. Dieser Schritt wird ähnlich wie bei Bloom von einem eigenen Kernel 'CSWriteBack' übernommen:

\captionsetup{type=figure}
\includegraphics[height=100pt]{lf_ca_on.png}
\includegraphics[height=100pt]{lf_writeback_2.png}
\captionof{figure}{Links: Lens Flare Features, Rechts: Resultat}

So sieht der Lens Flare schon fast wie im fertigen Spiel aus, es fehlen nur noch die Streifen und Unreinheiten auf der Linse. Dafür wird die Textur lensDirt verwendet. Diese wird zu Beginn des Spiels aus einer fertigen Textur mit Linsendreck und einem kreisförmig gerenderten Barcode durch einen Fragment Shader generiert:

\captionsetup{type=figure}
\includegraphics[width=1\linewidth]{lf_lensgen.png}
\captionof{figure}{Links: Linsendreck, Mitte: Barcode, Rechts: Resultat}

Dieses Bild wird im Writeback Kernel des Lens Flare multiplikativ mit den Ghosts und dem Halo kombiniert, bevor diese auf die Ausgangstextur zurückgeschrieben werden (siehe \fullref{img:lfcubix}). Der Lens Flare Effekt ist somit vollständig.

\captionsetup{type=figure}
\includegraphics[height=150pt]{lf_cubix.png}
\captionof{figure}{Lens Flare in Cubix}
\label{img:lfcubix}

%https://docs.unrealengine.com/4.26/Images/RenderingAndGraphics/PostProcessEffects/Bloom/DirtMaskTextureExample.png


\subsection{Vignette, warped Display und Scanlines}

Die drei Effekte Vignette, Displaykrümmung und Scanlines werden unter anderem von alten CRT-Monitoren verursacht. Vignette bezeichnet eine Abdunklung verschiedener Bildbereiche (meist der Ränder). Scanlines sind (meist horizontale) dunkle Linien, die dadurch entstehen, dass das Bild vom Monitor zeilenweise gezeichnet wird. Auch hatten diese Monitore gewölbte Displays, deren Effekt ebenfalls emuliert wird.
%link CRT

Sie sind in einer Sektion zusammengefasst, da sie in einem einzigen Fragment Shader Pass gemeinsam durchgeführt werden können.

Übergebene Variablen:
\begin{description}
\item[sampler2D MainTex] Die Eingabetextur
\item[float vignetteAmount] gibt an, wie stark die Ränder abgedunkelt werden
\item[float vignetteWidth] gibt an, wie weit die Verdunklung in Richtung Bildmitte vordringt
\end{description}

Zuerst wird der Warped Display Effekt durchgeführt. Dafür werden die UV-Koordinaten, welche Werte von 0 bis 1 annehmen können, zuerst in einen Wertebereich zwischen -1 und 1 transformiert: $uv = i.uv * 2 - 1$.
Anschlie{\ss}end wird mit folgenden Formeln der x-Wert der Koordinaten gestreckt: $uv.x = uv.x * (1 + pow(abs(uv.y) / 8, 2))$

Das lässt sich so verstehen, dass der x-Wert für gro{\ss}e Absolutwerte von y (also weit oben und unten im Bild) besonders gestreckt wird. Es werden so bei der Erstellung des Bildes schon vor den Rändern Koordinaten erreicht, die über den Wertebereich der ursprünglichen Textur hinausgehen:

\includegraphics[height=100pt]{warp_x.png}

Diese Bereiche bleiben schwarz.

Für die y-Werte wird analog verfahren: $uv.y = uv.y * (1 + pow(abs(uv.x) / 8, 2))$

\includegraphics[height=100pt]{warp_y.png}

Nun werden dem Bild noch Scanlines und Vignette hinzugefügt. Da die Scanlines vertikal angeordnet sind, kann deren Stärke als Sinusfunktion der y-Koordinate dargestellt werden und wird wie folgt berechnet: $sin(texcoord.y * 6.28 * 100) / 2 + 0.5$ 

Die Faktoren 6.28 und 100 bedeuten, dass 100 Periodendurchläufe der Sinusfunktion (6.28 ist rund 2 Pi, also eine Periode) für y-Werte zwischen 0 und 1 geschehen. Da sich die Koordinaten in einem Wertebereich von -1 bis 1 befinden, entstehen auf der gesamten Displayhöhe insgesamt 200 Scanlines. Das Ergebnis der Sinusfunktion wird halbiert und es werden 0.5 addiert, womit es in einen Wertebereich zwischen 0 und 1 gebracht wird. Zuletzt wird der erhaltene Wert als Faktor mir der Farbe des Bildes multipliziert. 

Da starke Scanlines besonders in der Mitte des Bildschirmes die Sichtbarkeit erheblich beeinträchtigen können, wird eine zusätzliche Variable 'scanLineIntensity' angelegt, die als Faktor mit in das Ergebnis einflie{\ss}t und die Intensität der Scanlines entsprechend verringert. Dazu wird die Funktion 'smoothstep' verwendet: $scanLineIntensity = smoothstep(.8, 1.41422, length(texcoord))$. Somit erhält man ein Ergebnis von 0 (also keine Scanlines) für alle Bereiche, die weniger als 0.8 vom Zentrum entfernt sind, danach steigt die Intensität linear bis auf einen Wert 1 in den Ecken. Es entsteht folgendes Resultat:

\includegraphics[height=100pt]{scanlines.png}

Zuletzt erfolgt die Vignettierung der Ränder. Dabei wird wie bei der Displaykrümmung mit den Absolutwerten der Koordinaten gearbeitet, da die Vignette an allen Rändern symmetrisch auftritt. Wieder kommt smoothstep zum Einsatz. Es wird geprüft, ob der grö{\ss}ere der beiden Koordinatenwerte (mit 'max' bestimmt) nicht weiter als der übergebene Wert vignetteWidth vom Rand entfernt ist: $vignetteStrength = smoothstep(1 - vignetteWidth, 1, max(texcoord.x, texcoord.y))$

Damit der Effekt zur Mitte hin schneller abnimmt, wird das Ergebnis anschlie{\ss}end hoch 4 gerechnet, au{\ss}erdem wird der erhaltene Wert mit der zweiten Uniform Variable 'vignetteAmount' multipliziert, um die Stärke des Effekts vom Kontrollskript aus steuern zu können. Zusätzlich muss das Resultat mit 1 invertiert werden, um richtig mit den Farbwerten multipliziert werden zu können: $vignette = 1 - (pow(vignetteStrength, 4) * vignetteAmount)$
Für vignetteWidth gleich 0.1 und vignetteAmount gleich 0.8 sieht das Ergebnis wie folgt aus:

\includegraphics[height=100pt]{vignette.png}







\chapter{Assets und Tools}

\begin{description}
\item[Git] Versionsverwaltungstool
\item[Unity Engine] Game engine
\item[Unity Editor] Editor der Game Engine
\item[Microsoft Visual Studio] Code Editor mit Unity Integration
\item[Pro Builder] Simpler Mesh-Editor für Unity zum Erstellen der Entity-Modelle
\item[Ableton Live 10] Digital Audio Workstation zur Erstellung der Soundeffekte
\vspace{5mm}
\item[Modelle] Selbst erstellt mit Pro Builder
\item[Texturen] In Unity angelegt, ausschlie{\ss}lich einfarbig
\item[Audioeffekte] Selbst erstellt mit Ableton Live
\item[Soundtrack] Selbst erstellt mit Ableton Live

\end{description}






\chapter{Ausblick}

% Bossstages, mehr testdaten & statistiken
% mehr Inhalt & Abwechslung (generischere Gegner & Stages, mehr presets & Gegnertypen)
\lipsum[3]





\chapter{Fazit}

\lipsum[3]





\chapter{Glossar}

\begin{description}
\item[GUI] Graphical User Interface
\item[Prefab] Vorlage eines Unity-Objektes, welches beliebig oft in der Szene wiederverwendet werden kann ohne neu definiert werden zu müssen
\vspace{5mm}
\item[Shader] Programm, das auf der Grafikkarte ausgeführt wird
\item[Uniform] Variable, die aus dem Hauptprogramm aus für ein Shaderprogramm gesetz werden kann
\item[Compute Shader] Shader, der Kernels in mehreren Arbeitsgruppen ausführt
\item[Kernel] Bestandteil des Compute Shaders, enthält die Arbeitsanweisungen für die Arbeitsgruppen
\item[Billboard] Projektion eines Meshes auf ein Rechteck, welches immer richtung Kamera ausgerichtet wird
\vspace{5mm}
\item[Postprocessing] Nachbearbeitung (zum Beispiel eines Bildes)
\item[Bloom] Überblendung, helle Bereiche eines Bildes überstrahlen naheliegende Bereiche
\item[Lens Flare] Linsenreflexion, helle Bereiche eines Bildes erzeugen blasse Abbilder auf einer Linie in Richtung der Mitte das Bildes
\item[Ghosts] Begriff für besagte Abbilder des Lens Flare
\item[Chromatic Aberration] Chromatische Abweichung, Licht unterschiedlicher Wellenlängen wird in einem Bild unterschiedlich stark verzerrt
\item[Vignette] Abdunklung eines Bildes in der Nähe der Bildschirmränder
\item[Scanlines] Dunkle, horizontale Linien. Artefakt alter CRT Monitore
\end{description}




\chapter{Dateiverzeichnis}
\dirtree{%
.1 Assets/.
.2 Animation/.
.2 Materials/. .3 Shader/.
.2 Prefabs/.
.3 Actors/. .4 Enemies/.
.3 StageElements/.
.3 Bases/.
.2 Resources/.
.2 Scenes/.
.2 Scripts/.
.3 Shader/.
.3 Stages/.
.3 Test/.
.3 Actors/. .4 Player/. .4 Enemies/.
.3 StageElements/.
.3 Bases/.
.3 Controller/.
.2 Shader/. .3 Particles/. .3 PostProcessing/.
}




\chapter{Quellenverzeichnis}

\begin{thebibliography}{999}

\bibitem{qNierHM} Nier:Automata Hacking-Mode Demo vom 23.04.2021,  Zugriff:  16.05.2021, \\ URL:
\url{https://www.youtube.com/watch?v=jT2jOeFo5HQ}

\bibitem{qCtxSteer} Context-Steering Demo vom 10.10.2020,  Zugriff:  19.05.2021, \\ URL:
\url{https://www.youtube.com/watch?v=6BrZryMz-ac}
\url{http://www.gameaipro.com/GameAIPro2/GameAIPro2_Chapter18_Context_Steering_Behavior-Driven_Steering_at_the_Macro_Scale.pdf}

\bibitem{qPartS1} Drawing Thousands of Meshes in Unity vom 01.11.2019, Zugriff 17.06.2021
\url{https://toqoz.fyi/thousands-of-meshes.html}

\bibitem{qPartS2} GPUParticles vom 22.12.2017, Zugriff 22.06.2021
\url{https://github.com/Robert-K/gpu-particles/tree/master/Assets/GPUParticles}

\end{thebibliography}
    




\chapter{Eigenständigkeitserklärung}

Hiermit erklären wir Alexander Feilke und Jakom Massal{\ss}ky, dass wir die vorliegende Arbeit selbstständig verfasst und keine anderen als die angegebenen Quellen und Hilfsmittel benutzt haben.

Alle sinngemä{\ss} und wörtlich übernommenen Textstellen aus fremden Quellen wurden kenntlich gemacht.

Mittweida, den 10.08.2021

\vspace{2cm}

Alexander Feilke \hspace{4cm} Jakob Ma{\ss}alsky





\newpage
\newpage
\chapter{Anhang}




\section{Installationsanleitung}

Laden Sie die Datei \href{https://www.dropbox.com/s/g82vexjznq1x9dd/Cubix.zip?dl=1}{Cubix.zip} herunter und entpacken Sie sie. Anschlie{\ss}end führen Sie die darin befindliche cubix.exe aus.
Im Falle einer Warnung vom Windows-Defender klicken Sie auf Mehr und Trotzdem ausführen - Die Warnung besagt nur, dass das Spiel von keinem offiziellen Hersteller veröffentlicht wurde.
Sie müssten nun einen Unity-Startbildschirm sehen und kurz darauf direkt im Spiel landen.




\section{Spielanleitung}

Du bist ein kleiner Würfel in einer Welt von Kugeln, Gegnern und Portalen. Versuche so viele Stufen wie möglich zu schaffen und dabei so wenige Fehler wie möglich zu machen. Wähle zwischen 3 Schussmodi aus, und wenn du in der Klemme steckst und genügend Ressourcen hast aktiviere den Farbkombinationsmodus und Rette damit deine Haut.



\subsection{Steuerung}
%TODO: Screenshots Spielfeld, Stageelemente, Farbschema, Pausemenü

Gesteuert wird mit WASD. Linksklick drücken und halten zum kontinuierlichen Abfeuern von Schüssen. Schussfarbe wechseln mit 1, 2 und 3 auf der Tastatur oder Numpad.

Wenn genügend Ressourcen vorhanden sind (mindestens 2 voll aufgeladene Balken) Drücke Space, um für eine Zeit lang kombinierte Farbschüsse abzugeben.

Pausiert wird das Spiel mit Escape.





\end{document}
