\documentclass[a4paper,ngerman,12pt]{report}
\usepackage[a4paper,left=3cm,right=3cm,top=3cm,bottom=2.5cm,includefoot,head=18pt]{geometry}

\usepackage{hyphsubst} %[ngerman=ngerman-x-latest]
\usepackage[onehalfspacing]{setspace}            % Zeilenabstand 1.5
\usepackage[ngerman]{babel}               % Sprache


%\usepackage{step} %fbb,step
%\usepackage{newtxmath,newtxtext}                % pdflatex Times New Roman
%\usepackage[utf8x]{inputenc}
\usepackage[T1]{fontenc}

%\usepackage{blindtext}
%\usepackage{enumitem}

\usepackage{footnote}
\usepackage{longtable}
\usepackage[font={footnotesize,it}]{caption} % scriptsize
\captionsetup{singlelinecheck=true,justification=centering} % justification=raggedright,

%\usepackage[Q=yes]{examplep}
%\usepackage{algpseudocode}        % Algorithmen
%\usepackage{amsmath}              % mathematische Befehle
%\usepackage{amssymb}              % mathematische Symbole
%\usepackage{arev}                 % monospace for Codeblocks
\usepackage{color}                % Farben
\usepackage{subfig}
\usepackage{float}
%\usepackage{fontspec}             % Schriftart
%\usepackage{glossaries}           % Glossar
\usepackage{graphicx}             % Bilder
\usepackage{listings}             % Quelltext
\usepackage{multirow}             %
\usepackage{colortbl}			  % hline color
%\usepackage{lmodern}              % Beliebige Schriftgrößen
\usepackage{mdframed}
\usepackage{scrlayer-scrpage}     % Kopf- und Fußzeilen
\PassOptionsToPackage{hyphens}{url}

\usepackage[allcolors=blue]{hyperref}             % Links & Verweise (last)
\usepackage[nameinlink]{cleveref}             % Verweise         (last > hyperref)

\usepackage{lipsum}

% Link- / Verweiseinstellungen
\hypersetup{
    colorlinks,breaklinks,
    % urlcolor=[rgb]{0,0,1},
    linkcolor=[rgb]{0,0.1,0.6}
}

\renewcommand\citeform[1]{[#1]}

%\renewcommand{\ttdefault}{pcr}
%\renewcommand{\normalsize}{\fontsize{12}{0.231in}\selectfont}
%\renewcommand*\familydefault{\sfdefault}
%\setmainfont[Ligatures=TeX]{Times New Roman}     % Schriftart Times New Roman

% \setkomafont{pageheadfoot}{\textrm}
\cfoot{\thepage}                  % Fußzeile Zentriert = Seitennummer
\def\UrlFont{\em}                 % italic URL

\newcommand*{\fullref}[1]{\hyperref[{#1}]{\autoref*{#1} \nameref*{#1}}}
\newcommand*{\bitem}[1]{ \setbox0\hbox{\bfseries{#1}} \item[\usebox0]}
\newcommand*{\hitem}[1]{ \setbox0\hbox{#1} \item[\usebox0] \hfill \\}
\newcommand*{\hbitem}[1]{ \setbox0\hbox{\bfseries{#1}} \item[\usebox0] \hfill \\}
\newcommand\rurl[1]{%
	\href{http://#1}{\nolinkurl{#1}}%
}
\newcommand{\savefootnote}[2]{\footnote{\label{#1}#2}}
\newcommand{\repeatfootnote}[1]{\textsuperscript{\ref{#1}}}

\newenvironment{mldescription}{
    \begin{addmargin}[1em]{1em}
    \setlength{\parindent}{-1em}
    \newcommand*{\mlitem}[1]{\par{##1\hfill\\}\quad}\indent
}{
    \end{addmargin}
    \medskip
}


\newcommand{\sq}[1]{`#1'}     % single quote
\renewcommand{\dq}[1]{``#1''} % double quote
\newcommand{\mono}[1]{\lstinline[breaklines=true]{#1}}
\newcommand{\inlind}{\vspace{-.65cm}\hspace{.7cm}}
\newcommand{\greyline}{
	\arrayrulecolor{cllgrey}
	%\setlength{\arrayrulewidth}{.3em}
	\hline
	%\setlength{\arrayrulewidth}{.1em}
}
\newcommand{\dgreyline}{
	\arrayrulecolor{clgrey}
	%\setlength{\arrayrulewidth}{.3em}
	\hline
	%\setlength{\arrayrulewidth}{.1em}
}
\newcommand{\cross}[1]{\includegraphics[height=#1]{cross.png}}
\newcommand{\crosss}[1]{\includegraphics[height=#1]{cross.png}\hspace{2mm}}


\newcommand{\itmspace}{5em}
\newcommand{\imgspace}{3em}

\newcommand\imgtxt[2]{%
\begin{flushleft}%
\parbox{\imgspace}{\includegraphics[width=\linewidth]{#1}}%
\hfil
\parbox{\dimexpr\textwidth-\imgspace-1em}{#2}%
\end{flushleft}}

\newcommand\imgitmtxt[3]{%
\begin{flushleft}%
\parbox{\imgspace}{\includegraphics[width=\linewidth]{#1}}%
\hspace{2mm}
\parbox{\itmspace}{\textbf{#2}}%
\parbox{\dimexpr\textwidth-\imgspace-\itmspace-1em}{#3}%
\end{flushleft}}


% Benutzerdefinierte Farben
\definecolor{clgreen}{rgb}{0,0.6,0}
\definecolor{clmauve}{rgb}{0.58,0,0.82}
\definecolor{clorange}{rgb}{1.0, 0.49, 0.0}
\definecolor{cldgrey}{gray}{0.3}
\definecolor{clgrey}{gray}{0.5}
\definecolor{clgray}{gray}{0.95}
\definecolor{cllgrey}{gray}{0.75}

\lstset{
    basicstyle=\selectfont\sffamily,
    % postbreak=\mbox{\textcolor{red}{$\hookrightarrow$}\space},
    captionpos=b, keepspaces=true, numberfirstline=false,
    numbers=left, numbersep=9pt, showspaces=false, showstringspaces=false,
    numberstyle=\tiny\color{clgrey},
    showtabs=false, stepnumber=5, tabsize=4, title=\lstname, firstnumber=1
    %, literate={ß}{{\ss}}1 {€}{{\euro}}1 {£}{{\pounds}}1
    %{ä}{{\"a}}1 {ö}{{\"o}}1 {ü}{{\"u}}1 {Ä}{{\"A}}1 {Ö}{{\"O}}1 {Ü}{{\"U}}1
}

\lstset{literate=
  {á}{{\'a}}1 {é}{{\'e}}1 {í}{{\'i}}1 {ó}{{\'o}}1 {ú}{{\'u}}1
  {Á}{{\'A}}1 {É}{{\'E}}1 {Í}{{\'I}}1 {Ó}{{\'O}}1 {Ú}{{\'U}}1
  {à}{{\`a}}1 {è}{{\`e}}1 {ì}{{\`i}}1 {ò}{{\`o}}1 {ù}{{\`u}}1
  {À}{{\`A}}1 {È}{{\'E}}1 {Ì}{{\`I}}1 {Ò}{{\`O}}1 {Ù}{{\`U}}1
  {ä}{{\"a}}1 {ë}{{\"e}}1 {ï}{{\"i}}1 {ö}{{\"o}}1 {ü}{{\"u}}1
  {Ä}{{\"A}}1 {Ë}{{\"E}}1 {Ï}{{\"I}}1 {Ö}{{\"O}}1 {Ü}{{\"U}}1
  {â}{{\^a}}1 {ê}{{\^e}}1 {î}{{\^i}}1 {ô}{{\^o}}1 {û}{{\^u}}1
  {Â}{{\^A}}1 {Ê}{{\^E}}1 {Î}{{\^I}}1 {Ô}{{\^O}}1 {Û}{{\^U}}1
  {ã}{{\~a}}1 {ẽ}{{\~e}}1 {ĩ}{{\~i}}1 {õ}{{\~o}}1 {ũ}{{\~u}}1
  {Ã}{{\~A}}1 {Ẽ}{{\~E}}1 {Ĩ}{{\~I}}1 {Õ}{{\~O}}1 {Ũ}{{\~U}}1
  {œ}{{\oe}}1 {Œ}{{\OE}}1 {æ}{{\ae}}1 {Æ}{{\AE}}1 {ß}{{\ss}}1
  {ű}{{\H{u}}}1 {Ű}{{\H{U}}}1 {ő}{{\H{o}}}1 {Ő}{{\H{O}}}1
  {ç}{{\c c}}1 {Ç}{{\c C}}1 {ø}{{\o}}1 {å}{{\r a}}1 {Å}{{\r A}}1
  {€}{{\euro}}1 {£}{{\pounds}}1 {«}{{\guillemotleft}}1
  {»}{{\guillemotright}}1 {ñ}{{\~n}}1 {Ñ}{{\~N}}1 {¿}{{?`}}1 {¡}{{!`}}1 
}

\lstdefinestyle{codecolors}{
    keywordstyle=\color{blue}, 
    commentstyle=\color{clgrey}, 
    stringstyle=\color{clmauve}, 
    rulecolor=\color{black},
    basicstyle=\fontsize{9.8}{11}\selectfont\ttfamily,
}

\lstdefinestyle{codeblock}{
    columns=fullflexible, frame=tlbr, framesep=0pt, framerule=0pt,
    breaklines=true, aboveskip=10pt, belowskip=-5pt,
    backgroundcolor=\color{clgray}
}

\lstdefinestyle{codeblock1}{
    columns=fullflexible, frame=tlbr, framesep=0pt, framerule=0pt,
    breaklines=true, aboveskip=5pt, belowskip=-30pt
}

\lstdefinestyle{codeblock2}{
    columns=fullflexible, frame=tlbr, framesep=5pt, framerule=0pt,
	breaklines=true, aboveskip=-10pt, belowskip=-15pt, backgroundcolor=\color{clgray}
}
\lstdefinestyle{style_csh}{
    language=[sharp]c, style=codecolors,
    morecomment=[l]{\#},
    classoffset=1,
    morekeywords={
        % Custom enums
        Shape,
        % Custom classes
        Particles,
        % Custom structs
        Stats, GeneralProps, RenderSettings, DynamicEffect, Colors,
        % C# types
        Serializable,
        % Unity types
        MonoBehaviour, ComputeShader, ComputeBuffer, Material, Texture, Texture2D, Gradient,
        % Vector types
        Vector2, Vector3, Vector3Int, Vector4, Color,
        %  Attribute types
        Header, Tooltip, NotNull, WarnNull, Range
    },
    keywordstyle=\color{clgreen},
}

\newcommand{\cshinline}[1]{\lstinline[style=style_csh]{#1}}

\newcommand{\cshmlinline}[1]{
    \lstinline[style=style_csh,style=codeblock]{#1}
}

\lstnewenvironment{csh}[1][]{\lstset{style=style_csh, style=codeblock1, #1}}{}
\lstnewenvironment{csh2}[1][]{\lstset{style=style_csh, style=codeblock2, #1}}{}

\newcommand*{\cshitem}[1]{
  \setbox0\hbox{\cshinline{#1}}
  \item[\usebox0] \hfill \\
}

\newcommand*{\cshmlitem}[1]{
	\setbox0\hbox{\cshmlinline{#1}}
	\item[\usebox0] \hfill \\
}

\surroundwithmdframed[
    hidealllines=true,
    skipabove=12pt,
    skipbelow=16pt,
    backgroundcolor=clgray,
    innerleftmargin=5pt,
    innertopmargin=10pt,
    innerbottommargin=0pt
]{csh}
\include{code_hlsl}

\graphicspath{{img/}{img/game/}{img/particles/}{img/postproc/}
    {img/coll/}{img/enem/}{img/shot/}}

% Daten für Titelseite
\title{Kombinierte Belegarbeit \\ GameProgramming und \\ GPU Programmierung}
\author{Alexander Feilke, Jakob Ma{\ss}alsky \\ Sem.Gr.: IF19wS-B}
\date{August 2021}

%help: https://de.wikibooks.org/wiki/LaTeX-Wörterbuch
% \part \chapter \section \subsection \subsubsection \paragraph \subparagraph
% ß: {\ss}
% label{X} \nameref \pageref \fullref \vref {X}
% href{URL}{text}
% für Lit.Verz.: \cite[extra]{X,Y}



% Glossaries:
%\usepackage[chapter,toc]{glossaries}
%\newcommand{\printmyglossary}{\printunsrtglossaries}
%\newcommand{\printmyglossary}{\printglossaries}
%\makeglossaries % use TeX to sort
%\renewcommand{\glstextformat}[1]{\textit{#1}}
%\newglossaryentry{Basisklasse}{name={Basisklasse},
%description={Vererbt Eigenschaften an davon abgeleitete Klassen}}




\begin{document}

%\thispagestyle{empty}\quad\newpage % leeres Deckblatt

\normalfont

\makeatletter
\begin{titlepage}
\begin{center}
    \includegraphics[width=0.4\linewidth]{hsmwLogo.png}\\[10ex]
    %\selectfont\sffamily
    {\huge \@title}\\[10ex]
    {\Large Erstprüfer: \\
        Prof. Dr. rer. nat. Marc Ritter \\
        Dipl. Inf. Daniel Stockmann \\ \vspace{1em}
        Co-Prüfer: \\ Manuel Heinzig, M.Sc.
    }\\[20ex]
    {\Large \@author}\\[10ex]
    {\large \@date}
\end{center}
\end{titlepage}
\makeatother

%\maketitle
%\newpage % Titelblatt

\renewcommand{\contentsname}{Inhaltsverzeichnis}
\tableofcontents
\newpage


\lefthyphenmin 5
\righthyphenmin 5
\sloppy
\flushbottom



\chapter{Aufgabenstellung}

\lipsum[3]





\chapter{Spielbeschreibung}

%Konzept, ~farben
Cubix ist ein Top-Down Bullet-Hell Shooter angelehnt an den Hacking-Mode aus Nier:Automata\cite{qNierHM}.
Dort steuert man eine Figur in einer Welt von Gegnern und Hindernissen und muss möglichst schnell und Fehlerfrei sämtliche Gegner durch gekonntes Ausweichen und gezielte Schüsse eliminieren.

Als Neuerung wird ein Farbkonzept verwendet, bei dem die Grundfarben Rot, Grün und Blau mit verschiedenen Eigenschaften assoziiert werden. Alle Akteure des Spiels (sowohl der Spieler selbst als auch die Gegner und aufsammelbare Objekte) können in Beliebiger Kombination dieser Farben auftreten. Ist mehr als eine Farbe aktiv, so addieren sich deren Eigenschaften. Dieses System ist für den Spieler leicht verständlich, denn das Wissen über die Eigenschaften der drei Grundfarben reicht aus, um die Ergebnisse jeder möglichen Kombination schon im Voraus zu erschlie{\ss}en. Diese Kombinationsmöglichkeiten sorgen aber trotz des simplen Systems für genug Abwechslung und Tiefe. Die einzige Ausnahme beim Kombinieren bilden die sammelbaren Objekte, dazu später mehr (siehe \fullref{sect:Collectables}).




\section{Spielziel}

Das Spiel ist aus verschiedenen Stufen (siehe \fullref{sect:stages}) aufgebaut, die vom Spieler abgeschlossen werden müssen.
Solange sich der Spieler im roten Ladebereich (Charger) in der Mitte des Spielfeldes aufhält, wird dieser aufgeladen.
Sobald der Charger aktiviert wird, fangen Gegner an zu erscheinen (spawnen). Diese können vom Spieler durch Schie{\ss}en besiegt werden, sie selbst schie{\ss}en aber ebenfalls auf den Spieler und können ihm so schaden.
Ist der Charger vollständig geladen, hören die Gegner auf zu spawnen und ein Portal zur nächsten Stufe wird aktiviert.




\section{Schussmodi}

Der Spieler kann seinen Schüssen jeweils eine der Eigenschaften der drei Grundfarben verleihen, welche er im Tutorial der Reihe nach freischalten und ausprobieren kann. Die ausgewählte Farbe kann jederzeig gewechselt werden:

\renewcommand{\itmspace}{4.5em}
\imgitmtxt{s_red.png}{Rote Kugeln}
{sind langsamer aber explodieren beim Aufprall.}

\imgitmtxt{s_green.png}{Grüne Kugeln}
{sind schneller und fügen mehr Schaden zu, prallen von Wänden ab und durchdringen Gegner, haben aber eine geringere Schussrate.}

\imgitmtxt{s_blue.png}{Blaue Kugeln}
{werden in einem Fächer von fünf Kugeln in einer höheren Schussrate verschossen, fügen aber weniger Schaden zu.}

Die gleichzeitige Kombination mehrerer Farben ist nur auf Kosten von Ressourcen erlaubt  (siehe \fullref{sect:ressourcen}). Dabei erhalten die Kugeln alle Eigenschaften der aktiven Farben.
Beispiel (in Klammern jeweils die Farbe, von der der Effekt stammt): Kombiniert man die Farben rot und grün, so erhält man gelbe Kugeln, die etwas langsamer als Grüne sind (durch rot wird der Bonus von grün verringert) und eine geringere Feuerrate haben (grün). Sie durchdringen Gegner und prallen von Wänden ab (grün), wobei jeder Kontakt eine Explosion erzeugt (rot), die zusätzlich mehr Schaden verursacht (grün).

Bei den anderen Kombinationen (magenta, cyan, wei{\ss}) ist analog zu verfahren.


\section{Gegner}

Im Spiel gibt es drei verschiedene Gegnertypen, erkennbar an ihrer Form:

\renewcommand{\itmspace}{4.5em}
\imgitmtxt{e_hunter.png}{Hunter}
{jagen den Spieler um ihm Kollisionsschaden zuzufügen.}

\imgitmtxt{e_archer.png}{Archer}
{verfolgen den Spieler aber bleiben auf Abstand und schie{\ss}en aus der Ferne auf ihn.}

\imgitmtxt{e_stray.png}{Strays}
{irren langsam über das Spielfeld und feuern 4 Schüsse in alle Richtungen.}

Auch die Gegner können in drei verschiedenen Farben vorkommen:

\renewcommand{\itmspace}{7.5em}
\imgitmtxt{e_red.png}{Rote Gegner}
{sind langsam, haben dafür aber mehr Leben.}

\imgitmtxt{e_green.png}{Grüne Gegner}
{bewegen sich schneller.}

\imgitmtxt{e_blue.png}{Blaue Gegner}
{spawnen in Zweiergruppen, haben aber weniger Leben.}

Bei der Kombination der Gegnerfarben ist ebenfalls wie bei den Schussfarben vorzugehen. Beispiel:

\imgitmtxt{e_white.png}{Wei{\ss}e Gegner}
{spawnen als Kombination aller drei Farben und haben mehrere Boni. Sie sind so schnell wie grüne Gegner, haben etwas weniger Leben als Rote und spawnen ebenfalls in Gruppen.}




\section{Collectables}

Wenn Gegner besiegt werden, lassen sie zu einer gewissen Wahrscheinlichkeit aufsammelbare Objekte (Collectables) abhängig von ihrer eigenen Farbe fallen.

\renewcommand{\itmspace}{5.5em}
\imgitmtxt{c_black.png}{Schwarz}
{lässt den Spieler einen Lebenspunkt zurückerhalten.}

\imgitmtxt{c_blue.png}{Rot, Grün, Blau}
{verleihen dem Spieler Ressourcen in der entsprechenden Farbe.}

\imgitmtxt{c_yellow.png}{Gelb}
{lässt den Spieler für kurze Zeit in alle Richtungen feuern.}

\imgitmtxt{c_cyan.png}{Cyan}
{erhöht kurzzeitig die Schussrate stark.}

\imgitmtxt{c_magenta.png}{Magenta}
{verleiht für einen kurzen Zeitraum Unsichtbarkeit/Unverwundbarkeit.}

\imgitmtxt{c_gold.png}{Gold}
{lässt einen goldenen Container fallen der bei Kontakt eine gro{\ss}e Explosion auslöst, die sämtliche Gegner auf dem Spielfeld auslöscht. (Selten und nur von wei{\ss}en Gegnern erhältlich.)}




\section{Ressourcen}

Für jede der drei Grundfarben existiert eine entsprechende Ressource, deren Stand in der Ressourcenanzeige oben rechts im Fenster angezeigt wird:

\includegraphics[height=100pt]{ressources.png}

Ressourcen erhält man durch das Aufsammeln von entsprechend gefärbten Collectables und das Besiegen von farbigen Gegnern. Die erhaltenen Ressourcen werden für mehrfarbige Gegner auf die entsprechenden Grundfarben aufgeteilt, so gibt zum Beispiel ein Cyanfarbiger Gegner blaue und grüne Ressourcen.

Benutzen kann man Ressourcen, sobald die Balken von mindesten zwei Farben vollständig gefüllt sind. Drückt man nun die Taste zum Aktivieren der Farbkombination (Leertaste), so werden den Schüssen die Eigenschaften der aufgefüllten Farben verliehen und die entsprechenden Ressourcen leeren sich über ungefähr 10 Sekunden. Die Farbkombination bleibt so lange aufrecht erhalten, bis eine der benutzten Ressourcen leer ist. Während dieser Zeit kann die Schussfarbe nicht mehr manuell gewechselt werden, dafür ist eine Kombination aus mehreren Farben deutlich stärker als jede einzelne Farbe.

Diese stärkere Phase wird also durch die Ressourcen zeitlich begrenzt, wodurch sie sich für den Spieler besonders anfühlt und dieser darauf hinarbeiten kann, sie erneut zu erleben. Zusätzlich lassen sich die Ressourcen auch während sie aktiv sind weiter aufladen (zum Beispiel durch das Töten von Gegnern), wodurch ein agressiver, spannender Spielstil gefördert wird.



\section{Interface und Hauptmenü}

Um die Immersion zu steigern, ist das Interface auf das Nötigste beschränkt. Allein ein Knopf zum Pausieren und die Ressourcenanzeige bilden gemeinsam das GUI. Ist das Spiel pausiert, können über insgesamt 5 Knöpfe die grundlegensten Funktionen wie 'Fortfahren', 'Spiel beenden' oder 'zurück ins Hauptmenü'  aufgerufen werden. 
Das Hauptmenü selbst ist dabei kein wirkliches Menü, sondern bereits die erste Stufe. Anstatt wie üblich per Knopfdruck einen bestimmten Spielmodus auszuwählen (hier Tutorial oder Endlos), muss der Spieler erst den Charger aufladen und dann zum entsprechenden Portal navigieren. Gegner erscheinen hier noch keine. Hinweise für den Spieler, wie zum Beispiel die grundlegende Steuerung oder Informationen über Schussfarben und Gegner, erscheinen ebenfalls nicht im GUI, sondern als Text auf dem Boden. 

So ist das Hauptmenü schon stark mit dem Gameplay verbunden, es gibt einen nahtlosen Übergang zum eigentlichen Spiel und der Fokus wird auf das Wesentliche gesetzt.

Auf eine herkömmliche Lebensanzeige des Spielers wurde auch verzichtet, stattdessen zeigen kleine um die Spielfigur kreisende Würfel die aktuellen Lebenspunkte an.



\section{Tutorial}

Das Tutorial besteht aus einer Folge von 10 Stufen (das Hauptmenü eingeschlossen), in denen der Spieler der Reihe nach mit den einzelnen Spielmechaniken konfrontiert wird. Ereignisgesteuerte Hinweistexte erklären neue Sachverhalte, sodass sich der Spieler mit ihnen vertraut machen kann. Das Tutorial ist wie folgt aufgebaut:

\begin{description}
\item[Hauptmenü] Steuerung, Spielmodusauswahl
\item[Stufe 1] Gegner, Schie{\ss}en
\item[Stufe 2] Erste Schussfarbe freischalten
\item[Stufe 3] Gegner in der freigschalteten Farbe erscheinen
\item[Stufe 4] Zweite Schussfarbe freischalten
\item[Stufe 5] Farbkombimnation erlaubt
\item[Stufe 6] Zweifarbige Gegner erscheinen
\item[Stufe 7] Dritte Schussfarbe freischalten
\item[Stufe 8] Möglichkeit zum Ausprobieren der neuen Farbkombinationen
\item[Stufe 9] Wei{\ss}e Gegner, letzte Herausforderung vor dem Endlosmodus
\item[Stufe 10+] Endlosmodus
\end{description}

Sobald Stufe 10 das erste mal erreicht wurde, wird die Möglichkeit freigeschalten, den Endlosmodus vom Hauptmenü aus direkt zu starten und das Tutorial zu überspringen.



\chapter{Dokumentation}




\section{Software-Architektur}

\lipsum[1]

\begin{figure}[h!]
\begin{center}
\includegraphics[height=170pt]{cross.png}
\caption{Klassendiagramm}
\label{fig:classdiag}
\end{center}
\end{figure}
    



\section{Stages}
\label{sect:stages}


\subsection{Aufbau}

Eine Stufe besteht im Wesentlichen aus 2 Gruppen von Komponenten: den statischen, unveränderlichen und den dynamisch generierten.

\renewcommand{\imgspace}{8em}
\imgtxt{stageInspect.png}{
Zu den statischen gehören Wände, die den Spielbereich abgrenzen und ggf. Hindernisse auf dem Spielfeld darstellen, der Spawn auf dem der Spieler die Stufe betritt, das Portal durch das der Spieler in die nächste Stufe gelangt und der Charger, den der Spieler aufladen muss, um das Portal zu aktivieren. (siehe \fullref{img:stage}) \\

Zu den dynamischen (im Bild links markiert) gehören die Gegner, abgefeuerte Bullets, Collectables, ggf. dynamisch generierte Hindernisse sowie dazugehörige Partikeleffekte. Zudem kann es in einer Stufe ein oder mehrere für den Spieler unsichtbare Gegnerspawner (siehe \nameref{subsect:spawner}) geben.
}

Alle dynamischen Objekte werden in einem dafür angelegten Containerobjekt in der Stage hinterlegt, damit er in seiner Gesamtheit beim Entladen einer Stufe gelöscht werden kann.



\subsection{Tutorialstages}

Das Tutorial besteht aus geskripteten Stages mit voreingestellten Spawnern, Farbcollectables und Hinweistexten. Die Hinweistexte und manchmal auch andere Komponenten werden hierbei durch zu den Stages gehörigen Skripte gesteuert. Dabei kann es unterschiedliche Verhalten geben, wenn der Endlosmodus bereits freigeschalten wurde, bspw. für die Anzeige von Statistiken und dem Highscore im Hauptmenü.



\subsection{Stagegenerator}

\lipsum[3]



\section{Spieler}

\lipsum[3]



\subsection{Input}

\lipsum[3]


\subsection{Movement}

\lipsum[3]


\subsection{Shooting}

\lipsum[3]


\subsection{Ressource}

\lipsum[3]




\section{Gegner}

\lipsum[3]



\subsection{Context-Steering}

Um das Gefecht interessanter zu machen wurde die Gegner-KI mit dem sog. Context-Steering Algorithmus\cite{qCtxSteer} (auch gen. Behavior-Driven Steering) ausgestattet.

Grundlegend wird hierbei jedem Objekt ein Wert zugeordnet der angibt, wie anstrebenswert es von bestimmten Entitäten ist. Ein positiver Wert ist hierbei ein Ziel, welches verfolgt wird, während ein negativer Wert ein zu vermeidendes Objekt darstellt.

In dieser Implementierung gibt es sog. Effektoren, die jeweils vier Werte speichern:

\begin{description}
\item[Zieltag] Zeichenkette, die die Gruppe an Objekten mit dieser Begehrlichkeit beschreibt
\item[Begehrlichkeit] wie anstrebenswert ein Objekt ist. Ein positiver Wert ist hierbei ein Ziel, welches verfolgt wird, während ein negativer Wert ein zu vermeidendes Objekt darstellt.
\item[Mindestabstand] Wie nah eine Entität sein muss, um von diesem Effektor beeinflusst zu werden
\item[Typ] Wie stark der Effektor abhängig vom Abstand wirkt. Das kann hier entweder logisch, linear oder quadratisch sein. 
\end{description}

Jede Entität bestimmt dann anhand der Formel $ clamp(typed(1 - dist / dist_{min}), 0, 1) * v_{dir} $ einen Vektor der die die Richtung und die Kraft zu diesem Objekt angibt.

Für die Erkennung von Wänden schaut jede Entität in in 30° Schritten um sich herum und schickt in jede dieser Richtungen einen Physik-Raycast. Von den Effektor-Vektoren wird dann  mittels Punktprodukt ein anteiliger Faktor in diese Richtung ermittelt und die resultierende Richtung durch die Summe dieser Richtungsvektoren ermittelt. Für etwas mehr Abwechslung wird diese noch durch Perlin-Noise ein wenig abgeändert.

\begin{figure}[H]
\centering
\includegraphics[width=0.7\linewidth]{ctxSteer.png}
\caption{Debug-Linien vom Kontext-Steering}
\label{img:ctxSteer}
\end{figure}

Da jede Gegnerart ein eigenes Verhalten hat, haben diese auch unterschiedliche Effektorenlisten. Strays besitzen nur das Standardverhalten und bewegen sich rein durch noise und vermeiden Wände und andere Gegner. Hunter haben zusätzlich eine Kraft zum Spieler hin und Archer wiederum noch einen Effektor mit kleinerem Radius vom Spieler weg, ebenfalls mit etwas Noise damit der Archer scheinbar vom SPieler strategisch vor- und zurückweicht.



\subsection{Gegnerspawner}
\label{subsect:spawner}

Die Gegnerspawner sind standardisierte Rechteck-Objekte, die unsichtbar für den Spieler einen Bereich markieren in dem Gegner mit einstellbaren Wahrscheinlichkeiten und Eigenschaften erscheinen:

\begin{description}
\item[Typverteilung] Wahrscheinlichkeitsverteilung der Gegnertypen.
\item[Farbverteilung] Wahrscheinlichkeitsverteilung der Farbmodifikatoren, abhängig von der Reihenfolge, in der die Schussfarben eingesammelt wurden.
\item[maxColors] maximale Anzahl an Farben die die Gegner haben können
\item[amount] maximale Anzahl an Gegner die dieser Spawner spawnt
\item[wavesize] Anzahl an Gegnern die der Spawner auf einmal spawnt (typischerweise 1-3)
\item[initDelay] Zeitspanne die der Spawner nach Aktivierung des Chargers wartet bis er anfängt Gegner zu spawnen
\item[delay] Zeitspanne zwischen Gegnerwellen
\item[variation] Zeitspanne, nach der einzelne Gegnerspawns verzögert werden
\end{description}

\begin{figure}[H]
\centering
\includegraphics[width=\linewidth]{spawner.png}
\caption{Spawner im Unity-Inspektor}
\label{img:spawner}
\end{figure}

Zudem hat jede Stage eine Maximale Anzahl an Gegnern, die sich Zeitgleich in der Stage aufhalten können, damit der Spieler nicht von Gegnern überrant wird.

In jeder Stage können beliebig viele Spawner platziert werden, jedoch ist durch die verfügbaren Einstellungsmöglichkeiten schon mit einem einzelnen Spawner meist ausreichend Variabilität möglich. Mehrere Spawner werden erst vonnöten, wenn man in einer Stage zeitlich verteilt verschiedene Wellen spawnen möchte.




\section{Physics}

\lipsum[3]



\section{Sound}

\lipsum[3]


\chapter{Grafikeffekte}




\section{Shader in Unity}

\lipsum[3]


\section{Partikelsystem}

Ein Partikelsystem ist im Bereich der Computersimulation die Simulation einer Ansammlung gleichartiger Teilchen, deren Verhalten durch Zufälligkeit variiert werden kann. Partikelsysteme erlauben durch die Auslagerung der Berechnungen auf die Grafikeinheit eines PCs die flüssige Simulation einer gro{\ss}en Menge an Teilchen auf einmal (auf einer NVidia 1070Ti bis zu 10 Mio. Partikel mit 60fps in dieser Implementierung). Diese werden zumeist für die Simulation von Feuer, Rauch, Flüsigkeiten, Explosionen und ähnlich komplexen physikalischen Vorgängen verwendet.

In Cubix werden Partikelsysteme an verschiedensten Stellen eingesetzt, um die Optik aufzuwerten und das visuelle Feedback von Explosionen und Treffern zu verstärken. Zudem werden Partikel für Portale, für Bewegungsspuren, Ressourcen und für Explosionen eingesetzt.



\subsection{Implementierung}

Das Partikelsystem wurde zunächst auf Basis eines Artikels über das Zeichnen von \sq{tausenden Meshes auf einmal}\cite{qPartS1} in Unity basierend auf der Render-Funktion \href{https://docs.unity3d.com/ScriptReference/Graphics.DrawMeshInstancedIndirect.html}{DrawMeshInstancedIndirect} konstruiert.

Für die gesteuerte Emission von Partikeln wurde später ein GitHub-Repository eines ebenfalls selbst implementierten Partikelsystems\cite{qPartS2} zu Rate gezogen. Die wesentliche Neuerung war die etwas einfacher zu verwendende Render-Funktion \href{https://docs.unity3d.com/ScriptReference/Graphics.DrawProceduralNow.html}{DrawProceduralNow} und das Benutzen von Gradient-Texturen zur Farbänderung.

\begin{figure}[H]
\centering
\includegraphics[width=0.8\linewidth]{partInspect.png}
\caption{Ansicht im Unity-Inspektor}
\label{img:partInspect}
\end{figure}


\subsection{Aufbau}

Das Partikelsystem besteht aus einer Controller-Klasse in CSharp, welche sämtliche Einstellungen für das Partikelsystem dem Objektinspektor bereitstellt\#, einem Compute-Shader welcher Kernel für das Initialisieren, Emittieren und Aktualisieren von Partikeln besitzt sowie einem Unlit-Shader für die Darstellung der Partikel inklusive Billboard.


\subsection{Partikel-Struktur}

Um den aktuellen Status eines einzelnen Partikels zu repräsentieren sollten möglichst kompakt alle für die individuelle Bewegung und Darstellung notwendigen Daten gespeichert werden. Dazu gehören die aktuelle Position, Geschwindigkeit und einwirkende Kraft, die Farbe und grö{\ss}e des Partikels und ein Random-Seed wert, der ggf. für zufälliges Verhalten verwendet wird.

\begin{csh}[caption=Partikelstruktur,label=lst:particle]
public struct Particle
{
    public Vector3 pos, vel, force;
    public Vector4 col, size; // xy: size,  z: age, w: maxage
    public float seed;
}
\end{csh}


\subsection{Partikel-Verhalten}

Darüber hinaus gibt es Variablen, die für jedes Partikel im Partikelsystem gleich gelten. Diese bestimmen, wie genau sich Partikel verhalten. Dazu gehören Ausgangswerte aller Positionsbezogenen Daten (Spawnposition und Geschwindigkeit), Grö{\ss}e, Farbe und Lebensdauer und wie sie sich über letztere hinweg verändern.

\begin{csh}
public class Particles : MonoBehaviour
{
    public Stats stats;                   // ps statistics for inspector
    public GeneralProps properties;       // general ps properties
    public RenderSettings renderSettings; // render settings (alpha blending)

    public DynamicEffect pos, vel, force, posFac; // dynamic mechanic effects
    public Colors color;          // Color [with variation & gradient]
    public Capsule<Vector3> size; // xy: offset, z: timefac
    public Vector4 attractor;     // Vector3 + w: factor
    /* ... */
}
\end{csh}



\subsubsection{Positionsdaten}

Für Positionsdaten wird eine allgemeine Struktur mit standardisierter Uniform-Benennung verwendet, die einen Ausgangsvektor, einen Abweichungsvektor und eine Form in der diese Abweichung auftritt, definiert.

\begin{csh}[caption=DynamicEffect Struktur]
public enum Shape { DOT, CIRCLE, RECT, SPHERE, CUBE };
public struct DynamicEffect
{
    public Vector3 offset, scale;
    public Shape shape;
}
\end{csh}



\subsubsection{Farbdaten}

Farbdaten werden entweder absolut oder als Gradient angegeben. Dabei kann optional Variation durch zufällige Interpolation zwischenden Farbwerten erreicht werden. Die Gradiente werden als dynamisch generierte Textur an den Shader gereicht, die absoluten Farbwerte als einfache Uniforms.

\begin{csh}[caption=Generierung der Gradient-Textur]
int height = useVariation ? 2 : 1;
tex = new Texture2D(steps, height);
for (int i = 0; i < steps; i++)
{
    tex.SetPixel(i, 0, gradient.Evaluate(i / (float)steps));
    if (useVariation) tex.SetPixel(i, 1, gradient2.Evaluate(i / (float)steps));
}
tex.Apply();
\end{csh}



\subsection{Partikelsystem Eigenschaften}

Zu guter Letzt hat ein Partikelsystem noch einige globale Eigenschaften, die deren Verhalten definieren. Dazu gehört die maximale Partikelzahl, die Emmitierrate, die Lebensdauer eines Partikels, sowie ob das PS aktiv ist, ob es sich wiederholt (d.h. ständig generiert wird oder nur bis insgesamt die maximale Partikelzahl erreicht wurde), ob es vorwärmen (d.h. einen Lebenszyklus beim Initialisieren emulieren soll) und einige weniger Relevante Eigenschaften auf die ich ggf. später zurückkomme.

\begin{csh}[caption=Partikel Struktur]
public struct GeneralProps
{
    public bool enabled, repeat, prewarm;
    public PerformanceMode performance;
    public int maxParts;
    public float startDelay, emissionRate, lifetime;
}
\end{csh}



\subsection{Berechnung}


\subsubsection{Setup}

Um die Partikel möglichst effizient berechnen zu können werden Compute Shader verwendet. Alle Daten die an den ComputeShader gegeben werden sollen werden in sog. ComputeBuffer geschrieben. Diese werden direkt auf der Grafikkarte abgelegt. Das sind zum einen sämtliche Partikeldaten als Standard StructuredBuffer und ein AppendBuffer von IDs toter Partikel.

\begin{csh}[caption=Controller Setup,label=lst:partSetup]
// Calculation
[NotNull] public ComputeShader compute;
private ComputeBuffer particlesBuf, deadBuf, counterBuf;

// Display
private ComputeBuffer quadVertBuf;
public RenderSettings renderSettings = RenderSettings.Default;
[NotNull] public Texture tex;  // particle texture
[NotNull] public Material mat; // UnlitShader rendering material
\end{csh}

Der ComputeShader beinhaltet an dem Punkt nur die Deklarationen der Compute-Kernel und der Partikelstruktur (siehe \fullref{lst:particle}), sowie einer Variable für den Partikel-Buffer.

\begin{hlsl}[caption=Compute Setup]
#pragma kernel Init
#pragma kernel Emit
#pragma kernel Update
#include "./Particle.cginc"
RWStructuredBuffer<Particle> _Particles;
\end{hlsl}

\subsubsection{Initialisierung}

Bei der Initialisierung werden im Init-Kernel sämtliche Partikel IDs dem deadBuffer hinzugefügt und deren Lebenszeit auf 0 gesetzt. Das ist der Ausgangszustand, bei dem es keine lebendingen Partikel gibt.

\subsubsection{Emission}

Um Partikel zu emmittieren wird in zu jedem Frame-Update geprüft, wie viele Partikel abhängig von der Spawn- und der Framerate gespawnt werden sollen. Um auch sehr niedrige Raten zu unterstützen wird eine Partikelzeit mitgezählt, die jedes Frame um die Frametime erhöht, und bei jedem Emit um die Emissiontime gesenkt wird. So kann bspw. auch bei einer Bildrate von 60fps ein einzelnes Partikel pro Sekunde emmittiert werden, da die Zeit zwischen zwei Frames bei dem kein Partikel emmittiert wird nicht verloren geht.

\begin{csh}[caption=Controller Emission]
if (properties.enabled && properties.emissionRate > 1e-2)
{
    partTimer += Time.deltaTime;
    int emitted = DispatchEmit((int)(partTimer * properties.emissionRate));
    partTimer -= emitted / properties.emissionRate;
}
\end{csh}

Bevor die Partikel tatsächlich gespawnt werden wird noch die Anzahl gegen verschiedene Grenzen getestet - ua. die maximal lebendige Partikelzahl und bei nichtwiederholendem PS die bisher insgesamt emmittierte Partikelzahl.

Dann werden dem Compute-Shader sämtliche für den Spawn relevanten Informationen per Uniform überreicht. Ua. wie beim Init-Kernel die Partikeldaten und die IDs toter Partikel - diesmal aber als Comsume- statt AppendBuffer, da wiederbelebten Partikel nicht tot sind und somit aus dem deadBuffer entfernt werden müssen. Weiterhin werden einige Flags als Bitmaske in einem Integer übergeben, um ggf. den Datenverkehr gering zu halten, sowie eine Reihe an Zufallswerten für die Randomisierung einiger Werte.

\begin{csh}[caption=Controller Emission Dispatch]
// calculate actually possible emission count
count = Mathf.Min(count, 1 << 15, curMaxParts - stats.alive);
if (!properties.repeat && stats.emitted + count > curMaxParts)
    count = curMaxParts - stats.emitted;

UniformEmit(kernelEmit);
compute.Dispatch(kernelEmit, count, 1, 1);
\end{csh}

Im Shader werden nun Partikel aus dem DeadBuffer entnommen und diese mit Eigenschaften eines Lebendingen Partikels initialisiert. Um eine zufällige Streuung von Vektoren ua. für die Positionsdaten zu erreichen, wurde eine Funktion definiert, die aus drei Eingabewerten mittels einer stark schwankenden Funktion einen neuen, möglichst zufälligen Wert generiert. Damit kann durch Umordnung der vier einmalig im Hauptprogramm berechneten Seedwerte, kombiniert mit der Partikel-ID eine Vielzahl an Pseudozufallswerten generiert und das Verhalten der Partikel individualisiert werden.

Um die Partikel in einem bestimmten Bereich spawnen zu lassen werden Scales verwendet, zwischen dem die resultierenden Vektoren schwanken können. Um noch etwas mehr Variabilität zu sorgen kann man die Vektoren mittels einer Shape-ID 'formen'. Derzeit werden ausschlie{\ss}lich DOT (formlos) CIRCLE, RECT (2D), SPHERE und CUBE (3D) unterstützt. Die Dimensionsunterscheidung muss unternommen werden um die Vektoren nicht auf einer Achse zu verteilen, die garnicht verwendet wird und dann dort gehäuft auftreten.

\begin{figure}[H]
\centering
\includegraphics[width=0.8\linewidth]{partShapes.png}
\caption{Spawhshapes}
\label{img:partShapes}
\end{figure}

In folgendem Listing wurde noch beispielhaft die Berechnung der Spawnposition gezeigt, die von Geschwindigkeit und Beschleunigung erfolgt analog, wenn auch aus Intuitivitätsgründen nicht 100\% identisch.

\begin{hlsl}[caption=Compute Emit Kernel]
uint id = _Alive.Consume(); // pop dead particle to alive
Particle p = _Particles[id];

p.size.xy = float4(_Size, _Lifetime);
p.rand = rand(_Seeds.wzy);
p.color = F(P_CLRVARY) ? lerp(_Color, _Color2, p.rand) : _Color;

// position
float3 d = float3(ran(_Seeds.xzy), ran(_Seeds.yxz), ran(_Seeds.zyx));
d = getShaped(d, _PosShape);
p.pos = d * (_PosOffset + _PosScale * rand(_Seeds.wxy));
/* ... */
\end{hlsl}


\subsubsection{Update}

Bei jedem Frame-Update wird der Status jedes Partikels aktualisiert. Dazu werden nur noch die Daten benötigt, die das Partikel während seiner Lebensspanne von au{\ss}en verändern. Dazu gehört in jedem Fall die Zeitdifferenz zum letzten Update, die Statusflags, der Farbgradient falls verwendet, sowie die Grö{\ss}enänderung, da diese bisher für alle Partikel gleich ist. Des weiteren kann man ein Objekt definieren, zu dessem Position das Partikel konstant hingezogen werden soll. Das wurde ua. für den Ressourcenpartikeleffekt benötigt.

\begin{csh}[caption=Controller Update]
DispatchUpdate();
if (properties.performance > PerformanceMode.LOW)
{
    _alive -= Time.deltaTime * Mathf.Ceil(stats.alive) / properties.lifetime;
    stats.alive = Mathf.CeilToInt(_alive);
    stats.dead = curMaxParts - stats.alive;
}
\end{csh}

Im Shader werden sämtliche Änderungen an einem Partikel abhängig von der Zeitdifferenz vorgenommen. Erreicht es hierbei ein Alter grö{\ss}er als die Lebensdauer, wird es wieder als tot markiert und dem DeadBuffer hinzugefügt. Bereits tote Partikel werden übersprungen, es wird aber dennoch für jedes Partikel die Update-Funktion aufgerufen, da der Shader das nicht im vorhinein bestimmen kann. Die Farbe wird ggf. aus der Gradienttextur gesampled und interpoliert.

\begin{hlsl}[caption=Compute Update]
float4 getLifeColor(float2 life, float seed);
/* ... */

if (p.size.w == 0) return; // skip dead
if (p.size.z < _DeltaTime) /* make dead */;

float3 magnet = _Attractor.w * normalize(_Attractor.xyz - p.pos);

p.size.z -= _DeltaTime;             // aging
p.size.xy += _SizeVel * _DeltaTime; // growth
p.vel += p.force * _DeltaTime;      // acceleration
p.pos += (p.vel + magnet) * _DeltaTime; // movement
if (F(P_CLRGRAD)) p.color = getLifeColor(p.size.zw, p.rand);
\end{hlsl}

Der Update-Kernel hat hierbei eine zusätzliche Funktion zum Vorwärmen das Partikelsystems. Beim Vorwärmen eines Partikelsystems werden beim Zurücksetzen zunächst so viele Partikel emmittiert, wie in einem Lebenszyklus durchschnittlich leben würden. Dann berechnet der Shader für jedes lebendige Partikel ein zufälliges Alter und führt so oft die Update-Funktion mit der eingestellten Deltatime auf, bis dieses erreicht ist.

\begin{hlsl}[caption=Compute Prewarm Update Kernel]
Particle p = _Particles[id.x];
float age = p.size.w * p.rand - _DeltaTime;
for (float i = 0; i < age; i += _DeltaTime) DoUpdate(id);
\end{hlsl}

Nach jedem Update wird anhand der durchschnittlichen Sterbenderate $r_{mort} = n_{alive} / t_{alive}$ berechnet, wie viele Partikel aktuell am leben sind. Das ist erheblich effizienter im Vergleich zu einer vorherigen Version, in der ein separater CounterBuffer dazu verwendet wurde, die genaue Grö{\ss}e des deadBuffers auszulesen. Da dieser physisch nur auf der Grafikkkarte existiert scheint Unity dazu eine Funktion namens Gfx.GetComputeBufferData\_Request zu benutzen, die zur Synchronisation eine Semaphore benutzt, was schon bei wenigen Partikelsystemen erhebliche Performanceeinbu{\ss}en mit sich brachte.

\begin{figure}[H]
\centering
\includegraphics[width=0.95\linewidth]{partBufCount.png}
\includegraphics[width=0.95\linewidth]{partDataReq.png}
\caption{ReadDeadCount im Unity Profiler}
%\label{img:partBufCount}
\end{figure}
    

\subsection{Darstellung}

Das Rendern der Partikel erfolgt über einen Unlit-Shader. Das Grundgerüst besteht aus einer Shader-Deklaration mit allen vom Inspektor übergebenen Eigenschaften, ua. die Rendertextur und BlendModes, die für korrektes Alpha-Blending nach Bedarf eingestellt werden können. Die BlendModes werden in der Struktur RenderSettings hinterlegt, zusammen mit einigen voreingestellten Modi\footnote{nach Artikel \href{https://elringus.me/blend-modes-in-unity}{BlendModes in Unity} vom 12.05.2015, Zugriff 23.06.2021}.

Dem Folgt ein Subshader-Pass der sämtliche Rendereinstellungen vornimmt. Da Partikel transparent sind, wird die Transparent queue zum Rendern verwendet. Es ist au{\ss}erdem wichtig ZWrite auszuschalten, da die Partikel sonst nicht korrekt sortiert werden (siehe \fullref{img:zoff}). Darauf folgt schlie{\ss}lich der tatsächliche Shader-Code des Fragment und des Vertex-Shaders.

\begin{figure}[H]
\centering
\includegraphics[width=0.5\linewidth]{zwrite.png}
\caption{ZWrite links: On, rechts: Off}
\label{img:zoff}
\end{figure}


Die Variablen für das Rendering sind im \fullref{lst:partSetup} bereits enthalten. Erwähnenswert ist der Buffer für Vertexkoordinaten eines Partikels - in diesem Falle die eines Quads, der an den Unlit-Shader für die Berechnung des Billboards gegeben wird.

Das Rendering wird im Controller letztendlich mit der Funktion \href{https://docs.unity3d.com/ScriptReference/Graphics.DrawProceduralNow.html}{DrawProceduralNow} ausgelöst. Dieser bezieht seine Daten aus dem aktuell eingestellten ShaderPass (0), an dem das Shader-Material gebunden wurde. Die Partikeldaten werden somit direkt vom ComputeShader an den UnlitShader weitergereicht.

\begin{csh}[caption=Partikel Rendering]
mat.SetBuffer("_Particles", particlesBuf);
mat.SetBuffer("_QuadVert", quadVertBuf);
renderSettings.Uniform(mat);
mat.SetPass(0);

Graphics.DrawProceduralNow(MeshTopology.Triangles, meshVerts.Length, deadBuf.count);
\end{csh}

Der Vertex Shader berechnet zunächst von lebendigen Partikeln das Billboard - dh. es dreht die Vertices richtung Kamera. Dafür werden die Partikel zunächst durch die Kamera-View Matrix transformiert, dann die Vertex-Positionen addiert und schlie{\ss}lich durch die Projektionsmatrix richtung Kamera gedreht. Die UV-Koordinaten sowie die Farbdaten werden dann an den Fragment-Shader weitergegeben, der die Farbdaten der Textur mit der Partikelfarbe multipliziert.

\begin{hlsl}[caption=Unlit Billboard Vertex Shader]
v2f vert(uint vid : SV_VertexID, uint iid : SV_INSTANCEID)
{
    float3 vpos = float3(p.size.xy * _QuadVert[vid], 0);
    float4 ppos = float4(p.pos, 1);
    o.pos = mul(UNITY_MATRIX_P, float4(vpos, 1) + mul(UNITY_MATRIX_V, ppos));
    /* ... */
}
\end{hlsl}

% \\begin\{.*?\}[\s\S\n]*?\\end\{.*



\section{Postprocessing}

Postprocessing bezeichnet im Kontext der Videospielgrafik die Nachbearbeitung eines bereits ganz oder teilweise gerenderten Bildes mit einem oder mehreren Bildeffekten. Das Ziel des Postprocessings bei Cubix ist es, alte Videoaufnahme- und -wiedergabegeräte nachzubilden. Um diesen Effekt zu erzielen, werden Bildartefakte, die durch diese Hardware entstehen, als Nachbearbeitungseffekte dem fertigen Bild hinzugefügt.

Die verwendeten Effekte, die von alten Kameras verursacht werden, sind Bloom, Lens Flare, Chromatic Aberration und Vignette. 
Ein alter Monitor kann Scanlines verursachen, gewölbt sein und ebenfalls für Vignette und Chromatic Aberration sorgen.

Jeder der genannten Effekte wird in den späteren Abschnitten genauer erläutert. Man beachte, dass sowohl alte Displays als auch Linsen Vignette und Chromatic Aberration verursachen können, was dann zu kleinen Unterschieden im Auftreten dieser Effekte führen kann. Die Entscheidung, wie diese Effekte letztendlich darzustellen sind ist rein artistisch.



\subsection{Postprocessing - Pipeline}

Das Postprocessing wird über ein ebenso benanntes Skript gesteuert. Bloom und Lens Flare werden von einem jeweils eigenen Compute Shader durchgeführt, alle anderen Effekte werden in eigenen Passes eines Image Effect Shaders durchgeführt. Diese sind im Skript als Objekte angelegt:

\begin{csh}
    public ComputeShader bloom;
    public ComputeShader lensFlare;
    public Material postProcMat;
\end{csh}

Weiterhin werden für den Lens Flare eine Textur einer schmutzigen Linse und eine Starburst Textur benötigt:

\begin{csh}
    public Texture2D lensDirtTex;
    public Texture2D starburstTex;
\end{csh}

Zur Sequenziellen Ausführung aller Effekte werden insgesamt 6 RenderTextures angelegt:

\begin{csh}
    public RenderTexture sourceTex;
    public RenderTexture brightTex;
    public RenderTexture blurrBuff;
    public RenderTexture caResult;
    public RenderTexture lfResult;
    public RenderTexture lensDirt;
\end{csh}

Zum Start des Programms oder falls die Fenstergrö{\ss}e geändert wird, müssen alle diese RenderTextures nmit der Funktion createTexture() neu angelegt werden.

Durchgeführt wird das Postprocessing in der Funktion 
\begin{csh} 
private void OnRenderImage(RenderTexture source, RenderTexture destination) 
\end{csh}
Diese Funktion wird von der Unity Engine zur Verfügung gestellt und in einem Skript, das an eine Kamera angehängt ist, immer dann aufgerufen, wenn ein neues Bild gerendert wurde. Source ist dabei das gelieferte Bild und Destination ist das Bild, welches zum Schluss abgebildet wird. Source kann also bevor es nach Destination geschrieben wird noch beliebig verändert werden. Der erste Effekt, welcher auf das Bild angewendet wird, ist Bloom. Bloom ist als Compute Shader verfasst, für den Texturen, in die geschrieben werden soll, das Flag 'enableRandomWrite' benötigen. Dieses lässt sich für einmal erstellte Texturen im Nachhinein nicht mehr ändern und ist in Source standardmäßig deaktiviert. Also ist der erste Schritt, mit der Funktion 
\begin{csh}
Graphics.Blit(source, sourceTex);
\end{csh}
 den Inhalt von Source in die zuvor angelegte Textur sourceTex zu schreiben, für welche das benötigte Flag gesetzt ist.

%image source to source tex

Nun kann der Bloom-Effekt angewandt werden. Zuerst werden mithilfe des Bloomshaders alle hellen Stellen aus sourceTex nach brightTex geschrieben. Der Inhalt von brightTex wird nun verschwommen gemacht, wozu die Textur blurrBuff benötigt wird. Anschlie{\ss}end wird das verschwommene Resultat (brightTex) additiv zurück nach sourceTex geschrieben.

%image bloom

Als nächstes wird Chromatic Aberration auf das fertige Bild mit Bloom (sourceTex) angewandt. Es wird keine zusätzliche Textur benötigt und das Resultat befindet sich in caResult.

%image ca

Nun ist Lens Flare an der Reihe. Als ausgangstextur dient ebenfalls sourceTex und ähnlich wie bei Bloom werden bestimmte Features des Ursprungsbildes in eine separate Textur (lfResult) geschrieben. Diese wird dann wiederum mithilfe von blurrBuff verschwommen gemacht und dieses mal additiv auf caResult zurückgeschrieben. 

%image lf

Vorher jedoch wird das Ergebnis mit der Textur lensDirt kombiniert, die zum Start des Programms einmal aus den beiden Texturen lensDirtTex und starburstTex erstellt wird.

%image lensDirt

Die Effekte Vignette, Scanlines und Displaykrümmung können zuletzt alle mit einmal auf caResult angewendet werden, wobei das Ergebnis direkt nach destination (Argument von OnRenderImage) geschrieben wird.

%image crt

Somit hat das resultierende Bild alle gewünschten Effekte



\subsection{Blur}

Der Blur-Shader ist ein Fragment Shader und wendet einen einfachen Gau{\ss}-schen Weichzeichnungsalgorithmus auf eine Textur an. Benötigt wird ein Ursprungsbild und ein Puffer der selben Grö{\ss}e.



\subsection{Bloom}

\lipsum[3]



\subsection{Chromatic Aberration}

\lipsum[3]



\subsection{Lens Flare}

\lipsum[3]



\subsection{Vignette, warped Display und Scanlines}

\lipsum[3]






\chapter{Assets und Tools}

\begin{description}
\item[Git] Versionsverwaltungstool
\item[Unity Engine] Game engine
\item[Unity Editor] Editor der Game Engine
\item[Microsoft Visual Studio] Code Editor mit Unity Integration
\item[Pro Builder] Simpler Mesh-Editor für Unity zum Erstellen der Entity-Modelle
\item[Ableton Live 10] Digital Audio Workstation zur Erstellung der Soundeffekte
\vspace{5mm}
\item[Modelle] Selbst erstellt mit Pro Builder
\item[Texturen] In Unity angelegt, ausschlie{\ss}lich einfarbig
\item[Audioeffekte] Selbst erstellt mit Ableton Live
\item[Soundtrack] Selbst erstellt mit Ableton Live

\end{description}






\chapter{Ausblick}

% Bossstages, mehr testdaten & statistiken
% mehr Inhalt & Abwechslung (generischere Gegner & Stages, mehr presets & Gegnertypen)
\lipsum[3]






\chapter{Nutzerstatistiken}
\label{chapt:PlayerStats}

\lipsum[3]




\section{Auswertung}





\chapter{Fazit}

\lipsum[3]





\chapter{Glossar}

\begin{description}
\item[GUI] Graphical User Interface
\item[Prefab] Vorlage eines Unity-Objektes, welches beliebig oft in der Szene wiederverwendet werden kann ohne neu definiert werden zu müssen
\vspace{5mm}
\item[Shader] Programm, das auf der Grafikkarte ausgeführt wird
\item[Uniform] Variable, die aus dem Hauptprogramm aus für ein Shaderprogramm gesetz werden kann
\item[Compute Shader] Shader, der Kernels in mehreren Arbeitsgruppen ausführt
\item[Kernel] Bestandteil des Compute Shaders, enthält die Arbeitsanweisungen für die Arbeitsgruppen
\item[Billboard] Projektion eines Meshes auf ein Rechteck, welches immer richtung Kamera ausgerichtet wird
\vspace{5mm}
\item[Postprocessing] Nachbearbeitung (zum Beispiel eines Bildes)
\item[Bloom] Überblendung, helle Bereiche eines Bildes überstrahlen naheliegende Bereiche
\item[Lens Flare] Linsenreflexion, helle Bereiche eines Bildes erzeugen blasse Abbilder auf einer Linie in Richtung der Mitte das Bildes
\item[Ghosts] Begriff für besagte Abbilder des Lens Flare
\item[Chromatic Aberration] Chromatische Abweichung, Licht unterschiedlicher Wellenlängen wird in einem Bild unterschiedlich stark verzerrt
\item[Vignette] Abdunklung eines Bildes in der Nähe der Bildschirmränder
\item[Scanlines] Dunkle, horizontale Linien. Artefakt alter CRT Monitore
\end{description}




\chapter{Dateiverzeichnis}

\definecolor{cGPU}{rgb}{0,0,.8}
\definecolor{cGPR}{rgb}{0,.6,0}

\begin{flushleft}%
\parbox{0.5\linewidth}{
    \dirtree{%
    .1 \color{cGPR}Assets/.
    .2 \color{cGPR}Animation/.
    .2 \color{cGPR}Materials/. .3 \color{cGPU}Shader/.
    .2 \color{cGPR}Prefabs/.
    .3 \color{cGPR}Actors/. .4 \color{cGPR}Enemies/.
    .3 \color{cGPR}StageElements/.
    .3 \color{cGPR}Bases/.
    .2 \color{cGPR}Resources/.
    .2 \color{cGPR}Scenes/.
    .2 \color{cGPR}Scripts/.
    .3 \color{cGPU}Shader/.
    .3 \color{cGPR}Stages/.
    .3 \color{cGPR}Test/.
    .3 \color{cGPR}Actors/. .4 \color{cGPR}Player/. .4 \color{cGPR}Enemies/.
    .3 \color{cGPR}StageElements/.
    .3 \color{cGPR}Bases/.
    .3 \color{cGPR}Controller/.
    .2 \color{cGPU}Shader/. .3 \color{cGPU}Particles/. .3 \color{cGPU}PostProcessing/.
    }
}%
\parbox{0.4\linewidth}{
    {\color{cGPR} Game Programming} \\
    {\color{cGPU} GPU Programmierung}
}%
\end{flushleft}






\chapter{Quellenverzeichnis}

\begin{thebibliography}{999}

\bibitem{qNierHM} Nier:Automata Hacking-Mode Demo vom 23.04.2021,  Zugriff:  16.05.2021, \\ URL:
\url{https://www.youtube.com/watch?v=jT2jOeFo5HQ}

\bibitem{qCtxSteer} Context-Steering Demo vom 10.10.2020,  Zugriff:  19.05.2021, \\ URL:
\url{https://www.youtube.com/watch?v=6BrZryMz-ac}
\url{http://www.gameaipro.com/GameAIPro2/GameAIPro2_Chapter18_Context_Steering_Behavior-Driven_Steering_at_the_Macro_Scale.pdf}

\bibitem{qPartS1} Drawing Thousands of Meshes in Unity vom 01.11.2019, Zugriff 17.06.2021
\url{https://toqoz.fyi/thousands-of-meshes.html}

\bibitem{qPartS2} GPUParticles vom 22.12.2017, Zugriff 22.06.2021
\url{https://github.com/Robert-K/gpu-particles/tree/master/Assets/GPUParticles}

\end{thebibliography}





\newpage
\newpage
\chapter{Anhang}




\section{Installationsanleitung}

Laden Sie die Datei \href{https://www.dropbox.com/s/g82vexjznq1x9dd/Cubix.zip?dl=1}{Cubix.zip} herunter und entpacken Sie sie. Anschlie{\ss}end führen Sie die darin befindliche cubix.exe aus.
Im Falle einer Warnung vom Windows-Defender klicken Sie auf Mehr und Trotzdem ausführen - Die Warnung besagt nur, dass das Spiel von keinem offiziellen Hersteller veröffentlicht wurde.
Sie müssten nun einen Unity-Startbildschirm sehen und kurz darauf direkt im Spiel landen.




\section{Spielanleitung}

Du bist ein kleiner Würfel in einer Welt von Kugeln, Gegnern und Portalen. Versuche so viele Stufen wie möglich zu schaffen und dabei so wenige Fehler wie möglich zu machen. Wähle zwischen 3 Schussmodi aus, und wenn du in der Klemme steckst und genügend Ressourcen hast aktiviere den Farbkombinationsmodus und Rette damit deine Haut.



\subsection{Steuerung}
%TODO: Screenshots Spielfeld, Stageelemente, Farbschema, Pausemenü

Gesteuert wird mit WASD. Linksklick drücken und halten zum kontinuierlichen Abfeuern von Schüssen. Schussfarbe wechseln mit 1, 2 und 3 auf der Tastatur oder Numpad.

Wenn genügend Ressourcen vorhanden sind (mindestens 2 voll aufgeladene Balken) Drücke Space, um für eine Zeit lang kombinierte Farbschüsse abzugeben.

Pausiert wird das Spiel mit Escape.





\chapter{Eigenständigkeitserklärung}

Hiermit erklären wir Alexander Feilke und Jakom Massal{\ss}ky, dass wir die vorliegende Arbeit selbstständig verfasst und keine anderen als die angegebenen Quellen und Hilfsmittel benutzt haben.

Alle sinngemä{\ss} und wörtlich übernommenen Textstellen aus fremden Quellen wurden kenntlich gemacht.

Mittweida, den 10.08.2021

\vspace{2cm}

%\includegraphics[width=4cm]{u_afeilke1.png} \hspace{3cm} \includegraphics[width=4cm]{u_jmassals.png}

Alexander Feilke \hspace{4cm} Jakob Ma{\ss}alsky

\end{document}
