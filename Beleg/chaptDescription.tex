
\chapter{Spielbeschreibung}

%Konzept, ~farben
Cubix ist ein Top-Down Bullet-Hell Shooter angelehnt an den Hacking-Mode aus Nier:Automata\cite{qNierHM}.
Dort steuert man eine Figur in einer Welt von Gegnern und Hindernissen und muss möglichst schnell und fehlerfrei sämtliche Gegner durch gekonntes Ausweichen und gezielte Schüsse eliminieren.

Als Neuerung wird ein Farbkonzept verwendet, bei dem die Grundfarben Rot, Grün und Blau mit verschiedenen Eigenschaften assoziiert werden. Alle Akteure des Spiels (sowohl der Spieler selbst als auch die Gegner und aufsammelbare Objekte) können in beliebiger Kombination dieser Farben auftreten. Ist mehr als eine Farbe aktiv, so addieren sich deren Eigenschaften. Dieses System ist für den Spieler leicht verständlich, denn das Wissen über die Eigenschaften der drei Grundfarben reicht aus, um die Ergebnisse jeder möglichen Kombination schon im Voraus zu erschlie{\ss}en. Diese Kombinationsmöglichkeiten sorgen aber trotz des simplen Systems für genug Abwechslung und Tiefe. Die einzige Ausnahme beim Kombinieren bilden die sammelbaren Objekte, dazu später mehr (siehe \fullref{sect:Collectables}).




\section{Spielziel}

Das Spiel ist aus verschiedenen Stufen aufgebaut (siehe \fullref{sect:stages}), die vom Spieler abgeschlossen werden müssen.
Solange sich der Spieler im roten Ladebereich (Charger) in der Mitte des Spielfeldes aufhält, wird dieser aufgeladen.
Sobald der Charger das erste Mal aktiviert wird, fangen Gegner an zu erscheinen (spawnen). Diese können vom Spieler durch Schie{\ss}en besiegt werden. Sie selbst schie{\ss}en aber ggf. ebenfalls auf den Spieler und können ihm so schaden zufügen, bis er keine Leben mehr hat und die Stage von vorn beginnen muss.
Ist der Charger vollständig geladen, hören die Gegner auf zu spawnen und ein Portal zur nächsten Stufe wird aktiviert.




\section{Schussmodi}

Der Spieler kann seinen Schüssen jeweils eine der Eigenschaften der drei Grundfarben verleihen, welche er im Tutorial der Reihe nach freischalten und ausprobieren kann. Die ausgewählte Farbe kann jederzeit gewechselt werden:

\renewcommand{\itmspace}{4.5em}
\imgitmtxt{s_red.png}{Rote Kugeln}
{sind langsamer aber explodieren beim Aufprall.}

\imgitmtxt{s_green.png}{Grüne Kugeln}
{sind schneller und fügen mehr Schaden zu, prallen von Wänden ab und durchdringen Gegner, haben aber eine geringere Schussrate.}

\imgitmtxt{s_blue.png}{Blaue Kugeln}
{werden in einem Fächer von fünf Kugeln in einer höheren Schussrate verschossen, fügen aber weniger Schaden zu.}

Die gleichzeitige Kombination mehrerer Farben ist nur auf Kosten von Ressourcen erlaubt (siehe \fullref{sect:ressourcen}). Dabei erhalten die Kugeln alle Eigenschaften der aktiven Farben.

Beispiel: Kombiniert man die Farben rot und grün, so erhält man gelbe Kugeln, die etwas langsamer sind als Grüne, eine geringere Feuerrate haben und mehr Schaden zufügen. Eigenschaften wie das durchdringen von Gegnern, abprallen von Wänden und erzeugen von Explosion bei Kontakt bleiben erhalten. Bei den anderen Kombinationen (magenta, cyan) ist analog zu verfahren.

Eine Ausnahme gibt es bei blau: wird diese Farbe kombiniert werden statt fünf nur drei Kugeln abgefeuert, um den Bonus der anderen Farbe nicht durch die gro{\ss}e Anzahl von Kugeln übermä{\ss}ig zu steigern.


\section{Gegner}

Im Spiel gibt es drei verschiedene Gegnertypen, erkennbar an ihrer Form:

\renewcommand{\itmspace}{4.5em}
\imgitmtxt{e_hunter.png}{Hunter}
{jagen den Spieler um ihm Kollisionsschaden zuzufügen.}

\imgitmtxt{e_archer.png}{Archer}
{verfolgen den Spieler aber bleiben auf Abstand und schie{\ss}en aus der Ferne auf ihn.}

\imgitmtxt{e_stray.png}{Strays}
{irren langsam über das Spielfeld und feuern 4 Schüsse in alle Richtungen.}

Auch die Gegner können in drei verschiedenen Farben vorkommen:

\renewcommand{\itmspace}{7.5em}
\imgitmtxt{e_red.png}{Rote Gegner}
{sind langsam, haben dafür aber mehr Leben.}

\imgitmtxt{e_green.png}{Grüne Gegner}
{bewegen sich schneller.}

\imgitmtxt{e_blue.png}{Blaue Gegner}
{spawnen in Zweiergruppen, haben aber weniger Leben.}

Bei der Kombination der Gegnerfarben ist ebenfalls wie bei den Schussfarben vorzugehen. Beispiel:

\imgitmtxt{e_white.png}{Wei{\ss}e Gegner}
{spawnen als Kombination aller drei Farben und haben mehrere Boni. Sie sind so schnell wie grüne Gegner, haben etwas weniger Leben als Rote und spawnen ebenfalls in Gruppen.}




\section{Collectables}
\label{sect:Collectables}

Wenn Gegner besiegt werden, lassen sie zu einer gewissen Wahrscheinlichkeit aufsammelbare Objekte (Collectables) abhängig von ihrer eigenen Farbe fallen.

\renewcommand{\itmspace}{5.5em}
\imgitmtxt{c_black.png}{Schwarz}
{lässt den Spieler einen Lebenspunkt zurückerhalten.}

\imgitmtxt{c_blue.png}{Rot, Grün, Blau}
{verleihen dem Spieler Ressourcen in der entsprechenden Farbe.}

\imgitmtxt{c_yellow.png}{Gelb}
{lässt den Spieler für kurze Zeit in alle Richtungen feuern.}

\imgitmtxt{c_cyan.png}{Cyan}
{erhöht kurzzeitig die Schussrate stark.}

\imgitmtxt{c_magenta.png}{Magenta}
{verleiht für einen kurzen Zeitraum Unsichtbarkeit/Unverwundbarkeit.}

\imgitmtxt{c_gold.png}{Gold}
{lässt einen goldenen Container fallen der bei Kontakt eine gro{\ss}e Explosion auslöst, die sämtliche Gegner auf dem Spielfeld auslöscht. (Selten und nur von wei{\ss}en Gegnern erhältlich.)}




\section{Ressourcen}
\label{sect:ressourcen}

Für jede der drei Grundfarben existiert eine entsprechende Ressource, deren Stand in der Ressourcenanzeige oben rechts im Fenster angezeigt wird:

\includegraphics[height=100pt]{ressources.png}

Ressourcen erhält man durch das Aufsammeln von entsprechend gefärbten Collectables und das Besiegen von farbigen Gegnern. Die erhaltenen Ressourcen werden für mehrfarbige Gegner auf die entsprechenden Grundfarben aufgeteilt, so gibt zum Beispiel ein cyanfarbiger Gegner blaue und grüne Ressourcen.

Benutzen kann man Ressourcen, sobald die Balken von mindesten zwei Farben vollständig gefüllt sind. Drückt man nun die Taste zum Aktivieren der Farbkombination (Leertaste), so werden den Schüssen die Eigenschaften der aufgefüllten Farben verliehen und die entsprechenden Ressourcen leeren sich über ungefähr 10 Sekunden. Die Farbkombination bleibt so lange aufrecht erhalten, bis eine der benutzten Ressourcen leer ist. Während dieser Zeit kann die Schussfarbe nicht mehr manuell gewechselt werden, dafür ist eine Kombination aus mehreren Farben deutlich stärker als jede einzelne Farbe.

Diese stärkere Phase wird also durch die Ressourcen zeitlich begrenzt, wodurch sie sich für den Spieler besonders anfühlt und dieser darauf hinarbeiten kann, sie erneut zu erleben. Zusätzlich lassen sich die Ressourcen auch während sie aktiv sind, weiter aufladen (zum Beispiel durch das Töten von Gegnern), wodurch ein aggressiver, spannender Spielstil gefördert wird.



\section{Interface und Hauptmenü}

Um die Immersion zu steigern, ist das Interface auf das Nötigste beschränkt. Allein ein Knopf zum Pausieren und die Ressourcenanzeige bilden gemeinsam das GUI. Ist das Spiel pausiert, können über insgesamt 5 Knöpfe die grundlegendsten Funktionen wie 'Fortfahren', 'Spiel beenden' oder 'zurück ins Hauptmenü' aufgerufen werden. 
Das Hauptmenü selbst ist dabei kein wirkliches Menü, sondern bereits die erste Stufe. Anstatt wie üblich per Knopfdruck einen bestimmten Spielmodus auszuwählen (hier Tutorial oder Endlos), muss der Spieler erst den Charger aufladen und dann zum entsprechenden Portal navigieren. Gegner erscheinen hier noch keine. Hinweise für den Spieler, wie zum Beispiel die grundlegende Steuerung oder Informationen über Schussfarben und Gegner, erscheinen ebenfalls nicht im GUI, sondern als Text auf dem Boden. 

So ist das Hauptmenü schon stark mit dem Gameplay verbunden, es gibt einen nahtlosen Übergang zum eigentlichen Spiel und der Fokus wird auf das Wesentliche gesetzt.

Auf eine herkömmliche Lebensanzeige des Spielers wurde auch verzichtet, stattdessen zeigen kleine, um die Spielfigur kreisende Würfel die aktuellen Lebenspunkte an.

\begin{figure}[H]
\centering
\includegraphics[width=0.8\linewidth]{stage.png}
\caption{Spielansicht im Hauptmenü}
\label{img:stage}
\end{figure}


\section{Tutorial}

Das Tutorial besteht aus einer Folge von 10 Stufen (das Hauptmenü eingeschlossen), in denen der Spieler der Reihe nach mit den einzelnen Spielmechaniken konfrontiert wird. Ereignisgesteuerte Hinweistexte erklären neue Sachverhalte, sodass sich der Spieler mit ihnen vertraut machen kann. Das Tutorial ist wie folgt aufgebaut:

\begin{description}
\item[Hauptmenü] Steuerung, Spielmodusauswahl
\item[Stufe 1] Gegner, Schie{\ss}en
\item[Stufe 2] Erste Schussfarbe freischalten
\item[Stufe 3] Gegner in der freigschalteten Farbe erscheinen
\item[Stufe 4] Zweite Schussfarbe freischalten
\item[Stufe 5] Farbkombimnation erlaubt
\item[Stufe 6] Zweifarbige Gegner erscheinen
\item[Stufe 7] Dritte Schussfarbe freischalten
\item[Stufe 8] Möglichkeit zum Ausprobieren der neuen Farbkombinationen
\item[Stufe 9] Wei{\ss}e Gegner, letzte Herausforderung vor dem Endlosmodus
\item[Stufe 10+] Endlosmodus
\end{description}

Sobald Stufe 10 das erste Mal erreicht wurde, wird die Möglichkeit freigeschalten, den Endlosmodus vom Hauptmenü aus direkt zu starten und das Tutorial zu überspringen.
